
\chapter{Solutions for exercises to chapter 2}

\global\long\def\biggp#1{\bigg( #1 \bigg)}

\begin{cBoxA}[Problem 2.1 - Bernoulli distribution's expectation, variance, normalization,
entropy]{}
 Verify that the Bernoulli distribution (2.2) satisfies the following
properties 
\begin{align*}
\sum_{x=0}^{1}f(x\vert\mu) & =1,\\
\E[X] & =\mu,\\
\mathrm{Var}[X] & =\mu(1-\mu).
\end{align*}
Show that the entropy $H(X)$ of a Bernoulli distributed random binary
variable $X$ is given by 
\[
H(X)=-\mu\ln\mu-(1-\mu)\ln(1-\mu).
\]
\end{cBoxA}

In the discussion below, $X$ is a random variable following Bernoulli
distribution. 
\begin{enumerate}[leftmargin={*}]
\item  To ch`eck normalization, we note 
\[
\sum_{x=0}^{1}f_{X}(x\vert\mu)=\mu+(1-\mu)=1.
\]
\item To find the expectation, note that 
\[
\E[X]=\sum_{x=0}^{1}xf_{X}(x\vert\mu)=1\cdot\mu+0\cdot(1-\mu)=\mu.
\]
\item To find the variance, we note that 
\[
\text{Var}[X]=\E[X^{2}]-(\E X)^{2}=\mu-\mu^{2}=\mu(1-\mu).
\]
\item To find the entropy, we note that 
\[
H(X)=-\sum_{x=0}^{1}f_{X}(x\vert\mu)f\log f_{X}(x|\mu)=-\mu\log\mu-(1-\mu)\log1-\mu.
\]
\end{enumerate}
%
\begin{cBoxA}[Problem 2.2 - Symmetric Bernoulli distribution's expectation, variance,
normalization, entropy]{}
 The form of the Bernoulli distribution given by (2.2) is not symmetric
between the two values of $X$. In some situations, it will be more
convenient to use an equivalent formulation for which $X\in\{-1,1\}$,
in which case the distribution can be written
\end{cBoxA}

\begin{enumerate}[leftmargin={*}]
\item  To show it's normalized, we note
\[
\sum_{x\in\{-1,1\}}f_{X}(x|\mu)=\left(\frac{1-\mu}{2}\right)^{2/2}\left(\frac{1+\mu}{2}\right)^{0}+\left(\frac{1-\mu}{2}\right)^{0}\left(\frac{1+\mu}{2}\right)^{1}=1.
\]
\item To find its expectation, we note 
\[
\E[X]=\sum_{x\in\{-1,1\}}xf_{X}(x|\mu)=\left(\frac{1+\mu}{2}\right)-\left(\frac{1-\mu}{2}\right)=\mu.
\]
\item To find its variance, we note 
\[
\text{Var}[X]=\E[X^{2}]-(\E[X])^{2}=\left(\frac{1+\mu}{2}\right)+\left(\frac{1-\mu}{2}\right)-\mu^{2}=1-\mu^{2}.
\]
\item To find its entropy, we note 
\[
H(X)=-\sum_{x\in\{-1,1\}}f_{X}(x|\mu)\log f_{X}(x|\mu)=-\left(\frac{1-\mu}{2}\right)\log\frac{1-\mu}{2}-\left(\frac{1+\mu}{2}\right)\log\frac{1+\mu}{2}.
\]
\end{enumerate}
\begin{cBoxA}[Problem 2.3 - Binomial distribution is normalized]{}
 In this exercise, we prove that the binomial distribution (2.9)
is normalized. First use the definition (2.10) of the number of combinations
of $m$ identical objects chosen from a total of $N$ to show that
\[
\binom{N}{m}+\binom{N}{m-1}=\binom{N+1}{m}.
\]
Use this result to prove by induction the following result
\[
(1+x)^{N}=\sum_{m=0}^{N}\binom{N}{m}x^{m},
\]
which is known as the binomial theorem, and which is valid for all
real values of $x.$ Finally, show that the binomial distribution
is normalized, so that 
\[
\sum_{m=0}^{N}\binom{N}{m}\mu^{m}(1-\mu)^{N-m}=1,
\]
which can be done by first pulling out a factor $(1-\mu)^{N}$ out
of the summation and then making use of the binomial theorem. 
\end{cBoxA}

\begin{enumerate}[leftmargin={*}]
\item  First, we show Eq.(2.262) holds: note that 
\begin{align*}
\binom{N}{m}+\binom{N}{m-1} & =\frac{N!}{m!(N-m)!}+\frac{N!}{(m-1)!(N-m+1)!}\\
 & =\frac{N!(N-m+1)}{m!(N-m+1)!}+\frac{mN!}{m!(N-m+1)!}\\
 & =\frac{(N+1)!}{m!((N+1)-m)!}\\
 & =\binom{N+1}{m}.
\end{align*}
\item To prove the binomial theorem, we induce on $N.$ For the base case
$N=1$ and $0$, it is trivially true:
\begin{align*}
(1+x)^{1} & =\binom{1}{0}x^{0}+\binom{1}{1}x^{1}=1+x,\\
(1+x)^{0} & =\binom{0}{0}x^{0}=1.
\end{align*}
Now suppose the claim holds for $N=k.$ Then for $N=k+1$ we have
\begin{align*}
(1+x)^{k+1} & =(1+x)(1+x)^{k}=(1+x)\sum_{m=0}^{N}\binom{N}{m}x^{m}\\
 & =\sum_{m=0}^{M}\binom{N}{m}x^{m}+\sum_{m=0}^{N}\binom{N}{m}x^{m+1}\\
 & =\binom{N}{0}x^{0}+\sum_{m=1}^{M}\binom{N}{m}x^{m}+\sum_{m=1}^{N}\binom{N}{m-1}x^{m}+\binom{N+1}{N+1}x^{N+1}\\
 & =\binom{N+1}{0}x^{0}+\sum_{m=1}^{M}\left(\binom{N}{m}+\binom{N}{m-1}\right)x^{m}+\binom{N+1}{N+1}x^{N+1}\\
 & =\binom{N+1}{0}x^{0}+\sum_{m=1}^{M}\binom{N+1}{m}x^{m}+\binom{N+1}{N+1}x^{N+1}\\
 & =\sum_{m=0}^{N+1}\binom{N}{m}x^{m}.
\end{align*}
\item Now to show that the binomial distribution is normalized, we note
that 
\begin{align*}
\sum_{m=0}^{N}\binom{N}{m}\mu^{m}(1-\mu)^{N-m} & =(1-\mu)^{N}\sum_{m=0}^{N}\binom{N}{m}\mu^{m}(1-\mu)^{-m}\\
 & =(1-\mu)^{N}\sum_{m=0}^{N}\binom{N}{m}\left(\frac{\mu}{1-\mu}\right)^{m}\\
 & =(1-\mu)^{N}\left(1+\frac{\mu}{1-\mu}\right)^{N}.\tag{by binomial theorem}\\
 & =\left[(1-\mu)\left(1+\frac{\mu}{1-\mu}\right)\right]^{N}
\end{align*}
Since 
\begin{align*}
(1-\mu)\left(1+\frac{\mu}{1-\mu}\right) & =1+\frac{\mu}{1-\mu}-\mu-\frac{\mu^{2}}{1-\mu}\\
 & =1+\frac{\mu-\mu+\mu^{2}-\mu^{2}}{1-\mu}\\
 & =1,
\end{align*}
it follows that $\sum_{m=0}^{N}\binom{N}{m}\mu^{m}(1-\mu)^{N-m}=1,$
and thus the result follows. 
\end{enumerate}
%
\begin{cBoxA}[Problem 2.4 - Binomial distribution's expectation and variance ]{}
 Show that the mean of the binomial distribution is given by (2.11).
To do this, differentiate both sides of the normalization condition
(2.264) with respect to $\mu$ and then rearrange to obtain an expression
for the mean of $n$. Similarly, by differentiating (2.264) twice
with respect to $\mu$ and making use of the result (2.11) for the
mean of the binomial distribution prove the result (2.12) for the
variance of the binomial. 
\end{cBoxA}

\begin{enumerate}[leftmargin={*}]
\item  Following the hint, we differentiate Eq.(2.264) w.r.t $\mu$ once:
\begin{align*}
\frac{\partial}{\partial\mu}\left\{ \sum_{n=0}^{N}\binom{N}{n}\mu^{n}(1-\mu)^{N-n}\right\} = & n\cdot\sum_{n=1}^{N-1}\binom{N}{n}\mu^{n-1}(1-\mu)^{N-n}-(N-n)\cdot\sum_{n=1}^{N-1}\binom{N}{n}\mu^{n}(1-\mu)^{N-n-1}\\
 & -N(1-\mu)^{N-1}+N\mu^{N-1}\\
= & n\cdot\sum_{n=1}^{N}\binom{N}{n}\mu^{n-1}(1-\mu)^{N-n}-(N-n)\cdot\sum_{n=0}^{N-1}\binom{N}{n}\mu^{n}(1-\mu)^{N-n-1}\\
= & n\cdot\sum_{n=0}^{N}\binom{N}{n}\mu^{n-1}(1-\mu)^{N-n}-(N-n)\cdot\sum_{n=0}^{N}\binom{N}{n}\mu^{n}(1-\mu)^{N-n-1}\\
= & \frac{n}{\mu}\cdot\sum_{n=0}^{N}\binom{N}{n}\mu^{n}(1-\mu)^{N-n}-\frac{N-n}{1-\mu}\cdot\sum_{n=0}^{N}\binom{N}{n}\mu^{n}(1-\mu)^{N-n}\\
= & \sum_{n=0}^{N}\binom{N}{n}\mu^{n}(1-\mu)^{N-n}\left(\frac{n}{\mu}-\frac{N-n}{1-\mu}\right).
\end{align*}
Since $\sum_{n=0}^{N}\binom{N}{n}\mu^{n}(1-\mu)^{N-n}=1,$ it follows
that 
\begin{align*}
\sum_{n=0}^{N}\binom{N}{n}\mu^{n}(1-\mu)^{N-n}\left[\frac{n}{\mu}-\frac{N-n}{1-\mu}\right]=0 & \iff\sum_{n=0}^{N}\binom{N}{n}\mu^{n}(1-\mu)^{N-n}\left[\frac{n}{\mu}-\frac{N-n}{1-\mu}\right][\mu(1-\mu)]=0\\
 & \iff\sum_{n=0}^{N}\binom{N}{n}\mu^{n}(1-\mu)^{N-n}[n(1-\mu)-(N-n)\mu]=0\\
 & \iff\sum_{n=0}^{N}\binom{N}{n}\mu^{n}(1-\mu)^{N-n}(n-N\mu)=0.\tag{1}
\end{align*}
Now we rearrange Eq.(1):
\[
\sum_{n=0}^{N}n\cdot\binom{N}{n}\mu^{n}(1-\mu)^{N-n}=N\mu\left(\sum_{n=0}^{N}\binom{N}{n}\mu^{n}(1-\mu)^{N-n}\right)=N\mu.
\]
The result follows by observing that 
\[
\E[X]=\sum_{n=0}^{N}n\cdot\binom{N}{n}\mu^{n}(1-\mu)^{N-n}
\]
\item To facilitate notation, we let $\varphi(\mu)=\sum_{n=1}^{N}\binom{N}{n}\mu^{n}(1-\mu)^{N-n}\left[\frac{n}{\mu}-\frac{N-n}{1-\mu}\right].$
Then following the hint, we differentiate twice Eq.(2.264) w.r.t.
$\mu$ and get 
\begin{align*}
\frac{\partial^{2}}{\partial\mu^{2}}\left\{ \sum_{n=0}^{N}\binom{N}{n}\mu^{n}(1-\mu)^{N-n}\right\}  & =\frac{\partial\varphi(\mu)}{\partial\mu}\\
 & =\sum_{n=0}^{N}\underbrace{\frac{\partial}{\partial\mu}\left\{ \binom{N}{n}\mu^{n}(1-\mu)^{N-n}\left(\frac{n}{\mu}-\frac{N-n}{1-\mu}\right)\right\} }_{:=H(\mu).}.
\end{align*}
Hence, it suffices to evaluate $H(\mu)$
\begin{align*}
H(\mu) & =\frac{\partial}{\partial\mu}\left\{ \binom{N}{n}\mu^{n}(1-\mu)^{N-n}\right\} \left(\frac{n}{\mu}-\frac{N-n}{1-\mu}\right)+\binom{N}{n}\mu^{n}(1-\mu)^{N-n}\frac{\partial}{\partial\mu}\left\{ \frac{n}{\mu}-\frac{N-n}{1-\mu}\right\} \\
 & =\binom{N}{n}\mu^{n}(1-\mu)^{N-n}\left(\frac{n}{\mu}-\frac{N-n}{1-\mu}\right)^{2}+\binom{N}{n}\mu^{n}(1-\mu)^{N-n}\left[-\frac{N-n}{(1-\mu)^{2}}-\frac{n}{\mu^{2}}\right]\\
 & =\binom{N}{n}\mu^{n}(1-\mu)^{N-n}\left[\left(\frac{n}{\mu}-\frac{N-n}{1-\mu}\right)^{2}-\frac{N-n}{(1-\mu)^{2}}-\frac{n}{\mu^{2}}\right].
\end{align*}
Hence, it follows that 
\[
\frac{\partial^{2}}{\partial\mu^{2}}\left\{ \sum_{n=0}^{N}\binom{N}{n}\mu^{n}(1-\mu)^{N-n}\right\} =\underbrace{\sum_{n=0}^{N}\binom{N}{n}\mu^{n}(1-\mu)^{N-n}\left[\left(\frac{n}{\mu}-\frac{N-n}{1-\mu}\right)^{2}-\frac{N-n}{(1-\mu)^{2}}-\frac{n}{\mu^{2}}\right]=0}_{(2)}.
\]
Now, we arrange Eq.(2) and get 
\begin{align*}
 & \sum_{n=0}^{N}\binom{N}{n}\mu^{n}(1-\mu)^{N-n}\left[\left(\frac{n}{\mu}-\frac{N-n}{1-\mu}\right)^{2}-\frac{N-n}{(1-\mu)^{2}}-\frac{n}{\mu^{2}}\right]=0\\
\iff & \sum_{n=0}^{N}\binom{N}{n}\mu^{n}(1-\mu)^{N-n}\left[\left(\frac{n}{\mu}-\frac{N-n}{1-\mu}\right)^{2}-\frac{N-n}{(1-\mu)^{2}}-\frac{n}{\mu^{2}}\right](\mu^{2}(1-\mu)^{2})=0\\
\iff & \sum_{n=0}^{N}\binom{N}{n}\mu^{n}(1-\mu)^{N-n}\left[(n(1-\mu)-(N-n)\mu)^{2}-(N-n)\mu^{2}-n(1-\mu)^{2}\right]=0\\
\iff & \sum_{n=0}^{N}\binom{N}{n}\mu^{n}(1-\mu)^{N-n}\left[(n-N\mu)^{2}-(N-n)\mu^{2}-n(1-\mu)^{2}\right]=0\\
\iff & \sum_{n=0}^{N}\binom{N}{n}\mu^{n}(1-\mu)^{N-n}(n-N\mu)^{2}=\sum_{n=0}^{N}\binom{N}{n}\mu^{n}(1-\mu)^{N-n}\left[(N-n)\mu^{2}+n(1-\mu)^{2}\right]\\
\iff & \sum_{n=0}^{N}\binom{N}{n}\mu^{n}(1-\mu)^{N-n}(n-N\mu)^{2}=\sum_{n=0}^{N}\binom{N}{n}\mu^{n}(1-\mu)^{N-n}(N\mu^{2}+n-2n\mu)\\
\iff & \sum_{n=0}^{N}\binom{N}{n}\mu^{n}(1-\mu)^{N-n}(n-N\mu)^{2}=N\mu-N\mu^{2}=N\mu(1-\mu).
\end{align*}
The conclusion can be drawn by observing that
\[
\text{Var}[X]=\E[(X-\E[X])^{2}]=\sum_{n=0}^{N}\sum_{n=0}^{N}\binom{N}{n}\mu^{n}(1-\mu)^{N-n}(n-N\mu)^{2}.
\]
\end{enumerate}
\begin{cBoxA}[Problem 2.5 - Beta distribution is normalized ]{}
 In this exercise, we prove that the beta distribution, given by
(2.13), is correctly normalized, so that (2.14) holds. This is equivalent
to showing that 
\[
\int_{0}^{1}\mu^{a-1}(1-\mu)^{b-1}d\mu=\frac{\Gamma(a)\Gamma(b)}{\Gamma(a+b)}.
\]
From the definition (1.141) of the gamma function, we have 
\[
\Gamma(a)\Gamma(b)=\int_{0}^{\infty}\exp(-x)x^{a-1}dx\int_{0}^{\infty}\exp(-y)y^{b-1}dy.
\]
Use this expression to prove (2.265) as follows. First bring the integral
over $y$ inside the integrand of the integral of $x$, next make
the change of variable $t=y+x$, where $x$ is fixed, then interchange
the order of the $x$ and $t$ integrations, and finally make the
change of variable $x=t\mu$ where $t$ is fixed. 
\end{cBoxA}

First, we note that 
\begin{align*}
\Gamma(a)\Gamma(b) & =\int_{0}^{\infty}e^{-x}x^{a-1}dx\int_{0}^{\infty}e^{-y}y^{b-1}dy\\
 & =\int_{0}^{\infty}\int_{0}^{\infty}e^{-(x+y)}x^{a-1}y^{b-1}dydx.\tag{1}
\end{align*}
Now, we make a change of variable 
\[
x+y=t\implies\begin{cases}
y=t-x\\
y\geq0\Leftrightarrow t-x\geq0\Leftrightarrow t\geq x\\
x\geq0\\
dt=dy
\end{cases}.
\]
Therefore, it follows that 
\begin{align*}
\text{Eq.(1)} & =\int_{0}^{\infty}\int_{x}^{\infty}e^{-t}x^{a-1}(t-x)^{b-1}dtdx\\
 & =\int_{0}^{\infty}\int_{0}^{t}e^{-t}x^{a-1}(t-x)^{b-1}dxdt\tag{by Fubini's theorem}\\
 & =\int_{0}^{\infty}\int_{0}^{1}e^{-t}(t\mu)^{a-1}(t-t\mu)^{b-1}td\mu dt\tag{2}\\
 & =\int_{0}^{\infty}e^{-t}t^{a-1}t^{b-1}tdt\int_{0}^{1}\mu^{a-1}(1-\mu)^{b-1}d\mu\\
 & =\Gamma(a+b)\int_{0}^{1}\mu^{a-1}(1-\mu)^{b-1}d\mu,
\end{align*}
where $\text{Eq.(2)}$ follows from a change of variables 
\[
x=t\mu\implies\begin{cases}
0\leq x\leq t\Leftrightarrow0\leq t\mu\leq t\Leftrightarrow0\leq\mu\leq t\\
dx=td\mu
\end{cases}.
\]
Hence, it follows that 
\[
\int_{0}^{1}\mu^{a-1}(1-\mu)^{b-1}d\mu=\frac{\Gamma(a)\Gamma(b)}{\Gamma(a+b)},
\]
and as a result the Beta density integrates to 1. \\

\begin{cBoxA}[Problem 2.6 - Beta distribution's expectation, variance, mode ]{}
 Make use of the result (2.265) to show that the mean, variance,
and mode of the beta distribution (2.13) are given respectively by
\begin{align*}
\E[\mu] & =\frac{a}{a+b},\\
\mathrm{Var}[\mu] & =\frac{ab}{(a+b)^{2}(a+b+1)},\\
\mathrm{mode}[\mu] & =\frac{a-1}{a+b-1}.
\end{align*}
\end{cBoxA}

\textit{In the discussion below, let $X$ be a random variable that
follows Beta distribution with parameter $a,b\in\mathbb{R}^{+}.$}
\begin{enumerate}[leftmargin={*}]
\item  To find the expectation, note that 
\begin{align*}
\E[X] & =\int_{0}^{1}\frac{\Gamma(a+b)}{\Gamma(a)\Gamma(b)}x^{a-1}(1-x)^{b-1}xdx\\
 & =\frac{\Gamma(a+b)}{\Gamma(a)\Gamma(b)}\int_{0}^{1}x^{(a+1)-1}(1-x)^{b-1}dx\\
 & =\frac{\Gamma(a+b)}{\Gamma(a)\Gamma(b)}\frac{\Gamma(a+1)\Gamma(b)}{\Gamma(a+b+1)}\tag{by Problem 2.5}\\
 & =\frac{\Gamma(a+b)a\Gamma(a)\Gamma(b)}{\Gamma(a)\Gamma(b)\Gamma(a+b)\Gamma(a+b)}\tag{since \ensuremath{\Gamma(x+1)=x\Gamma(x).}}\\
 & =\frac{a}{a+b}.
\end{align*}
\item To find the variance, we first note 
\begin{align*}
\E[X^{2}] & =\int\frac{\Gamma(a+b)}{\Gamma(a)\Gamma(b)}x^{a-1}(1-x)^{b-1}x^{2}dx\\
 & =\frac{\Gamma(a+b)}{\Gamma(a)\Gamma(b)}\int_{0}^{1}x^{a+2-1}(1-x)^{b-1}dx\\
 & =\frac{\Gamma(a+b)}{\Gamma(a)\Gamma(b)}\frac{\Gamma(a+2)\Gamma(b)}{\Gamma(a+b+2)}\\
 & =\frac{a(a+1)}{(a+b+1)(a+b)}.
\end{align*}
Then it follows that 
\begin{align*}
\text{Var}[X] & =\E[X^{2}]-(\E[X])^{2}=\frac{a(a+1)}{(a+b+1)(a+b)}-\left(\frac{a}{a+b}\right)^{2}\\
 & =\frac{a(a+1)(a+b)-a^{2}(a+b+1)}{(a+b+1)(a+b)^{2}}\\
 & =\frac{ab}{(a+b+1)(a+b)^{2}}.
\end{align*}
\item Since the mode of a continuous probability distribution is defined
as its density function's critical point, it suffices for us to differentiate
$f_{X}(x)$ and find the critical points. Note that 
\begin{align*}
\frac{\partial f_{X}(x)}{\partial x} & =\frac{\partial}{\partial x}\left[\frac{\Gamma(a+b)}{\Gamma(a)\Gamma(b)}x^{a-1}(1-x)^{b-1}\right]\\
 & =\frac{\Gamma(a+b)}{\Gamma(a)\Gamma(b)}\left[(a-1)x^{a-2}(1-x)^{b-1}+(b-1)x^{a-1}(1-x)^{b-2}\right].
\end{align*}
Setting it to zero yields 
\begin{align*}
 & \frac{\Gamma(a+b)}{\Gamma(a)\Gamma(b)}\left[(a-1)x^{a-2}(1-x)^{b-1}+(b-1)x^{a-1}(1-x)^{b-2}\right]=0\\
\iff & \left[(a-1)x^{a-2}(1-x)^{b-1}+(b-1)x^{a-1}(1-x)^{b-2}\right]=0\\
\iff & (a-1)(1-x)=(b-1)x\\
\iff & x=\frac{a-1}{a+b-2}.
\end{align*}
\end{enumerate}
\begin{cBoxA}[Problem 2.7 - Comparison between posterior mean and MLE for Bernoulli
model ]{}
 Consider a binomial random variable $X$ given by (2.9), with prior
distribution for $\mu$ given by the beta distribution (2.13), and
suppose we have observed m occurrences of $X=1$ and $l$ occurrences
of $X=0$. Show that the posterior mean value of $X$ lies between
the prior mean and the maximum likelihood estimate for $\mu$. To
do this, show that the posterior mean can be written as $\lambda$
times the prior mean plus $(1-\lambda)$ times the maximum likelihood
estimate, where $0\leq\lambda\leq1$. This illustrates the concept
of the posterior distribution being a compromise between the prior
distribution and the maximum likelihood solution.
\end{cBoxA}

\textit{The book didn't go through the details of deriving some of
the calculations. Although these calculations are simple, they are
worth doing by hand at least once. Hence, we show them here. For notation,
we let $\mathcal{X}$ denote the sample data, $(x_{1},\ldots x_{N})$.
\medskip}\\
First, we find the posterior mean for the Bernoulli model. By assumption,
the parameter of interest, $\mu,$ follows beta distribution, i.e.
\[
f(\mu|a,b)=\frac{\Gamma(a+b)}{\Gamma(a)\Gamma(b)}\mu^{a-1}(1-\mu)^{b-1}.
\]
And the likelihood function after sampling the data is given by 
\[
f(\mathcal{X}\vert\mu)=\mu^{\sum_{i=1}^{N}x_{i}}(1-\mu)^{\sum_{i=1}^{N}(1-x_{i})}=\mu^{n}(1-\mu)^{m}.
\]
Therefore, we have the posterior as 
\begin{align*}
f(\mu|\vert\mathcal{X}) & \propto f(\mu\vert a,b)\cdot f(x_{1},\ldots,x_{N}\vert\mu)\\
 & =\frac{\Gamma(a+b)}{\Gamma(a)\Gamma(b)}\mu^{a-1}(1-\mu)^{b-1}\mu^{n}(1-\mu)^{m}\\
 & \propto\mu^{a+n-1}(1-\mu)^{b+m-1}.
\end{align*}
Since $f(\mu|\mathcal{X})$ should integrates to 1 in order to be
a valid probability density function, in view of Problem 2.5 we see
that 
\[
f(\mu\vert\mathcal{X})=\frac{\Gamma(a+b+n+m)}{\Gamma(a)\Gamma(b)}\mu^{a+n-1}(1-\mu)^{b+m-1}\sim\text{Beta}(a+n,b+m).
\]
Hence, it follows that 
\[
\E_{\mu|\mathcal{X}}[\mu]=\frac{a+n}{a+b+n+m}
\]
as desired. \medskip\\
Next, we find $\mu_{MLE}.$ First, we write out the likelihood equation,
\[
f(\mathcal{X}|\mu)=\prod_{i=1}^{N}\mu^{x_{i}}(1-\mu)^{1-x_{i}}=\mu^{\sum_{i=1}^{N}x_{i}}(1-\mu)^{\sum_{i=1}^{N}(1-x_{i})},
\]
from which we can get the log-likelihood equation as 
\[
\ell(\mu)=\log f(\mathcal{X}\vert\mu)=\left(\sum_{i=1}^{N}x_{i}\right)\log\mu+\left(\sum_{i=1}^{N}(1-x_{i})\right)\log(1-\mu).
\]
Now we differentiate and set to zero 
\begin{align*}
 & \frac{\partial\ell(\mu)}{\partial\mu}=\left(\sum_{i=1}^{N}x_{i}\right)\frac{1}{\mu}-\left(\sum_{i=1}^{N}(1-x_{i})\right)\frac{1}{1-\mu}=0\\
\iff & \left(\sum_{i=1}^{N}x_{i}\right)(1-\mu)-\left(\sum_{i=1}^{N}(1-x_{i})\right)\mu=0\\
\iff & \frac{1}{\mu}=\frac{\sum_{i=1}^{N}(1-x_{i})}{\sum_{i=1}^{N}x_{i}}+1=\frac{\sum_{i=1}^{N}1-\sum_{i=1}^{N}x_{i}+\sum_{i=1}^{N}x_{i}}{\sum_{i=1}^{N}x_{i}}\\
\iff & \mu_{MLE}=\frac{n}{n+m}.
\end{align*}
Now it suffices to show that 
\[
\frac{a+n}{a+b+n+m}\in\text{Seg}\left(\frac{a}{a+b},\frac{n}{n+m}\right),
\]
where Seg means the line segment whose endpoints are $a/(a+b)$ and
$n/(n+m)$. To show this, it suffices to show that the solution, denoted
as $\lambda_{*}$, to the equation 
\[
\lambda\left(\frac{a}{a+b}\right)+(1-\lambda)\frac{n}{n+m}=\frac{a+n}{a+b+n+m}
\]
lies in $(0,1).$ Solving the equation yields 
\[
\lambda_{*}=\frac{a+b}{a+b+m+n}.
\]
Then the claim is true since $a,b,n,m>0$ by assumption. \\

\begin{cBoxA}[Problem 2.9 - Dirichlet distribution is normalized ]{}
 In this exercise, we prove the normalization of the Dirichlet distribution
(2.38) using induction. We have already shown in Exercise 2.5 that
the beta distribution, which is a special case of the Dirichlet for
$M=2$, is normalized. We now assume that the Dirichlet distribution
is normalized for $M-1$ variables and prove that it is normalized
for $M$ variables. To do this, consider the Dirichlet distribution
over $M$ variables, and take account of the constraint $\sum_{k=1}^{M}\mu_{k}$
by eliminating $\mu_{m}$ , so that the Dirichlet is written 
\[
p_{M}(\mu_{1},\ldots,\mu_{M-1})=C_{M}\prod_{k=1}^{M-1}\mu_{k}^{\alpha_{k}-1}\bigg(1-\sum_{j=1}^{M-1}\mu_{j}\bigg)^{\alpha_{M}-1}
\]
and out goal is to find an expression for $C_{M}.$ To do this, integrate
over $\mu_{M-1}$, taking care over the limits of integration, and
then make a change of variable so that this integral has limits 0
and 1. By assuming the correct result for $C_{M-1}$ and making use
of (2.265), derive the expression for $C_{M}$ .
\end{cBoxA}

\textit{In the discussion below, we let $f_{D}(\mu)$ denote the density
function for a Dirichlet distribution whose parameter $\mu$ is in
$K$ dimensional Euclidean space. We will use a slightly different
approach from the one derived from the hint from the book.\medskip}\\
We need to show that 
\[
\int_{\mathbb{S}_{K}}f_{D}(\mu)d\mu=\int_{\mathbb{S}_{K}}\frac{\Gamma(\sum_{i=1}^{K}\alpha_{i})}{\prod_{i=1}^{K}\Gamma(\alpha_{i})}\prod_{i=1}^{K-1}\mu_{i}^{\alpha_{i}-1}\left(1-\sum_{i=1}^{K-1}\mu_{i}\right)^{\alpha_{K}-1}d\mu=1,
\]
where $\mathbb{S}_{k}:=\{x\in\mathbb{R}^{k}:\sum_{i=1}^{k}x_{k}=1,x_{i}\geq0,i=0,...,k\}$
is the $k$-simplex in Euclidean space. Following the idea in Problem
2.5, it suffices for us to show that 
\[
I_{\mu}(k):=\int_{\mathbb{S}_{k}}\prod_{i=1}^{k-1}\mu_{i}^{\alpha_{i}}\left(1-\sum_{i=1}^{k-1}\mu_{i}\right)^{\alpha_{k}-1}d\mu=\frac{\prod_{i=1}^{k}\Gamma(\alpha_{i})}{\Gamma(\sum_{i=1}^{k}\alpha_{i})},
\]
for any $\mathbb{N}\ni k\geq2.$ We prove this using inducting on
$k$. For the base case $k=2$, note 
\begin{align*}
I_{\mu}(2) & =\int_{\{\mu\in\mathbb{R}^{2}:\mu_{1}+\mu_{2}=1,\mu_{1}\geq0,\mu\geq0\}}\mu_{1}^{\alpha_{1}-1}(1-\mu_{1})^{\alpha_{2}-1}d\mu\\
 & =\int_{\{\mu\in\mathbb{R}^{2}:\mu_{1}\times\mu_{2}\in[0.1]\times[0,1]\}}\mu_{1}^{\alpha_{1}-1}(1-\mu_{1})^{\alpha_{2}-1}d\mu\tag{1}\\
 & =\int_{0}^{1}d\mu_{2}\int_{0}^{1}\mu_{1}^{\alpha_{1}-1}(1-\mu_{1})^{\alpha_{2}-1}d\mu_{1}\tag{by Fubini's theorem }\\
 & =\frac{\Gamma(a_{1})\Gamma(\alpha_{2})}{\Gamma(\alpha_{1}+\alpha_{2})}.
\end{align*}
where Eq.(1) follows from the observation that for any $\mathbb{N}\ni k\geq2$
\begin{align*}
\mathbb{S}_{k} & =\left\{ x\in\mathbb{R}^{k}:\sum_{i=1}^{k-1}x_{i}=1-x_{k},x_{k}\in[0,1],x_{i}\geq0,i=1,...,k\right\} \\
 & =\left\{ x\in\mathbb{R}^{k}:\sum_{i=1}^{k-1}x_{i}\leq1,x_{k}\in[0,1],x_{i}\geq0,i=1,...,k-1\right\} ,
\end{align*}
where the equality can be verified by an element trace. Now assume
the claim is true for $k=n.$ Before going into the inductive step,
we carefully formulate the inductive hypothesis: note that 
\begin{align*}
I_{\mu}(n) & =\int_{\mathbb{S}_{n}}\prod_{i=1}^{n-1}\mu_{i}^{\alpha_{i}-1}\left(1-\sum_{i=1}^{n-1}\mu_{i}\right)^{\alpha_{n}-1}d\mu\\
 & =\int_{\left\{ \mu\in\mathbb{R}^{n}:\sum_{i=1}^{n-1}\mu_{i}\leq1,\mu_{n}\in[0,1],\mu_{1\leq i\leq n-1}\geq0\right\} }\prod_{i=1}^{n-1}\mu_{i}^{\alpha_{i}-1}\left(1-\sum_{i=1}^{n-1}\mu_{i}\right)^{\alpha_{n}-1}d\mu\\
 & =\int_{\{\mu\in\mathbb{R}^{n}:\sum_{i=1}^{n-1}\mu_{i}\leq1,\mu_{1\leq i\leq n-1}\in[0,1]\}}\int_{0}^{1}\prod_{i=1}^{n-1}\mu_{i}^{\alpha_{i}-1}\left(1-\sum_{i=1}^{n-1}\mu_{i}\right)^{\alpha_{n}}d\mu_{n}d(\times_{i=1}^{n-1}\mu_{i})\\
 & =\int_{\{\mu\in\mathbb{R}^{n}:\sum_{i=1}^{n-1}\mu_{i}\leq1,\mu_{1\leq i\leq n-1}\in[0,1]\}}\prod_{i=1}^{n-1}\mu_{i}^{\alpha_{i}-1}\left(1-\sum_{i=1}^{n-1}\mu_{i}\right)^{\alpha_{n}}d(\times_{i=1}^{n-1}\mu_{i})\int_{0}^{1}d\mu_{n}\\
 & =\int_{\{\mu\in\mathbb{R}^{n}:\sum_{i=1}^{n-1}\mu_{i}\leq1,\mu_{1\leq i\leq n-1}\in[0,1]\}}\prod_{i=1}^{n-1}\mu_{i}^{\alpha_{i}-1}\left(1-\sum_{i=1}^{n-1}\mu_{i}\right)^{\alpha_{n}}d(\times_{i=1}^{n-1}\mu_{i})\\
 & =\int_{0}^{1}\mu_{1}^{\alpha_{1}-1}\int_{0}^{1-\mu_{i}}\mu_{2}^{\alpha_{2}-1}\cdots\int_{0}^{1-\sum_{i=1}^{n-2}\mu_{i}}\mu_{n-1}^{\alpha_{n-1}-1}\left(1-\sum_{i=1}^{n-1}\mu_{i}\right)^{\alpha_{n}-1}d\mu_{n-1}d\mu_{n-2}\cdots d\mu_{1}\tag{2}\\
 & =\frac{\prod_{i=1}^{n}\Gamma(\alpha_{i})}{\Gamma\left(\sum_{i=1}^{n}\alpha_{i}\right)}
\end{align*}
for any $\{\alpha_{1},\ldots,\alpha_{n}\}$ s.t. $\sum_{i=1}^{n}\alpha_{i}=1.$
Also note that Eq.(2) follows from repeated application of Fubini's
theorem in the following way: 
\begin{align*}
 & \text{Eq.(2)}\\
= & \int_{\{\mu\in\mathbb{R}^{n}:\sum_{i=2}^{n-1}\mu_{i}\leq1-\mu_{1},\mu_{1}\in[0,1],\mu_{2\leq i\leq n-1}\geq0\}}\prod_{i=1}^{n-1}\mu_{i}^{\alpha_{i}-1}\left(1-\sum_{i=1}^{n-1}\mu_{i}\right)^{\alpha_{n}}d(\times_{i=1}^{n-1}\mu_{i})\\
= & \int_{0}^{1}\mu_{1}^{\alpha_{1}-1}\int_{\{(\mu_{2},\dots,\mu_{n})\in\mathbb{R}^{n-1}:\sum_{i=2}^{n-1}\mu_{i}\leq1-\mu_{1},\mu_{2\leq n-1}\geq0\}}\prod_{i=2}^{n-1}\mu_{i}^{\alpha_{i}-1}\left(1-\sum_{i=1}^{n-1}\mu_{i}\right)^{\alpha_{n}}d(\times_{i=1}^{n-1}\mu_{i})\\
= & \int_{0}^{1}\mu_{1}^{\alpha_{1}-1}\int_{\big\{\stackrel{(\mu_{2},\dots,\mu_{n})\in\mathbb{R}^{n-1}:\sum_{i=3}^{n-1}\mu_{i}\leq1-\mu_{1}-\mu_{2}}{\mu_{3\leq i\leq n-1}\geq0,\mu_{2}\in[0,1-\mu_{1}]}\big\}}\prod_{i=2}^{n-1}\mu_{i}^{\alpha_{i}-1}\left(1-\sum_{i=1}^{n-1}\mu_{i}\right)^{\alpha_{n}}d(\times_{i=2}^{n-1}\mu_{i})d\mu_{1}\\
= & \int_{0}^{1}\mu_{1}^{\alpha_{1}-1}\int_{0}^{1-\mu_{1}}\mu_{2}^{\alpha_{2}-1}\int_{\big\{\stackrel{(\mu_{3},\dots,\mu_{n})\in\mathbb{R}^{n-2}:\sum_{i=3}^{n-1}\mu_{i}\leq1-\mu_{1}-\mu_{2},}{\mu_{3\leq i\leq n-1}\geq0}\big\}}\prod_{i=1}^{n-1}\mu_{i}^{\alpha_{i}-1}\left(1-\sum_{i=1}^{n-1}\mu_{i}\right)^{\alpha_{n}}d(\times_{i=3}^{n-1}\mu_{i})d\mu_{2}d\mu_{1}\\
\cdots\\
= & \int_{0}^{1}\mu_{1}^{\alpha_{1}-1}\int_{0}^{1-\mu_{i}}\mu_{2}^{\alpha_{2}-2}\cdots\int_{0}^{1-\sum_{i=1}^{n-2}\mu_{i}}\mu_{n-1}^{\alpha_{n-1}-1}\left(1-\sum_{i=1}^{n-1}\mu_{i}\right)^{\alpha_{n}-1}d\mu_{n-1}d\mu_{n-2}\cdots d\mu_{1}\tag{2}
\end{align*}
We also prove a lemma to facilitate the inductive step. 
\begin{lem}
For any $a\in\mathbb{R}-\{0\}$ and $m,n>0,$ the following integral
identity holds: \label{lem: problem2.9 lem1}
\[
\int_{0}^{1}x^{m-1}(1-x)^{n-1}dx=\frac{1}{a^{m+n-1}}\int_{0}^{a}y^{m-1}(a-y)^{n-1}dy,
\]
and as a result 
\[
\int_{0}^{a}y^{m-1}(a-y)^{n-1}=a^{m+n-1}\frac{\Gamma(m)\Gamma(n)}{\Gamma(m+n)}
\]
\end{lem}
\begin{proof}
By change of variable $x=y/a,$ 
\begin{align*}
\int_{0}^{1}x^{m-1}(1-x)^{n-1}dx & =\frac{1}{a}\int_{0}^{a}\left(\frac{y}{a}\right)^{m-1}\left(1-\frac{y}{a}\right)^{n-1}dy\\
 & =\frac{1}{a^{m+n-1}}\int_{0}^{a}a^{m+n-2}\left(\frac{y}{a}\right)^{m-1}\left(1-\frac{y}{a}\right)^{n-1}dy\\
 & =\frac{1}{a^{m+n-1}}\int_{0}^{a}a^{m-1}\left(\frac{y}{a}\right)^{m-1}a^{n-1}\left(\frac{a-y}{a}\right)^{n-1}dy\\
 & =\frac{1}{a^{m+n-1}}\int_{0}^{a}y^{m-1}(a-y)^{n-1}dy.
\end{align*}
That $\int_{0}^{a}y^{m-1}(a-y)^{n-1}=\frac{1}{a^{m+n-1}}\frac{\Gamma(m)\Gamma(n)}{\Gamma(m+n)}$
then directly follows from Problem 2.5. 
\end{proof}
\noindent Then for $k=n+1,$ again by repeated application of Fubini's
theorem we have 
\begin{align*}
I_{\mu}(n+1) & =\int_{\mathbb{S}_{n+1}}\prod_{i=1}^{n}\mu_{i}^{\alpha_{i}-1}\left(1-\sum_{i=1}^{n}\mu_{i}\right)^{\alpha_{n+1}-1}d\mu\\
 & =\int_{0}^{1}\mu_{1}^{\alpha_{1}-1}\int_{0}^{1-\mu_{1}}\mu_{2}^{\alpha_{2}-1}\cdots\int_{0}^{1-\sum_{i=1}^{n-1}\mu_{i}}\mu_{n}^{\alpha_{n}-1}\left(1-\sum_{i=1}^{n}\mu_{i}\right)^{\alpha_{n+1}-1}d\mu_{n}d\mu_{n-1}\cdots d\mu_{1}.\tag{3}
\end{align*}
Note that by \ref{lem: problem2.9 lem1}
\begin{align*}
 & \int_{0}^{1-\sum_{i=1}^{n-1}\mu_{i}}\mu_{n}^{\alpha_{n}-1}\left(1-\sum_{i=1}^{n}\mu_{i}\right)^{\alpha_{n+1}-1}d\mu_{n}\\
=\ \  & \int_{0}^{1-\sum_{i=1}^{n-1}\mu_{i}}\mu_{n}^{\alpha_{n}-1}\left(1-\sum_{i=1}^{n-1}\mu_{i}-\mu_{n}\right)^{\alpha_{n+1}-1}d\mu_{n}\\
=\ \  & \frac{\Gamma(\alpha_{n})\Gamma(\alpha_{n+1})}{\Gamma(\alpha_{n}+\alpha_{n+1})}\left(1-\sum_{i=1}^{n-1}\mu_{i}\right)^{\alpha_{n}+\alpha_{n+1}-1}.
\end{align*}
Therefore, 
\begin{align*}
\text{Eq.(3)} & =\frac{\Gamma(\alpha_{n})\Gamma(\alpha_{n+1})}{\Gamma(\alpha_{n}+\alpha_{n+1})}\int_{0}^{1}\mu_{1}^{\alpha_{1}-1}\cdots\int_{0}^{1-\sum_{i=1}^{n-2}\mu_{i}}\mu_{n-1}^{\alpha_{n-1}-1}\left(1-\sum_{i=1}^{n-1}\mu_{i}\right)^{\alpha_{n}+\alpha_{n+1}-1}d\mu_{n-1}\cdots d\mu_{1}\\
 & =\frac{\Gamma(\alpha_{n})\Gamma(\alpha_{n+1})}{\Gamma(\alpha_{n}+\alpha_{n+1})}\frac{\Gamma(\alpha_{1})\Gamma(\alpha_{2})\cdots\Gamma(\alpha_{n}+\alpha_{n+1})}{\Gamma(\alpha_{1}+\cdots+\alpha_{n+1})}\tag{by inductive hypothesis}\\
 & =\frac{\prod_{i=1}^{n}\Gamma(\alpha_{i})}{\Gamma\left(\sum_{i=1}^{n}\alpha_{i}\right)}
\end{align*}
as desired.\\

\begin{cBoxA}[Problem 2.10 - Dirichlet distribution's expectation, variance and
covariance]{}
 Using the property $\Gamma(x+1)=x\Gamma(x)$ of the gamma function,
derive the following results for the mean, and covariance of the Dirichlet
distribution given by (2.38) 
\begin{align*}
 & \E[\mu_{j}]=\frac{\alpha_{j}}{\alpha_{0}},\\
 & \mathrm{Var}[\mu_{j}]=\frac{\alpha_{j}(\alpha_{0}-\alpha_{j})}{\alpha_{0}^{2}(\alpha_{0}+1)},\\
 & \mathrm{Cov}[\mu_{j}\mu_{l}]=-\frac{\alpha_{j}\alpha_{l}}{\alpha_{0}^{2}(\alpha_{0}+1)},\ \ j\neq l,
\end{align*}
where $\alpha_{0}$ is defined by $(2.39)$
\end{cBoxA}

In the discussion below, we let $\mu$ be a $n$-dimensional random
vector s.t $\mu\sim\mathrm{Dir}(\alpha)$ and $\mathbb{S}_{n}$ denote
the standard simplex in $\mathbb{R}^{n}.$ 
\begin{enumerate}[leftmargin={*}]
\item  To find the expectation, note that 
\begin{align*}
\E[\mu_{j}] & =\int_{\mathbb{S}_{n}}\frac{\Gamma(\alpha_{1})\cdots\Gamma(\alpha_{n})}{\Gamma(\alpha_{1}+\cdots+\alpha_{n})}\mu_{j}\prod_{i=1}^{n-1}\mu_{i}^{\alpha_{i}-1}\left(1-\sum_{i=1}^{n-1}\mu_{i}\right)^{\alpha_{n}-1}d\mu\\
 & =\begin{cases}
\int_{\mathbb{S}_{n}}\frac{\Gamma(\alpha_{1})\cdots\Gamma(\alpha_{n})}{\Gamma(\alpha_{1}+\cdots+\alpha_{n})}\prod_{i=1}^{n-1}\mu_{i}^{\alpha_{i}-1}\left(1-\sum_{i=1}^{n-1}\mu_{i}\right)^{\alpha_{n}+1-1}d\mu & \text{if }j=n\\
\int_{\mathbb{S}_{n}}\frac{\Gamma(\alpha_{1})\cdots\Gamma(\alpha_{n})}{\Gamma(\alpha_{1}+\cdots+\alpha_{n})}\prod_{i\geq1,i\neq j}^{n-1}\mu_{j}^{\alpha_{i}-1}\mu_{j}^{\alpha_{j}+1-1}\left(1-\sum_{i=1}^{n-1}\mu_{i}\right)^{\alpha_{n}-1}d\mu & \text{if }j\in\{1,...,n-1\}
\end{cases}\\
 & =\frac{\Gamma(\alpha_{1}+\cdots+\alpha_{n})}{\Gamma(\alpha_{1})\cdots\Gamma(\alpha_{n})}\frac{\prod_{i\geq1,i\neq j}^{n-1}\Gamma(\alpha_{i})\Gamma(\alpha_{j}+1)}{\Gamma\left((\sum_{i=1}^{n}\alpha_{i})+1\right)}\\
 & =\frac{\alpha_{j}}{\sum_{i=1}^{n}\alpha_{i}}.
\end{align*}
\item To find the variance, note that using the same argument as in part(1),
\begin{align*}
\E[\mu_{j}^{2}] & =\begin{cases}
\int_{\mathbb{S}_{n}}\frac{\Gamma(\alpha_{1})\cdots\Gamma(\alpha_{n})}{\Gamma(\alpha_{1}+\cdots+\alpha_{n})}\prod_{i=1}^{n-1}\mu_{i}^{\alpha_{i}-1}\left(1-\sum_{i=1}^{n-1}\mu_{i}\right)^{\alpha_{n}+2-1}d\mu & \text{if }j=n\\
\int_{\mathbb{S}_{n}}\frac{\Gamma(\alpha_{1})\cdots\Gamma(\alpha_{n})}{\Gamma(\alpha_{1}+\cdots+\alpha_{n})}\prod_{i\geq1,i\neq j}^{n-1}\mu_{j}^{\alpha_{i}-1}\mu_{j}^{\alpha_{j}+2-1}\left(1-\sum_{i=1}^{n-1}\mu_{i}\right)^{\alpha_{n}-1}d\mu & \text{if }j\in\{1,...,n-1\}
\end{cases}\\
 & =\frac{\Gamma(\alpha_{1}+\cdots+\alpha_{n})}{\Gamma(\alpha_{1})\cdots\Gamma(\alpha_{n})}\frac{\prod_{i\geq1,i\neq j}^{n-1}\Gamma(\alpha_{i})\Gamma(\alpha_{j}+2)}{\Gamma\left((\sum_{i=1}^{n}\alpha_{i})+2\right)}\\
 & =\frac{\alpha_{j}(\alpha_{j}+1)}{(\sum_{i=1}^{n}\alpha_{i}+1)(\sum_{i=1}^{n}\alpha_{i})}.
\end{align*}
Hence, 
\begin{align*}
\text{Var}[\mu_{j}] & =\E[\mu_{j}^{2}]-(\E[\mu_{j}])^{2}\\
 & =\frac{\alpha_{j}(\alpha_{j}+1)}{(\sum_{i=1}^{n}\alpha_{i}+1)(\sum_{i=1}^{n}\alpha_{i})}-\frac{\alpha_{j}^{2}}{(\sum_{i=1}^{n}\alpha_{i})^{2}}\\
 & =\frac{(\sum_{i=1}^{n}\alpha_{i})\alpha_{j}(\alpha_{j}+1)-(\sum_{i=1}^{n}\alpha_{i}+1)\alpha_{j}^{2}}{(\sum_{i=1}^{n}\alpha_{i}+1)(\sum_{i=1}^{n}\alpha_{i})^{2}}\\
 & =\frac{\alpha_{j}(\sum_{i=1}^{n}\alpha_{i}-\alpha_{j})}{(\sum_{i=1}^{n}\alpha_{i}+1)(\sum_{i=1}^{n}\alpha_{i})^{2}}.
\end{align*}
\item To find the covariance, note that 
\begin{align*}
\E[\mu_{i}\mu_{j}] & =\begin{cases}
\int_{\mathbb{S}_{n}}\frac{\Gamma(\alpha_{1})\cdots\Gamma(\alpha_{n})}{\Gamma(\alpha_{1}+\cdots+\alpha_{n})}\prod_{k\geq1,k\notin\{i,j\}}^{n-1}\mu_{k}^{\alpha_{i}-1}\mu_{i\in\{i,j\}}^{\alpha_{i\in\{i,j\}}-1}\left(1-\sum_{i=1}^{n-1}\mu_{i}\right)^{\alpha_{n}+1-1}d\mu & \text{if }i=n\text{ or }j=n\\
\int_{\mathbb{S}_{n}}\frac{\Gamma(\alpha_{1})\cdots\Gamma(\alpha_{n})}{\Gamma(\alpha_{1}+\cdots+\alpha_{n})}\prod_{k\geq1,k\neq i,j}^{n-1}\mu_{k}^{\alpha_{i}-1}\mu_{i}^{\alpha_{i}+1-1}\mu_{j}^{\alpha_{j}+1-1}\left(1-\sum_{i=1}^{n-1}\mu_{i}\right)^{\alpha_{n}-1}d\mu & \text{if }i\neq n\text{ and }j\neq n
\end{cases}\\
 & =\frac{\Gamma(\alpha_{1})\cdots\Gamma(\alpha_{n})}{\Gamma(\alpha_{1}+\cdots+\alpha_{n})}\frac{\prod_{k\geq1,k\neq i,j}^{n}\Gamma(\alpha_{k})\Gamma(\alpha_{i}+1)\Gamma(\alpha_{j}+1)}{\Gamma(\sum_{i=k}^{n}\alpha_{k}+2)}\\
 & =\frac{\alpha_{i}\alpha_{j}}{(\sum_{i=1}^{n}\alpha_{i}+1)(\sum_{i=1}^{n}\alpha_{i})}.
\end{align*}
Therefore, it follows that 
\begin{align*}
\text{Cov}[\mu_{i},\mu_{j}] & =\E[\mu_{i}\mu_{j}]-\E[\mu_{i}]\E[\mu_{j}]\\
 & =\frac{\alpha_{i}\alpha_{j}}{(\sum_{i=1}^{n}\alpha_{i}+1)(\sum_{i=1}^{n}\alpha_{i})}-\frac{\alpha_{i}\alpha_{j}}{(\sum_{i=1}^{n}\alpha_{i})^{2}}\\
 & =\frac{\alpha_{i}\alpha_{j}(\sum_{i=1}^{n}\alpha_{i})-(\sum_{i=1}^{n}\alpha_{i}+1)\alpha_{i}\alpha_{j}}{(\sum_{i=1}^{n}\alpha_{i}+1)(\sum_{i=1}^{n}\alpha_{i})^{2}}\\
 & =-\frac{\alpha_{i}\alpha_{j}}{(\sum_{i=1}^{n}\alpha_{i}+1)(\sum_{i=1}^{n}\alpha_{i})^{2}}.
\end{align*}
\\
\end{enumerate}
\begin{cBoxA}[{Problem 2.11 - Expression for $\mathbb{E}[\log\mathrm{Dir}(\alpha)]$}]{}
 By expressing the expectation of $\ln\mu_{j}$ under the Dirichlet
distribution (2.38) as a derivative with respect to $\alpha_{j}$,
show that 
\[
\E[\ln\mu_{j}]=\psi(\alpha_{j})-\psi(\alpha_{0})
\]
where $\alpha_{0}$ is given by (2.39) and 
\[
\psi(a)\equiv\frac{d}{da}\ln\Gamma(a)
\]
is the digamma function. 
\end{cBoxA}

\textit{In the discussion below, let $X$ be a $n$-dimensional random
vector such that $X\sim\mathrm{Dir}(\alpha).$\medskip}\\
Note that for $(\mu_{1},\ldots,\mu_{n})$ in the $n$-dimensional
standard simplex, we have 
\begin{align*}
\frac{\partial}{\partial\alpha_{j}}\left[\prod_{i=1}^{n}\mu_{i}^{\alpha_{i}-1}\right] & =\ln\mu_{j}\prod_{i=1}^{n}\mu_{i}^{\alpha_{i}-1}.
\end{align*}
Then it follows that 
\begin{align*}
\E[\ln\mu_{j}] & =\int_{\mathbb{S}_{n}}\frac{\Gamma(\sum_{i=1}^{n}\alpha_{i})}{\Pi_{i=1}^{n}\Gamma(\alpha_{i})}\ln\mu_{j}\prod_{i=1}^{n}\mu_{i}^{\alpha_{i}-1}d\mu\\
 & =\int_{\mathbb{S}_{n}}\frac{\Gamma(\sum_{i=1}^{n}\alpha_{i})}{\Pi_{i=1}^{n}\Gamma(\alpha_{i})}\frac{\partial}{\partial\alpha_{j}}\left[\prod_{i=1}^{n}\mu_{i}^{\alpha_{i}-1}\right]d\mu\\
 & =\frac{\Gamma(\sum_{i=1}^{n}\alpha_{i})}{\Pi_{i=1}^{n}\Gamma(\alpha_{i})}\frac{\partial}{\partial\alpha_{j}}\int\prod_{i=1}^{n}\mu_{i}^{\alpha_{i}-1}d\mu\tag{by Leibniz rule}\\
 & =\frac{\Gamma(\sum_{i=1}^{n}\alpha_{i})}{\Pi_{i=1}^{n}\Gamma(\alpha_{i})}\frac{\partial}{\partial\alpha_{j}}\left[\frac{\Pi_{i=1}^{n}\Gamma(\alpha_{i})}{\Gamma(\sum_{i=1}^{n}\alpha_{i})}\right]\tag{(1)}.
\end{align*}
Now we simplify Eq.(1), 
\begin{align*}
\text{Eq.(1)} & =\frac{\Gamma(\sum_{i=1}^{n}\alpha_{i})}{\prod_{i=1}^{n}\Gamma(\alpha_{i})}\left[\frac{\prod_{i\geq1,i\neq j}^{n}\Gamma(\alpha_{i})}{\Gamma(\sum_{i=1}^{n}\alpha_{i})}\frac{\partial\Gamma(\alpha_{j})}{\partial\alpha_{j}}-\frac{1}{\Gamma(\sum_{i=1}^{n}\alpha_{i})^{2}}\frac{\partial\Gamma(\sum_{i=1}^{n}\alpha_{i})}{\partial\alpha_{j}}\right]\\
 & =\frac{\Gamma(\sum_{i=1}^{n}\alpha_{i})}{\prod_{i=1}^{n}\Gamma(\alpha_{i})}\Biggr[\frac{\prod_{i\geq1,i\neq j}^{n}\Gamma(\alpha_{i})}{\Gamma(\sum_{i=1}^{n}\alpha_{i})}\frac{\partial\Gamma(\alpha_{j})}{\partial\alpha_{j}}-\frac{1}{\Gamma(\sum_{i=1}^{n}\alpha_{i})^{2}}\frac{\partial\Gamma(\sum_{i=1}^{n}\alpha_{i})}{\partial(\sum_{i=1}^{n}\alpha_{i})}\underbrace{\frac{\partial(\sum_{i=1}^{n}\alpha_{i})}{\partial\alpha_{j}}}_{=1}\Biggr]\\
 & =\frac{1}{\Gamma(\alpha_{j})}\frac{\partial\Gamma(\alpha_{j})}{\partial\alpha_{j}}-\frac{1}{\Gamma(\sum_{i=1}^{n}\alpha_{i})}\frac{\partial\Gamma(\sum_{i=1}^{n}\alpha_{i})}{\partial(\sum_{i=1}^{n}\alpha_{i})}\\
 & =\frac{\partial}{\partial\alpha_{j}}\ln\Gamma(\alpha_{j})-\frac{\partial}{\partial(\sum_{i=1}^{n}\alpha_{i})}\ln\Gamma\bigg(\sum_{i=1}^{n}\alpha_{i}\bigg).
\end{align*}
Hence, $\E[\ln\mu_{j}]=\psi(\alpha_{j})-\psi(\sum_{i=1}^{n}\alpha_{i})$
as desired. \\

\begin{cBoxA}[Problem 2.12 - Uniform distribution's normalization, expectation,
variance ]{}
 The uniform distribution for a continuous variable $X$ is defined
by 
\[
\mathrm{U}(x\vert a,b)=\frac{1}{b-a},\ a\leq x\leq b.
\]
Verify that this distribution is normalized, and find expressions
for its mean and variance.
\end{cBoxA}

\textit{In the discussion below, the $X$ be a random variable such
that $X\sim\text{Uniform}(a,b)$ with $f_{X}(x)=\frac{1}{b-a}\mathbbm{1}_{[a,b]}.$}
\begin{enumerate}[leftmargin={*}]
\item  To see the normalization, note that 
\[
\int\frac{1}{b-a}\mathbbm{1}_{[a,b]}dx=\frac{1}{b-a}(b-a)=1.
\]
\item To find the expectation, note 
\[
\E[X]=\int_{a}^{b}x\frac{1}{b-a}dx=\left[\frac{x^{2}}{2}\right]_{a}^{b}(b-a)=\frac{b^{2}-a^{2}}{2}\frac{1}{(b-a)}=\frac{a+b}{2}.
\]
\item To find the variance, first we note that 
\begin{align*}
\E[X^{2}] & =\int_{a}^{b}x^{2}\frac{1}{b-a}dx=\left[\frac{x^{3}}{3}\right]_{a}^{b}(b-a)=\frac{b^{3}-a^{3}}{3}\frac{1}{(b-a)}\\
 & =\frac{(b-a)(a^{2}+b^{2}+ab)}{3(b-a)}=\frac{a^{2}+b^{2}+ab}{3}.
\end{align*}
 And thus 
\begin{align*}
\text{Var}[X] & =\E[X^{2}]-(\E[X])^{2}=\frac{a^{2}+b^{2}+ab}{3}-\left(\frac{a+b}{2}\right)^{2}\\
 & =\frac{4a^{2}+4b^{2}+4ab-3a^{2}-3b^{2}-6ab}{12}\\
 & =\frac{a^{2}+b^{2}-2ab}{12}\\
 & =\frac{(a-b)^{2}}{12}
\end{align*}
as desired. \\
\end{enumerate}
\begin{cBoxA}[Problem 2.14 - Multidimensional gaussian maximizes entropy ]{}
 This exercise demonstrates that the multivariate distribution with
maximum entropy, for a given covariance, is a Gaussian. The entropy
of a distribution $p(x)$ is given by 
\[
H(X)=-\int p(x)\ln p(x)dx.
\]
We with to maximize $H(X)$ over all distribution $p(x)$ subject
to the constraints that $p(x)$ be normalized and that it have specific
mean and covariance, so that 
\begin{align*}
 & \int p(x)dx=1,\\
 & \int p(x)xdx=\mu,\\
 & \int p(x)(x-\mu)(x-\mu)^{T}dx=\Sigma.
\end{align*}
By performing a variational maximization of (2.279) and using Lagrange
multipliers to enforce the constraints (2.280), (2.281), and (2.282),
show that the maximum likelihood distribution is given by the Gaussian
(2.43).
\end{cBoxA}

First, we write out the Lagrangian 
\begin{align*}
\mathcal{L}(p(x))= & -\int p(x)\ln p(x)dx+\left\langle \begin{bmatrix}\lambda_{1}\\
\lambda_{2}\\
\lambda_{3}
\end{bmatrix},\begin{bmatrix}\int p(x)dx-1\\
\int p(x)xdx-\mu\\
\int p(x)(x-\mu)(x-\mu)^{T}dx-\Sigma
\end{bmatrix}\right\rangle _{\text{prod}}\\
= & \int-p(x)\ln p(x)dx\\
 & +\left\langle \lambda_{1},\int p(x)dx-1\right\rangle +\left\langle \lambda_{2},\int p(x)xdx-\mu\right\rangle +\left\langle \lambda_{3},\int p(x)(x-\mu)(x-\mu)^{T}dx-\Sigma\right\rangle \\
= & \int-p(x)\ln p(x)dx\\
 & +\lambda_{1}\left(\int p(x)dx-1\right)+\lambda_{2}^{T}\left(\int p(x)xdx-\mu\right)+\text{tr}\left(\lambda_{3}^{T}\left(\int p(x)(x-\mu)(x-\mu)^{T}dx-\Sigma\right)\right)\\
= & \int-p(x)\ln p(x)dx\\
 & +\lambda_{1}\left(\int p(x)dx-1\right)+\lambda_{2}^{T}\left(\int p(x)xdx-\mu\right)+\text{tr}\left(\lambda_{3}\left(\int p(x)(x-\mu)(x-\mu)^{T}dx-\Sigma\right)\right)\tag{1}\\
= & \int-p(x)\ln p(x)+\lambda_{1}p(x)+\lambda_{2}^{T}p(x)x+\text{tr}((x-\mu)^{T}\lambda_{3}p(x)(x-\mu))dx-(\lambda_{1}+\lambda_{2}^{T}\mu+\lambda_{3}\Sigma)\\
:= & \int F(p(x)dx+C,\tag{by relabeling}
\end{align*}
where Eq.(1) follow can be justified as follows. Note that 
\begin{align*}
 & \int p(x)(x-\mu)(x-\mu)^{T}dx\\
=\ \  & \int\begin{bmatrix}p(x)(x_{1}-\mu)(x_{1}-\mu) & p(x)(x_{1}-\mu)(x_{2}-\mu) & \cdots & p(x)(x_{1}-\mu)(x_{n}-\mu)\\
p(x)(x_{2}-\mu)(x_{1}-\mu) & \ddots & \cdots & p(x)(x_{2}-\mu)(x_{n}-\mu)\\
\vdots & \vdots & \cdots & \vdots\\
p(x)(x_{n}-\mu)(x_{1}-\mu) & p(x)(x_{n}-\mu)(x_{2}-\mu) & \cdots & p(x)(x_{n}-\mu)(x_{n}-\mu)
\end{bmatrix}dx\\
=\ \  & \begin{bmatrix}\int p(x)(x_{1}-\mu)(x_{1}-\mu)dx & \int p(x)(x_{1}-\mu)(x_{2}-\mu)dx & \cdots & \int p(x)(x_{1}-\mu)(x_{n}-\mu)dx\\
\int p(x)(x_{2}-\mu)(x_{1}-\mu)dx & \ddots & \cdots & \int p(x)(x_{2}-\mu)(x_{n}-\mu)dx\\
\vdots & \vdots & \cdots & \vdots\\
\int p(x)(x_{n}-\mu)(x_{1}-\mu)dx & \int p(x)(x_{n}-\mu)(x_{2}-\mu)dx & \cdots & \int p(x)(x_{n}-\mu)(x_{n}-\mu)dx
\end{bmatrix},
\end{align*}
which is symmetric, whence by cyclic property of trace it follows
that 
\begin{align*}
\text{tr}\left(\lambda_{3}^{T}\left(\int p(x)(x-\mu)(x-\mu)^{T}dx-\Sigma\right)\right) & =\text{tr}\left(\lambda_{3}\left(\int p(x)(x-\mu)(x-\mu)^{T}dx-\Sigma\right)^{T}\right)\\
 & =\text{tr}\left(\lambda_{3}\left(\int p(x)(x-\mu)(x-\mu)^{T}dx-\Sigma\right)\right)
\end{align*}
To maximize, we take the functional derivative and set it to zero:
\begin{align*}
 & \frac{\delta L(p(x)}{\delta p(x)}=\frac{\partial F(p(x))}{\partial p(x)}=-\ln p(x)-1+\lambda_{1}+\lambda_{2}^{T}x+(x-\mu)^{T}\lambda_{3}(x-\mu)=0\\
\implies & p(x)=\exp\{\lambda_{1}-1+\lambda_{2}^{T}x+(x-\mu)^{T}\lambda_{3}(x-\mu)\}.
\end{align*}
Now we substitute $p(x)$ into the constraints:
\begin{align*}
\int p(x)xdx= & \int\exp\{\lambda_{1}-1+\lambda_{2}^{T}x+(x-\mu)^{T}\lambda_{3}(x-\mu)\}xdx\\
= & \int\exp\left\{ \left(x-\mu+\frac{1}{2}\lambda_{3}^{-1}\lambda_{2}\right)^{T}\lambda_{3}\left(x-\mu+\frac{1}{2}\lambda_{3}^{-1}\lambda_{2}\right)-\frac{1}{4}\lambda_{2}\lambda_{3}^{-1}\lambda_{2}+\lambda_{2}^{T}\mu+\lambda_{1}-1\right\} dx\\
= & \int\exp\left\{ y^{T}\lambda_{3}y-\frac{1}{4}\lambda_{2}\lambda_{3}^{-1}\lambda_{2}+\lambda_{2}^{T}\mu+\lambda_{1}-1\right\} \left(y+\mu-\frac{1}{2}\lambda_{3}^{-1}\lambda_{2}\right)dy\tag{2}\\
= & \int\exp\left\{ y^{T}\lambda_{3}y-\frac{1}{4}\lambda_{2}\lambda_{3}^{-1}\lambda_{2}+\lambda_{2}^{T}\mu+\lambda_{1}-1\right\} ydy\tag{3}\\
 & +\int\exp\left\{ y^{T}\lambda_{3}y-\frac{1}{4}\lambda_{2}\lambda_{3}^{-1}\lambda_{2}+\lambda_{2}^{T}\mu+\lambda_{1}-1\right\} \left(\mu-\frac{1}{2}\lambda_{3}^{-1}\lambda_{2}\right)dy\tag{4}\\
= & \mu,
\end{align*}
where Eq.(2) follows from change of variable $y=x-\mu+\frac{1}{2}\lambda_{3}^{-1}\lambda_{2}.$
We take a closer look at Eq.(2). A couple of claims are in order. 
\begin{lem}
The following identity holds:
\[
\int_{\mathbb{R}^{n}}\exp\left\{ y^{T}\lambda_{3}y-\frac{1}{4}\lambda_{2}\lambda_{3}^{-1}\lambda_{2}+\lambda_{2}^{T}\mu+\lambda_{1}-1\right\} ydy=0,
\]
where $y\in\mathbb{R}^{n}.$ 
\end{lem}
\begin{proof}
The key to proving it is that the integrand, denoted as $\varphi(y),$
is a \textquotedbl odd\textquotedbl{} function in multidimensional
space: 
\begin{align*}
\varphi(-y) & =\exp\left\{ (-y^{T})\lambda_{3}(-y)-\frac{1}{4}\lambda_{2}\lambda_{3}^{-1}\lambda_{2}+\lambda_{2}^{T}\mu+\lambda_{1}-1\right\} (-y)\\
 & =-\exp\left\{ y^{T}\lambda_{3}y-\frac{1}{4}\lambda_{2}\lambda_{3}^{-1}\lambda_{2}+\lambda_{2}^{T}\mu+\lambda_{1}-1\right\} y\\
 & =-\varphi(y).
\end{align*}
Then note that 
\[
\mathbb{R}^{n}=\underbrace{\left(\bigcup_{(\#_{1},\#_{2},\cdots,\#_{n})\in\prod_{i=1}^{n}\{+,-\}}\prod_{i=1}^{n}\mathbb{R}^{\#_{i}}\right)}_{:=P}\bigcup\underbrace{\left(\bigcup_{(\#_{1},\#_{2},\cdots,\#_{n})\in\prod_{i=1}^{n}\{0,1\},\exists\#_{i}=0\ \text{for some}\ i}\prod_{i=1}^{n}\mathbb{R}^{\#_{i}}\right)}_{:=N},
\]
where $\mathbb{R}^{+}:=\{x\in\mathbb{R}|x>0\}$, $\mathbb{R}^{-}:=\{x\in\mathbb{R}|x<0\}$,
$\mathbb{R}^{0}:=\{0\},$and $\mathbb{R}^{1}:=\mathbb{R}$. Now we
can rewrite the integral of interest as 
\[
\int_{P\cup N}\exp\left\{ y^{T}\lambda_{3}y-\frac{1}{4}\lambda_{2}\lambda_{3}^{-1}\lambda_{2}+\lambda_{2}^{T}\mu+\lambda_{1}-1\right\} ydy=\int_{N}\varphi(y)dy+\int_{P}\varphi(y)dy.
\]
Since $m(N)=0,$ (c.f. \cite[ Lemma 3.5]{stein2005real}), $\int_{N}\varphi(y)dy=0.$
On the other hand, since $P$ can be written as $2^{n}$ disjoint
unions by definition (note that for $(\#_{1},\ldots,\#_{n})\neq(\tilde{\#}_{1},\ldots,\tilde{\#}_{2}),$
we have ($\prod_{i=1}^{n}\mathbb{R}^{\#_{i}})\cap(\prod_{i=1}^{n}\mathbb{R}^{\tilde{\#}_{i}})=\emptyset$),
we have that 
\[
\int_{P}\varphi(y)dy=\int_{\sqcup_{i=1}^{2^{n}}P_{i}}\varphi(y)dy=\sum_{i=1}^{2^{n}}\int_{P_{i}}\varphi(y)dy.
\]
Since for any $i\in\{1,...,2^{n}\},$ there exists some $j\neq i\in\{1,...,2^{n}\}$
such that $P_{i}=(-1)\cdot P_{j}$, we can rewrite the last term in
the previous term as the sum over pairs
\begin{align*}
\sum_{i=1}^{2^{n}}\int_{P_{i}}\varphi(y)dy & =\sum_{(i,j)}\left(\int_{P_{i}}\varphi(y)dy+\int_{P_{j}}\varphi(y)dy\right)=\sum_{(i,j)}\left(\int_{P_{i}}-\varphi(-y)dy+\int_{P_{j}}\varphi(y)dy\right)\\
 & =\sum_{(i,j)}\left(\int_{-P_{i}}\varphi(-(-x))dx+\int_{P_{j}}\varphi(y)dy\right)\tag{by \ref{thm: transformation thm}}\\
 & =\sum_{(i,j)}\left(-\int_{P_{j}}\varphi(x)dx+\int_{P_{j}}\varphi(y)dy\right)\\
 & =0.
\end{align*}
Therefore, it follows that the desired integral is zero. \label{lem: problem-2.14-lem1}
\end{proof}
\noindent Therefore, we have 
\begin{align*}
\text{Eq.(2)} & =\int\exp\left\{ y^{T}\lambda_{3}y-\frac{1}{4}\lambda_{2}\lambda_{3}^{-1}\lambda_{2}+\lambda_{2}^{T}\mu+\lambda_{2}-1\right\} \left(\mu-\frac{1}{2}\lambda_{3}^{-1}\lambda_{2}\right)dy.\tag{5}
\end{align*}
Now we break it down and evaluate Eq.(3) term by term, note that 
\[
1=\int\exp\left\{ \lambda_{1}-1+\lambda_{2}^{T}x+(x-\mu)^{T}\lambda_{3}(x-\mu)\right\} dx=\int\exp\left\{ \lambda_{1}-1+\lambda_{2}^{T}\mu+y^{T}\lambda_{3}y-\frac{1}{4}\lambda_{2}\lambda_{3}^{-1}\lambda_{2}\right\} dy,
\]
by change of variable $y=x-\mu+\frac{1}{2}\lambda_{3}^{-1}\lambda_{2}.$
Hence, substituting these results back, we get 
\[
\text{Eq.(2)}=\mu-\frac{1}{2}\lambda_{3}^{-1}\lambda_{2}=\mu\iff\lambda_{3}^{-1}\lambda_{2}=0\implies\lambda_{2}=0,
\]
where the last implication can be justified as follows: suppose $\lambda_{3}=0,$
then $p(x)=\exp\{\lambda_{1}-1+\lambda_{2}^{T}x\}$ is a constant.
And thus $\int_{\mathbb{R}^{n}}p(x)dx=\infty$ unless $p(x)=0,$ in
which case the integral evaluates to 0, which does not satisfy Eq.(2.280).
Hence, it follows that 
\[
p(x)=\exp\left\{ \lambda_{1}-1+(x-\mu)^{T}\lambda_{3}(x-\mu)\right\} .
\]
Now, we substitute back into the last constraint: 
\[
\int\exp\left\{ \lambda_{1}-1+(x-\mu)^{T}\lambda_{3}(x-\mu)\right\} (x-\mu)(x-\mu)^{T}dx=\Sigma.
\]
In order to find a solution, we recall that 
\[
\int\frac{1}{(2\pi)^{D/2}\left|\Sigma\right|^{1/2}}\exp\left\{ -\frac{1}{2}(x-\mu)^{T}\Sigma^{-1}(x-\mu)\right\} (x-\mu)(x-\mu)^{T}dx=\Sigma.
\]
Hence, by comparison of the coefficients, we see that $\lambda_{3}=-\frac{1}{2}\Sigma^{-1}$
and 
\[
\exp\left\{ \lambda_{1}-1\right\} =\frac{1}{(2\pi)^{D/2}\left|\Sigma\right|^{1/2}}\implies\lambda_{1}=\log\left(\frac{1}{(2\pi)^{D/2}\left|\Sigma\right|^{1/2}}\right)+1
\]
forms a set of admissible solution. With this set of $\lambda$'s,
we see that the $p(x)$ is the Gaussian density and thus it follows
that multivariate Gaussian distribution is a minimizer of the calculus
of variation program proposed in this problem. \\

\begin{cBoxA}[Problem 2.15 - Entropy of multivariate gaussian]{}
 Show that the entropy of the multivariate Gaussian $\mathrm{N}(x\vert\mu,\Sigma)$
is given by 
\[
H(X)=\frac{1}{2}\ln\left|\Sigma\right|+\frac{D}{2}(1+\ln(2\pi))
\]
where $D$ is the dimensionality of $X$
\end{cBoxA}

In the discussion below, let $X$ be a random vector such that $X\sim\mathrm{MVN}(\mu,\Sigma)$
with density $\varphi(x\vert\mu,\Sigma)$. First, we give a lemma
to be used later. 
\begin{lem}
Let $A\in\mathrm{Mat}_{\mathbb{R}}(n,m)$ and $B(x)\in\mathrm{Mat}_{\mathbb{R}}(m,n)$.
Then the following identity holds:\label{lem: problem 2.15 lem1}
\[
\mathrm{tr}\left(\int AB(x)dx\right)=\int\mathrm{tr}(AB(x))dx.
\]
\end{lem}
\begin{proof}
Just write out the equation and follow definitions: 
\begin{align*}
\text{tr}\left(\int AB(x)dx\right) & =\sum_{i=1}^{n}\left(\int AB(x)dx\right)_{ii}=\sum_{i=1}^{n}\left(\int\sum_{j=1}^{n}A_{ij}B_{ij}(x)dx\right)\\
 & =\int\sum_{i=1}^{n}\sum_{j=1}^{n}A_{ij}B_{ij}(x)dx=\int\text{tr}(AB(x))dx
\end{align*}
as desired.
\end{proof}
\noindent Now, we go back to the proof. Note that 
\begin{align*}
H(X) & =-\int\varphi(x|\mu,\Sigma)\ln\varphi(x|\mu,\Sigma)\\
 & =\int\varphi(x|\mu,\Sigma)\left(\log\frac{1}{(2\pi)^{D/2}\left|\Sigma\right|^{1/2}}+\frac{1}{2}(x-\mu)^{T}\Sigma^{-1}(x-\mu)\right)d\mu\\
 & =\log(2\pi)^{D/2}+\log\left|\Sigma\right|^{1/2}+\frac{1}{2}\int\varphi(x|\mu,\Sigma)(x-\mu)^{T}\Sigma^{-1}(x-\mu)d\mu\\
 & =\log(2\pi)^{D/2}+\log\left|\Sigma\right|^{1/2}+\frac{1}{2}\int\varphi(x|\mu,\Sigma)\text{tr}((x-\mu)^{T}\Sigma^{-1}(x-\mu))d\mu\\
 & =\log(2\pi)^{D/2}+\log\left|\Sigma\right|^{1/2}+\frac{1}{2}\int\varphi(x|\mu,\Sigma)\text{tr}(\Sigma^{-1}(x-\mu)(x-\mu)^{T})d\mu\\
 & =\log(2\pi)^{D/2}+\log\left|\Sigma\right|^{1/2}+\frac{1}{2}\text{tr}\left(\Sigma^{-1}\int\varphi(x|\mu,\Sigma)(x-\mu)(x-\mu)^{T}d\mu\right)\tag{by \ref{lem: problem 2.15 lem1}}\\
 & =\log(2\pi)^{D/2}+\log\left|\Sigma\right|^{1/2}+\frac{1}{2}\text{tr}(\Sigma^{-1}\Sigma)\\
 & =\log(2\pi)^{D/2}+\log\left|\Sigma\right|^{1/2}+\frac{1}{2}\text{tr}(I_{D})\\
 & =\frac{D}{2}(\log2\pi+1)+\frac{1}{2}\log\left|\Sigma\right|.
\end{align*}
as desired.\\

\begin{cBoxA}[Problem 2.16 - Entropy of sum of two gaussians]{}
 Consider two random variables $X_{1}$ and $X_{2}$ having Gaussian
distribution with means $\mu_{1},\mu_{2}$ and precisions $\tau_{1},\tau_{2}$
respectively. Derive an expression for the differential entropy of
the variable $X=X_{1}+X_{2}.$ To do this, first find the distribution
of $X$ by using the relation 
\[
f(x)=\int_{-\infty}^{\infty}f(x\vert x_{2})f(x_{2})dx_{2}
\]
and completing the square in the exponent. Then observe that this
represents the convolution of two Gaussian distributions, which itself
will be Gaussian, and finally make use of the result (1.110) for the
entropy of the univariate Gaussian. 
\end{cBoxA}

\textit{This problem can be solved in various ways. The method proposed
by hint given in the problem is limited in the sense that it is hard
to generalized to arbitrary transformations and requires a lot of
computation, which is error prone. We shall take a different approach
here. }

\noindent  First, we give a lemma. 
\begin{lem}
Let $X$ be a $\mathbb{R}^{n}$-valued random vector such that $X\sim\mathrm{MVN}(\mu,\Sigma).$
Then $Y=AX+b\sim\mathrm{MVN}(A\mu+b,A\Sigma A^{*})$ for $A\in\mathrm{Mat}_{\mathbb{R}}(m,n)$
and $b\in\mathbb{R}^{m}.$ \label{lem: distribution of linear transformation of gaussian}
\end{lem}
To prove this lemma, we need to develop more theory and give more
background knowledge about multivariate Gaussian distribution, which
we present below.

\subsubsection*{Supplement knowledge }

In the book, the notion of a Gaussian distribution was mainly introduced
as a maximizer of a calculus of variation problem under some constraint.
This is completely valid and useful. But the addition of some more
auxiliary definitions and results will help us gain a more through
understanding of this distribution. \medskip\\
In the discussion below, assume we have derived the univariate normal
distribution using the book's view point. But now we use another route
to push the result to the general setting. First, a few lemmas. 
\begin{lem}
The characteristic function for the $\mathrm{N}(\mu,\sigma^{2})$
is given by 
\[
\varphi(t)=e^{it\mu}e^{-\frac{1}{2}\sigma^{2}t^{2}}.
\]
\end{lem}
\begin{proof}
Following the definition, we have 
\begin{align*}
\varphi(t) & =\E[\exp(ity)]=\int_{\mathbb{R}}\exp(itx)\frac{1}{\sqrt{2\pi}\sigma}\exp\left(-\frac{(x-\mu)^{2}}{2\sigma^{2}}\right)dx\\
 & =\frac{1}{\sqrt{2\pi}\sigma}\int_{\mathbb{R}}\exp(it(y+\mu))\exp\left(-\frac{y^{2}}{2\sigma^{2}}\right)dy\tag{by letting \ensuremath{y=x-\mu}}\\
 & =\frac{1}{\sqrt{2\pi}\sigma}e^{it\mu}\underbrace{\int_{\mathbb{R}}\exp(ity)\exp\left(-\frac{y^{2}}{2\sigma^{2}}\right)dy}_{:=\phi(t)}.
\end{align*}
In order to find a more explicit form of $\widehat{\mu}(t)$, we evaluate
$\phi(t).$ Now note that 
\begin{align*}
\left|\frac{\partial}{\partial t}\left(\exp(ity)\exp\left(-\frac{y^{2}}{2\sigma^{2}}\right)\right)\right| & =\left|iy\exp(ity)\exp\left(-\frac{y^{2}}{2\sigma^{2}}\right)\right|\leq y\exp\left(-\frac{y^{2}}{2\sigma^{2}}\right)\in L^{1}(\mathbb{R}).
\end{align*}
Then a corollary of DCT, we have 
\begin{align*}
\frac{\partial}{\partial t}\phi(t) & =\int\frac{\partial}{\partial t}\left\{ \exp(ity)\exp\left(-\frac{y^{2}}{2\sigma^{2}}\right)\right\} dy=\int iy\exp(ity)\exp\left(-\frac{y^{2}}{2\sigma^{2}}\right)dy\\
 & =\left[-(i\sigma^{2})^{2}\exp(ity)\exp\left(-\frac{y^{2}}{2\sigma^{2}}\right)\right]_{-\infty}^{\infty}-\sigma^{2}t\int_{-\infty}^{\infty}\exp(ity)\exp\left(-\frac{y^{2}}{2\sigma^{2}}\right)dy=-\sigma^{2}t\phi(t).
\end{align*}
Note that this is a first order differential equation. Moreover, observe
that we also have following initial condition: 
\[
\phi(0)=\int_{\mathbb{R}}\exp\left(-\frac{y^{2}}{2\sigma^{2}}\right)dy=\sqrt{2\pi}\sigma\underbrace{\frac{1}{\sqrt{2\pi}\sigma}\int_{\mathbb{R}}\exp\left(-\frac{y^{2}}{2\sigma^{2}}\right)dy}_{=1\text{ since it's gaussian density}}=\sqrt{2\pi}\sigma.
\]
Using integrating factor, we find its general solution if $\phi(t)=ce^{-\frac{1}{2}\sigma^{2}t^{2}}.$
Substituting back into the initial condition, we get that $c=\sqrt{2\pi}\sigma$
and as a result $\phi(t)=ce^{-\frac{1}{2}\sigma^{2}t^{2}}.$ Hence,
it follows that $\varphi(t)=e^{it\mu}e^{-\frac{1}{2}\sigma^{2}t^{2}}$
as desired. 
\end{proof}
\begin{lem}
The characteristic function for $\mathrm{MVN}(\mu,\Sigma)$ in $n$-dimensional
Euclidean space is given by \label{lem: cf of gaussian}
\[
\varphi(t)=\exp(i\left\langle t,\mu\right\rangle )\exp\left(-\frac{1}{2}\left\langle \Sigma t,t\right\rangle \right)
\]
\end{lem}
\begin{proof}
First, we follow the definition of multivariate characteristic function
to write 
\begin{align*}
\varphi(t) & =\int_{\mathbb{R}^{n}}\exp(i\left\langle t,x\right\rangle )\frac{1}{(2\pi)^{n/2}(\det\Sigma)^{1/2}}\exp\left(-\frac{1}{2}(x-\mu)^{T}\Sigma^{-1}(x-\mu\right)\\
 & =\int_{\mathbb{R}^{n}}\exp(i\left\langle t,y+\mu\right\rangle )\frac{1}{(2\pi)^{n/2}(\det\Sigma)^{1/2}}\exp\left(-\frac{1}{2}y^{T}\Sigma^{-1}y\right)\left|\det(J_{\varphi}(y))\right|dy\tag{let \ensuremath{x=\varphi(y)=y+\mu}}\\
 & =\frac{\exp(i\left\langle t,\mu\right\rangle )}{(2\pi)^{n/2}(\det\Sigma)^{1/2}}\underbrace{\int_{\mathbb{R}^{n}}\exp\left(i\left\langle t,y\right\rangle -\frac{1}{2}y^{T}\Sigma^{-1}y\right)}_{:=I_{1}(t)}dy\tag{1}.
\end{align*}
Note that since $\Sigma^{-1}$ is symmetric (by Problem 2.22), it
follows that $\Sigma^{-1}$ has an eigen-decomposition in the form
of $\Sigma^{-1}=V\Lambda V^{*}$ and $\Lambda=V^{*}\Sigma^{-1}V$,
where $\Lambda$ is a diagonal matrix containing eigen values which
are real and $V\in\mathrm{O}(n)$ according to Problem 2.18. Now we
make a change of variable as follows: 
\[
\varphi(x):\mathbb{R}^{n}\rightarrow\mathbb{R}^{n},x\mapsto Vx.\implies\left|\det J_{\varphi}(g)\right|=\left|\det V\right|=1.
\]
Then apply this change of variable to the $I_{1}(t)$ and we get 
\begin{align*}
I_{1}(t) & =\int_{\mathbb{R}^{n}}\exp\left(i\left\langle t,Vx\right\rangle -\frac{1}{2}x^{T}V^{T}\Sigma^{-1}Vx\right)dx=\int_{\mathbb{R}^{n}}\exp\left(i\left\langle V^{*}t,x\right\rangle -\frac{1}{2}x^{T}\Lambda x\right)dx.
\end{align*}
Now, we bring this result back to Eq.(1) and get 
\begin{align*}
\varphi(t) & =\frac{\exp(i\left\langle t,\mu\right\rangle )}{(2\pi)^{n/2}(\det\Sigma)^{1/2}}\int_{\mathbb{R}^{n}}\exp\left(i\left\langle s,x\right\rangle -\frac{1}{2}x^{T}\Lambda x\right)dx\tag{where \ensuremath{s=V^{*}t}}\\
 & =\frac{\exp(i\left\langle t,\mu\right\rangle )}{(2\pi)^{n/2}(\det\Sigma)^{1/2}}\int_{\mathbb{R}^{n}}\exp\left(i\sum_{i=1}^{n}s_{i}x_{i}-\frac{1}{2}\sum_{i=1}^{n}\frac{x_{i}^{2}}{\lambda_{i}}\right)dx\\
 & =\frac{\exp(i\left\langle t,u\right\rangle )}{(2\pi)^{n/2}(\det\Sigma)^{1/2}}\prod_{i=1}^{n}\int_{\mathbb{R}^{n}}\exp\left(is_{i}x_{i}-\frac{1}{2}\frac{x_{i}^{2}}{\lambda_{i}}\right)dx_{i}\tag{by Fubini's theroem}\\
 & =\exp(i\left\langle t,u\right\rangle )\prod_{i=1}^{n}\int_{\mathbb{R}^{n}}\exp(its_{i})\frac{1}{\sqrt{2\pi}\lambda_{i}^{1/2}}\exp\left(-\frac{1}{2}\frac{x_{i}^{2}}{\lambda_{i}}\right)dx_{i}\\
 & =\exp(i\left\langle t,u\right\rangle )\prod_{i=1}^{n}\varphi_{X_{i}\sim\mathrm{N}(0,\lambda_{i})}(s_{i})=\exp(i\left\langle t,\mu\right\rangle )\exp\left(\sum_{i=1}^{n}-\frac{\lambda_{i}s_{i}^{2}}{2_{i}}\right)\\
 & =\exp(i\left\langle t,\mu\right\rangle )\exp\left(-\frac{1}{2}t^{*}\Sigma t\right)=\exp(i\left\langle t,\mu\right\rangle )\exp\left(-\frac{1}{2}\left\langle \Sigma t,t\right\rangle \right),
\end{align*}
as desired. 
\end{proof}
Now since set of characteristic functions is an isomorphic to the
set of probability distributions (cf.???), we can alternatively define
Gaussian distribution using it's characteristic function. One advantage
of this characterization is the following lemma. 
\begin{lem}
Let $X$ be an $\mathbb{R}^{n}$-valued random variable such that
$X\sim\mathrm{MVN}(\mu,\Sigma).$ Then $X=_{d}\Sigma^{1/2}Z+\mu,$
where $Z\sim\mathrm{MVN}(0,I).$
\end{lem}
\begin{proof}
This is a standard result. For the sake of completeness, we provide
a complete proof here. Recall a useful lemma: 
\begin{lem}
Let $X$ be an $\mathbb{R}^{n}$-valued random variable. Then the
characteristic function for $AX+b$ where $A\in\mathrm{Mat_{\mathbb{K}}}(n,m)$
and $b\in\mathbb{R}^{m}$ can be characterized as $\varphi_{AX+b}=e^{i\left\langle t,b\right\rangle }\varphi_{X}(A^{*}t).$\label{lem: characteristic function of AX+b}
\end{lem}
\begin{proof}[Proof of \ref{lem: characteristic function of AX+b}]
 Following the definition, we have 
\begin{align*}
\varphi_{AX+b}(t) & =\int\exp\left(i\left\langle t,AX+b\right\rangle \right)d\Omega=\exp(i\left\langle t,b\right\rangle )\int\exp(i\left\langle t,Ax\right\rangle )d\Omega=\exp(i\left\langle t,b\right\rangle )\int\exp(i\left\langle A^{*}t,x\right\rangle )d\Omega\\
 & =\exp(i\left\langle t,b\right\rangle \varphi_{X}(A^{*}t),
\end{align*}
as desired. 
\end{proof}
Now in view of \ref{lem: characteristic function of AX+b} we have
\begin{align*}
\varphi_{\Sigma^{1/2}Z+\mu} & =\exp(i\left\langle t,b\right\rangle )\varphi_{Z}(\Sigma^{1/2}t)=\exp(i\left\langle t,b\right\rangle )\exp\left(-\frac{1}{2}\left\langle \Sigma^{1/2}t,\Sigma^{1/2}t\right\rangle \right)\\
 & =\exp(i\left\langle t,b\right\rangle )\exp\left(-\frac{1}{2}\left\langle \Sigma^{1/2}\Sigma^{1/2}t,t\right\rangle \right)=\exp(i\left\langle t,b\right\rangle )\exp\left(-\frac{1}{2}\left\langle \Sigma t,t\right\rangle \right)\\
 & =\varphi_{X}(t).
\end{align*}
Hence, it follows that $X=_{d}\Sigma^{1/2}Z+\mu$ as desired. 
%
\end{proof}
%
\begin{proof}[Proof of \ref{lem: distribution of linear transformation of gaussian}]
 In view of \ref{lem: cf of gaussian}, \ref{lem: characteristic function of AX+b},
we have 
\begin{align*}
\varphi_{AX+b}(t) & =\exp(i\left\langle t,b\right\rangle )\varphi_{X}(A^{*}t)\\
 & =\exp(i\left\langle t,b\right\rangle )\exp(i\left\langle A^{*}t,\mu\right\rangle )\exp\left(-\frac{1}{2}\left\langle \Sigma A^{*}t,A^{*}t\right\rangle \right)\\
 & =\exp(i\left\langle t,A\mu,b\right\rangle )\exp\left(-\frac{1}{2}\left\langle A\Sigma A^{*}t,t\right\rangle \right)\\
 & =\varphi_{Y\sim\mathrm{MVN}(A\mu+b,A\Sigma A^{*})}(t).
\end{align*}
Hence, it follows that $AX+b=_{d}\mathrm{MVN}(A\mu+b,A\Sigma A^{*})$. 
\end{proof}
%
Now we go back to the problem itself. Instead of $x_{1},x_{2}$, we
use $X_{1},X_{2}$ to denote the designated r.v, i.e., $X_{1}\sim\mathrm{N}(\mu_{1},\tau_{1}^{-1})$
and $X_{2}\sim\mathrm{N}(\mu_{2},\tau_{2}^{-1})$ and $X_{1}\perp X_{2}.$
Then it follows that the random vector $\widetilde{X}=\begin{bmatrix}X_{1}\\
X_{2}
\end{bmatrix}\sim\mathrm{MVN}\left(\begin{bmatrix}\mu_{1}\\
\mu_{2}
\end{bmatrix},\begin{bmatrix}\tau_{1}^{-1} & 0\\
0 & \tau_{2}^{-1}
\end{bmatrix}\right)$. Note that since $X=\mathbf{1}^{T}\widetilde{X}$ , an application
of \ref{lem: distribution of linear transformation of gaussian} we
have that 
\[
X\sim\mathrm{N}\left(\mathbf{1}^{T}\begin{bmatrix}\mu_{1}\\
\mu_{2}
\end{bmatrix},\mathbf{1}^{T}\Sigma\mathbf{1}\right)=\mathrm{N}(\mu_{1}+\mu_{2},\tau_{1}^{-1}+\tau_{2}^{-1}).
\]
Then, by Problem 1.35, it follows that $H(X_{1}+X_{2})=\frac{1}{2}(1+\ln(2\pi(\tau_{1}^{-1}+\tau_{2}^{-1}))).$

\paragraph{Alternative derivation of gaussian mean, covariance}

In the textbook, the mean and covariance matrix of MVN are derived
using change of variables techniques when evaluating the integral.
Since we have mentioned that we can characterize a distribution using
its characteristic function. Naturally it comes the question of deriving
moments of random variables from their characteristic function. Here,
we provide a general solution to this problem and apply it to the
Gaussian case. 
\begin{lem}
Let $X$ be a $\mathbb{R}^{n}$-valued random variable with $\E[\left\Vert X\right\Vert ^{N}]<\infty.$
Then 
\[
\mathrm{D}^{\alpha}\varphi_{X}(t)=i^{\left|\alpha\right|}\int X^{\alpha}e^{i\left\langle t,X\right\rangle }d\Omega,
\]
where $\alpha=(\alpha_{1},\alpha_{2},...,\alpha_{n})$ denote the
multi-index such that $\left|\alpha\right|:=\sum_{i=1}^{n}\alpha_{i}\leq N$
and $X^{\alpha}:=X_{1}^{\alpha_{1}}X_{2}^{\alpha_{2}}\cdots X_{n}^{\alpha_{n}}.$
As a result, 
\[
i^{\left|\alpha\right|}\E[X^{\alpha}]=\mathrm{D}^{\alpha}\varphi_{X}(0).
\]
\end{lem}
\begin{proof}
To prove this, we induct on $N$. Now for the base of $N=1,$ we note
that $\mathrm{D}^{\alpha}\varphi_{X}(t)$ then is reduces to $\frac{\partial}{\partial t_{i}}\varphi_{X}(t)$
for some $i\in\{1,...,n\}.$ Note $\frac{\partial}{\partial t_{i}}\varphi_{X}(t)=\frac{\partial}{\partial t_{i}}\int e^{i\left\langle t,X\right\rangle }d\Omega$
and that 
\[
\left|\frac{\partial}{\partial t_{i}}\left(\exp(i\left\langle t,X\right\rangle \right)\right|=\left|iX_{i}\exp\left(i\left\langle t,X\right\rangle \right)\right|\leq\left|X_{i}\right|.
\]
We claim that $\left|X_{i}\right|\in L^{1}(\Omega).$ To see this,
first we note that by Jensen's inequality $(\E[\left\Vert X\right\Vert ])^{N}\leq\E[\left\Vert X\right\Vert ^{N}]<\infty.$
Take the $N$-th root, we see that $\left\Vert X\right\Vert \in L^{1}.$
Clearly, we have $|X_{i}|\leq(\sum_{i=1}^{n}X_{i}^{2})^{1/2}$. Hence,
chaining these inequalities shows that $\left|X_{i}\right|\in L^{1}(\Omega).$
Then by a variant of DCT, we can move the differentiation inside and
get 
\[
\frac{\partial}{\partial t_{i}}\varphi_{X}(t)=\int\frac{\partial}{\partial t_{i}}\exp(i\left\langle t,X\right\rangle )d\Omega=i\int X_{i}\exp(i\left\langle t,X\right\rangle )d\Omega.
\]
Now assume that the claim holds for $N=n-1.$ Then for $N=n,$ we
first note that for some $\alpha$ with $\left|\alpha\right|=n,$
$\mathrm{D}^{\alpha}\varphi_{X}(t)=\frac{\partial}{\partial t_{i}}(\mathrm{D}^{\beta}\varphi_{X}(t))$
for some multi-index $\beta$ such that $\left|\beta\right|=n-1,$
and some $i\in\{1,...,n\}.$ Then, we note that first by inductive
hypothesis, $\mathrm{D}\beta\varphi_{X}(t)=i^{\left|\beta\right|}\int X^{\beta}\exp^{i\left\langle t,X\right\rangle }d\Omega$
and second, 
\[
\left|\frac{\partial}{\partial t_{i}}X^{\beta}\exp(i\left\langle t,X\right\rangle )\right|=\left|iX_{i}X^{\beta}\exp(i\left\langle t,X\right\rangle )\right|\leq\left|X^{\alpha}\right|=\prod_{i=1}^{n}\left|X_{i}\right|^{\alpha_{i}}.
\]
Now, we claim that $\prod_{i=1}^{n}\left|X_{i}\right|^{\alpha_{i}}\in L^{1}(\Omega)$.
To see this we note that for since $\sum_{i=1}^{n}\alpha_{i}=N,$
it follows that $\alpha_{i}\leq N.$ As a result, $\left|X_{i}\right|^{\alpha_{i}}\leq\left\Vert X\right\Vert ^{\alpha_{i}}.$
Therefore, $\prod_{i=1}^{n}\left|X_{i}\right|^{\alpha_{i}}\leq\prod_{i=1}^{n}\left\Vert X\right\Vert ^{\alpha_{i}}=\left\Vert X\right\Vert ^{\sum_{i=1}^{n}\alpha_{i}}=\left\Vert X\right\Vert ^{N}\in L^{1}(\Omega).$
Again, by the variant of DCT, we have 
\begin{align*}
\mathrm{D}^{\alpha}\varphi_{X}(t) & =\frac{\partial}{\partial t_{i}}i^{\left|\beta\right|}\int X^{\beta}\exp(i\left\langle t,X\right\rangle )d\Omega=i^{\left|\beta\right|}\int\frac{\partial}{\partial t_{i}}\left(X^{\beta}\exp(i\left\langle t,X\right\rangle )\right)d\Omega\\
 & =i^{\left|\beta\right|}\int iX^{\alpha}\exp(i\left\langle t,X\right\rangle )d\Omega=i^{\left|\beta\right|+1}\int X^{\alpha}\exp(i\left\langle t,X\right\rangle )d\Omega\\
 & =i^{\left|\alpha\right|}\int X^{\alpha}\exp(i\left\langle t,X\right\rangle )d\Omega,
\end{align*}
as desired. 
\end{proof}
Now we use this result to derive the mean and covariance of a MVN
random variable, which we denote as $X\sim\mathrm{MVN}(\mu,\Sigma).$
Recall that by \ref{lem: cf of gaussian}, $\varphi_{X}(t)=\exp(i\left\langle t,\mu\right\rangle -\frac{1}{2}\left\langle \Sigma t,t\right\rangle ).$
Now we find the Frechet derivative w.r.t $t$: note that by the chain
rule:
\begin{align*}
\mathrm{D}\varphi_{X}(t) & =\mathrm{D}\left(\exp(t)\right)\circ\left(t\mapsto i\left\langle t,\mu\right\rangle -\frac{1}{2}\left\langle \Sigma t,t\right\rangle \right)\circ\mathrm{D}\bigg(\underbrace{i\left\langle t,\mu\right\rangle -\frac{1}{2}\left\langle \Sigma t,t\right\rangle }_{:=H_{1}(t)}\bigg).
\end{align*}
Note that 
\begin{align*}
H_{1}(t+h) & =i\left\langle t+h,\mu\right\rangle -\frac{1}{2}\left\langle \Sigma(t+h),t+h\right\rangle \\
 & =H_{1}(t)+i\left\langle h,\mu\right\rangle -\frac{1}{2}\left\langle \Sigma h,t\right\rangle -\frac{1}{2}\left\langle \Sigma t,h\right\rangle -\frac{1}{2}\left\langle \Sigma h,h\right\rangle \\
 & =H_{1}(t)+i\left\langle h,\mu\right\rangle -\frac{1}{2}\left\langle \Sigma h,t\right\rangle -\frac{1}{2}\left\langle t,\Sigma^{*}h\right\rangle -\frac{1}{2}\left\langle \Sigma h,h\right\rangle \\
 & =H_{1}(t)+i\left\langle h,\mu\right\rangle -\left\langle \Sigma h,t\right\rangle -\frac{1}{2}\left\langle \Sigma h,h\right\rangle \\
 & =H_{1}(t)+\left\langle i\mu-\Sigma t,h\right\rangle -\frac{1}{2}\left\langle \Sigma h,h\right\rangle .
\end{align*}
Since $\left\langle \Sigma h,h\right\rangle \leq\left\Vert \Sigma\right\Vert _{\infty}\left\Vert h\right\Vert ^{2}\rightarrow0$
at $\left\Vert h\right\Vert \rightarrow0,$ it follows that $\frac{1}{2}\left\langle \Sigma h,h\right\rangle =o(\left\Vert h\right\Vert ).$
As $h\mapsto\left\langle i\mu-\Sigma t,h\right\rangle \in\text{Hom}(\mathbb{R}^{n},\mathbb{R}),$
it follows that $\mathrm{D}H_{1}(t)=i\mu-\Sigma t.$ Since $\mathrm{D}(\exp(t))=\exp(t)$
by elementary calculus, it follows that 
\[
\mathrm{D}\varphi_{X}(t)=\exp(i\left\langle t,\mu\right\rangle -\frac{1}{2}\left\langle \Sigma t,t\right\rangle )(i\mu-\Sigma t).
\]
Therefore, $\E[X]=\mathrm{D}\varphi_{X}(0)=i^{-1}i\mu=\mu.$ \medskip\\
Now to find the covariance, we can either partial differentiate term
by term or use the same notion of Frechet derivative. We adopt the
second method since it is consistent with our previous method, and
also yields the total derivative in its matrix form directly, which
is more elegant than piecing together terms. Before we go into the
calculation, we prepare ourselves with a tool to facilitate the calculation
- generalized product rule. 
\begin{lem}
Suppose the mapping $B:X_{1}\times X_{2}\rightarrow Y$ is bilinear
and bounded, i.e., 
\[
\left\Vert B(x_{1},x_{2})\right\Vert \leq C\left\Vert x_{1}\right\Vert \left\Vert x_{2}\right\Vert \text{ for all }x_{1}\in X_{1},x_{2}\in X_{2}
\]
where $C$ is fixed and $B$ linear in each argument. Suppose further
that the maps $f_{i}:X\rightarrow X_{i},$ $i=1,2$ are Frechet differentiable
at $x$, and there exist and open set $U$ such that $x\in U$ and
$U\subseteq\mathcal{D}_{f_{i}}.$ Then the function $H(x)=B(f_{1}(x),f_{2}(x))$
is differentiable at $x$, and \label{lem: prod rule}
\[
\mathrm{D}H(x)(h)=B(\mathrm{D}f_{1}(x)(h),f_{2}(x))+B(f_{1}(x),\mathrm{D}f_{2}'(x)(h)).
\]
Note: $X_{1},X_{2},X,Y$ are all assumed to be Banach spaces. 
\end{lem}
\begin{proof}
We follow the definition and write out $H(x+h)-H(x)$ for later analysis.
To facilitate notation, we let $f_{i}^{x}:=f_{i}(x)$ for $i=1,2.$
\begin{align*}
H(x+h)-H(x) & =B(f_{1}^{x+h},f_{2}^{x+h})-B(f_{1}^{x},f_{2}^{x})\\
 & =B(f_{1}^{x+h},f_{2}^{x+h})-B(f_{1}^{x+h},f_{2}^{x})+B(f_{1}^{x+h},f_{2}^{x})-B(f_{1}^{x},f_{2}^{x})\\
 & =B(f_{1}^{x+h},f_{2}^{x+h}-f_{2}^{x})+B(f_{1}^{x+h}-f_{1}^{x},f_{2}^{x})\\
 & =B(f_{1}^{x}+\mathrm{D}f_{1}^{x}(h)+\left\Vert h\right\Vert r_{1}(h),\mathrm{D}f_{2}^{x}(h)+\left\Vert h\right\Vert r_{2}(h))+B(\mathrm{D}f_{1}^{x}(h)+\left\Vert h\right\Vert r_{1}(h),f_{2}^{x})\\
 & =T_{x}(h)+R_{x}(h),
\end{align*}
where 
\[
\begin{cases}
\begin{split}T_{x}(h)= & B(f_{1}^{x},\mathrm{D}f_{2}^{x}(h))+B(\mathrm{D}f_{1}^{x}(h),f_{2}^{x})\\
R_{x}(h)= & B(f_{1}^{x},\left\Vert h\right\Vert r_{2}(h))+B(\mathrm{D}f_{1}^{x}(h),\mathrm{D}f_{2}^{x}(h))+B(\mathrm{D}f_{1}^{x}(h),\left\Vert h\right\Vert \\
 & +B(\left\Vert h\right\Vert r_{1}(h),\left\Vert h\right\Vert r_{2}(h))+B(\left\Vert h\right\Vert r_{1}(h),f_{2}^{x}).
\end{split}
r_{2}(h))+B(\left\Vert h\right\Vert _{1}r_{1}(h),\mathrm{D}f_{2}^{x}(h))\end{cases}
\]
In order to show that $T_{x}(h)=\mathrm{D}H(x)\circ h.$ We first
need to show that $T(h)\in\mathrm{Hom}(X,Y).$ Indeed, 
\begin{align*}
T_{x}(\alpha h+\beta g) & =B(f_{1}^{x},\mathrm{D}f_{2}(\alpha h+\beta g))+B(\mathrm{D}f_{1}^{x}(\alpha h+\beta g),f_{2}^{x})\\
 & =\alpha B(f_{1}^{x},\mathrm{D}f_{2}(h))+\beta B(f_{1}^{x},\mathrm{D}f_{2}(g))+\alpha B(\mathrm{D}f_{1}^{x}(h),f_{2}^{x})+\beta B(\mathrm{D}f_{1}^{x}(g),f_{2}^{x})\\
 & =\alpha T_{x}(h)+\beta T_{x}(g).
\end{align*}
Next, we need to show that $R_{x}(h)=o(\left\Vert h\right\Vert ).$
We analyze $R_{x}(h)$ term by term as follows 
\[
\begin{cases}
\begin{alignedat}{1} & B(f_{1}^{x},\left\Vert h\right\Vert r_{2}(h))\leq C\left\Vert f_{1}^{x}\right\Vert \left\Vert h\right\Vert \left\Vert r_{2}(h)\right\Vert =o(\left\Vert h\right\Vert )\\
 & B(\mathrm{D}f_{1}^{x}(h),\mathrm{D}f_{2}^{x}(h))\leq C\left\Vert \mathrm{D}f_{1}^{x}\right\Vert \left\Vert \mathrm{D}f_{2}^{x}\right\Vert \left\Vert h\right\Vert ^{2}=o(\left\Vert h\right\Vert )\\
 & B(\mathrm{D}f_{1}^{x}(h),\left\Vert h\right\Vert r_{2}(h)\leq C\left\Vert \mathrm{D}f_{1}^{x}\right\Vert \left\Vert h\right\Vert ^{2}\left\Vert r_{2}(h)\right\Vert =o(\left\Vert h\right\Vert )\\
 & B(\left\Vert h\right\Vert _{1}r_{1}(h),\mathrm{D}f_{2}^{x}(h))\leq C\left\Vert \mathrm{D}f_{2}^{x}\right\Vert \left\Vert h\right\Vert ^{2}\left\Vert r_{1}(h)\right\Vert =o(\left\Vert h\right\Vert )\\
 & B(\left\Vert h\right\Vert r_{1}(h),\left\Vert h\right\Vert r_{2}(h))\leq C\left\Vert h\right\Vert ^{2}\left\Vert r_{1}(h)\right\Vert \left\Vert r_{2}(h)\right\Vert =o(\left\Vert h\right\Vert )\\
 & B(\left\Vert h\right\Vert r_{1}(h),f_{2}^{x})\leq C\left\Vert f_{2}^{x}\right\Vert \left\Vert h\right\Vert \left\Vert r_{1}(h)\right\Vert =o(\left\Vert h\right\Vert ).
\end{alignedat}
.\end{cases}
\]
Since $R_{x}(h)$ is the sum of these terms, it follows that $R_{x}(h)=o(\left\Vert h\right\Vert )$.
Hence, it follows that $T_{x}(h)=\mathrm{D}H(x)\circ h.$ 
\end{proof}
%
Now, observe that $\mathrm{D}\varphi_{X}(t)=\exp(i\left\langle t,\mu\right\rangle -\frac{1}{2}\left\langle \Sigma t,t\right\rangle )(i\mu-\Sigma t)=B(\exp(i\left\langle t,\mu\right\rangle -\frac{1}{2}\left\langle \Sigma t,t\right\rangle ),i\mu-\Sigma t),$
where $B\in\mathrm{Hom}(\mathbb{R},\mathbb{R}^{n};\mathbb{R}^{n})$
is defined by $(x,y)\mapsto xy.$ Hence, by \ref{lem: prod rule},
it follows that 
\begin{align*}
\mathrm{D}^{2}\varphi_{X}(t)(h) & =B\left(\left\langle \exp\left(i\left\langle t,\mu\right\rangle -\frac{1}{2}\left\langle \Sigma t,t\right\rangle \right)(i\mu-\Sigma t),h\right\rangle ,i\mu-\Sigma t\right)+B\left(\exp\left(i\left\langle t,\mu\right\rangle -\frac{1}{2}\left\langle \Sigma t,t\right\rangle \right),-\Sigma h\right)\\
 & =\exp\left(i\left\langle t,\mu\right\rangle -\frac{1}{2}\left\langle \Sigma t,t\right\rangle \right)(i\mu-\Sigma t)(i\mu-\Sigma t)^{T}h-\exp\left(i\left\langle t,\mu\right\rangle -\frac{1}{2}\left\langle \Sigma t,t\right\rangle \right)\Sigma h\\
 & =\exp\left(i\left\langle t,\mu\right\rangle -\frac{1}{2}\left\langle \Sigma t,t\right\rangle \right)((i\mu-\Sigma t)(i\mu-\Sigma t)^{T}h-\Sigma h).
\end{align*}
Therefore, 
\[
\mathrm{D}\varphi_{X}(0)h=(i^{2}\mu\mu^{T}-\Sigma)h=(-\mu\mu^{T}-\Sigma)h\implies\E[XX^{T}]=-\mathrm{D}^{2}\varphi_{X}(0)=\mu\mu^{T}+\Sigma.
\]
 As a result, we have according to definition the covariance matrix
is $\E[XX^{T}]-\mu\mu^{T}=\Sigma.$\\

\begin{cBoxA}[Problem 2.17 - Suffices to assume the parameter $\Sigma$ in Gaussian
to be symmetric]{}
 Consider the multivariate Gaussian distribution given by (2.43).
By writing the precision matrix (inverse covariance matrix) $\Sigma^{-1}$
as the sum of a symmetric and an anti-symmetric matrix, show that
the anti-symmetric term does not appear in the exponent of the Gaussian,
and hence that the precision matrix may be taken to be symmetric without
loss of generality. Because the inverse of a symmetric matrix is also
symmetric (see Exercise 2.22), it follows that the covariance matrix
may also be chosen to be symmetric without loss of generality. 
\end{cBoxA}

This is an direct application of Problem 1.14. Recall the MVN in $n$-dimensional
space has density in the following form:
\[
\frac{1}{(2\pi)^{n/2}(\det\Sigma)^{1/2}}\exp\left(-\frac{1}{2}(x-\mu)^{T}\Sigma^{-1}(x-\mu)\right).
\]
Since $(x-\mu)^{T}\Sigma(x-\mu)$ is a bilinear form. In problem 1.14,
we showed that it suffices to assume $\Sigma^{-1}=\Sigma_{S}^{^{-1}}=\frac{1}{2}(\Sigma^{-1}+(\Sigma^{-1})^{T}),$
which is symmetric since $(x-\mu)^{T}\Sigma^{-1}(x-\mu)=(x-\mu)^{T}\Sigma_{S}^{-1}(x-\mu)$
for all $x\in\mathbb{R}^{n}.$\\

\begin{cBoxA}[Problem 2.18 - Eigen-decomposition for symmetric matrices ]{}
 Consider a real, symmetric matrix \textgreek{S} whose eigenvalue
equation is given by (2.45). By taking the complex conjugate of this
equation and subtracting the original equation, and then forming the
inner product with eigenvector ui, show that the eigenvalues \textgreek{l}i
are real. Similarly, use the symmetry property of \textgreek{S} to
show that two eigenvectors ui and uj will be orthogonal provided \textgreek{l}j
�= \textgreek{l}i. Finally, show that without loss of generality,
the set of eigenvectors can be chosen to be orthonormal, so that they
satisfy (2.46), even if some of the eigenvalues are zero. 
\end{cBoxA}

Before go into the proof, a lemma. This lemma is usually known as
Gram-Schmidt orthogonalization. We prove it in the context of Hilbert
space. Proof of this result at various level of generality can be
found in any standard linear algebra textbook. 
\begin{lem}
Let $\mathcal{H}$ be an Hilbert space. Suppose $\mathcal{U}=\{u_{1},u_{2},\ldots,u_{n}\}$
is a set of linearly independent vectors of $V.$ Then there exists
a set $\mathcal{V=}\{v_{1},v_{2},\ldots,v_{n}\}$ of elements of $\mathcal{H}$
such that \label{lem: gram schimt}
\begin{enumerate}
\item $\left\Vert v_{i}\right\Vert =1$ for $1\leq i\leq n.$
\item $\left\langle v_{i},v_{j}\right\rangle =0$ for $1\leq i,j,\leq n$
with $i\neq j$. 
\item $\mathrm{span}(\mathcal{V})=\mathrm{span}(\mathcal{U}).$
\end{enumerate}
\end{lem}
\begin{proof}
We prove this constructively. The construction goes as follows: first
we let $v_{1}=\frac{u_{1}}{\left\Vert u_{1}\right\Vert }$ and $v_{2}=\frac{w_{2}}{\left\Vert w_{2}\right\Vert },$
where $w_{2}=u_{2}-\left\langle u_{2},v_{1}\right\rangle v_{1}$ ,
and inductively $v_{i}=\frac{w_{i}}{\left\Vert w_{i}\right\Vert },$
where $w_{i}=u_{i}-\sum_{j=1}^{i-1}\left\langle u_{i},v_{j}\right\rangle v_{j},$
assuming $v_{1},...,v_{i-1}$ have already been defined. \medskip\\
To prove the correctness of our construction, we induct on $n.$ For
the base case where $k=1.$ We see that (2) and (3) is trivially satisfied
for $v_{1}=u_{1}/\left\Vert u_{1}\right\Vert .$ Also, we have $\left\Vert v_{1}\right\Vert =\left\Vert v_{1}\right\Vert /\left\Vert v_{1}\right\Vert =1.$
Now suppose, the construction yields the desired set of vectors $\mathcal{V}_{k=n-1}$
that satisfies condition (1)-(3) for a given set of vectors $\mathcal{U}_{n-1}$.
Then for $k=n,$we are given a set of elements in $\mathcal{H}$,
$\mathcal{U}_{n}=\{u_{1},\ldots,u_{n}\}$. By induction hypothesis,
we can construct a set of vectors $\mathcal{V}_{n-1}=\{v_{1},\ldots,v_{n-1}\}$
from the set $\mathcal{U}_{n}-\{u_{n}\}$ that satisfies the conditions
(1)-(3) stipulated above with $\mathcal{V}=\mathcal{V}_{n-1}$ and
$\mathcal{U}=\mathcal{U}_{n-1}$. Now let $v_{n}=w_{n}/\left\Vert w_{n}\right\Vert ,$
where $w_{n}=u_{n}-\sum_{i=1}^{n-1}\left\langle u_{n},v_{i}\right\rangle v_{i}.$
First, we show that $v_{n}$ is well-defined. To prove this, we show
that $w_{n}\notin\mathrm{span}(\mathcal{V}_{n-1}).$ Suppose otherwise,
then $w_{n}=\sum_{i=1}^{n-1}\alpha_{i}v_{n}$ for some scalars $\{\alpha_{i}\}_{i=1}^{n-1}.$
Then we have 
\[
w_{n}=\sum_{i=1}^{n-1}\alpha_{i}v_{i}=u_{n}-\sum_{i=1}^{n-1}\left\langle u_{n},v_{i}\right\rangle v_{i}\implies u_{n}=\sum_{i=1}^{n-1}(\alpha_{i}+\left\langle u_{n},v_{i}\right\rangle )v_{i}\in\mathrm{span}(\mathcal{V}_{n-1}).
\]
Since $\mathrm{span}(\mathcal{V}_{n-1})=\mathrm{span}(\mathcal{U}_{n-1})$,
it follows that $u_{n}\in\mathrm{span}(\mathcal{U}_{n-1}).$ This
is a contradiction since then $u_{n}$ are independent of $u_{1},...,u_{n-1}$
by assumption. We claim that $\mathcal{V}_{n-1}\cup\{v_{n}\}$ satisfies
conditions (1)-(3) with $\mathcal{U}=\mathcal{U}_{n}$ and $\mathcal{V}=\mathcal{V}_{n-1}\cup\{v_{n}\}$.
Note that by construction $\left\Vert v_{n}\right\Vert =1.$ And since
for $j\in\{1,...,n-1\}$
\begin{align*}
\left\langle w_{n},v_{j}\right\rangle  & =\left\langle u_{n}-\sum_{i=1}^{n-1}\left\langle u_{n},v_{i}\right\rangle v_{i},v_{j}\right\rangle \\
 & =\left\langle u_{n},v_{j}\right\rangle -\sum_{i=1}^{n-1}(\left\langle u_{n},v_{i}\right\rangle \left\langle v_{i},v_{j}\right\rangle )\\
 & =\left\langle u_{n}-v_{j}\right\rangle -\left\langle u_{n}-v_{j}\right\rangle =0,
\end{align*}
it follows that $\left\langle w_{n}/\left\Vert w_{n}\right\Vert ,v_{j}\right\rangle =\left\langle v_{n},v_{j}\right\rangle =0$
for all $j=1,...,n$. What's left to prove is that $\mathrm{span}(\mathcal{V}_{n-1}\cup\{v_{n}\}\}=\mathrm{span}(\mathcal{U}_{n}).$
Pick $\mathrm{span}(\mathcal{V}_{n-1}\cup\{v_{n}\})\ni x=\sum_{i=1}^{n}\alpha_{i}v_{i}.$
If $\alpha_{n}=0,$ then $x\in\mathrm{span}(\mathcal{V}_{n-1})=\mathrm{span}(\mathcal{U}_{n-1})\subset\mathrm{span}(\mathcal{U}_{n}).$
If $\alpha_{n}\neq0,$ then 
\begin{align*}
x & =\sum_{i=1}^{n-1}\alpha_{i}v_{i}+\frac{\alpha_{n}}{\left\Vert u_{n}-\sum_{i=1}^{n-1}\left\langle u_{n},v_{i}\right\rangle v_{i}\right\Vert }\left(u_{n}-\sum_{i=1}^{n-1}\left\langle u_{n},v_{i}\right\rangle v_{i}\right)\\
 & =\sum_{i=1}^{n-1}(\alpha_{i}-\left\langle u_{i},v_{i}\right\rangle )v_{i}+\frac{\alpha_{n}}{\left\Vert u_{n}-\sum_{i=1}^{n-1}\left\langle u_{n},v_{i}\right\rangle v_{i}\right\Vert }u_{n}\tag{combining coefficients}\\
 & =\sum_{i=1}^{n-1}\beta_{i}u_{i}+\frac{\alpha_{n}}{\left\Vert u_{n}-\sum_{i=1}^{n-1}\left\langle u_{n},v_{i}\right\rangle v_{i}\right\Vert }u_{n}.
\end{align*}
Therefore, $x\in\mathrm{span}(\mathcal{U}_{n})$. On the other hand,
suppose $x\in\mathrm{span}(\mathcal{U}_{n}),$ i.e. $x=\sum_{i=1}^{n}\alpha_{i}u_{i}$.
If $\alpha_{n}=0,$ then $x\in\mathrm{span}(\mathcal{U}_{n-1})=\mathrm{span}(\mathcal{V}_{n-1})$
by induction hypothesis. If $\alpha_{n}\neq0,$ then we have 
\begin{align*}
x & =\sum_{i=1}^{n-1}\alpha_{i}u_{i}+u_{n}=\sum_{i=1}^{n-1}\beta_{i}v_{i}+u_{n}-\sum_{i=1}^{n-1}\left\langle u_{n},v_{i}\right\rangle v_{i}+\sum_{i=1}^{n-1}\left\langle u_{n},v_{i}\right\rangle v_{i}\\
 & =\sum_{i=1}^{n-1}(\beta_{i}+\left\langle u_{n},v_{i}\right\rangle )v_{i}+v_{n},
\end{align*}
whence $x\in\mathrm{span}(\mathcal{V}_{n-1}\cup\{v_{n}\}).$ 
\end{proof}
\begin{enumerate}[leftmargin={*}]
\item  We first show that the eigenvalues are real. First, we note that
by definition, the pair $(\lambda_{i},\mu_{i})$ is an eigenvector,
eigenvalue pair iff $\Sigma\mu_{i}=\lambda_{i}\mu.$ Now we fix one
such pair $(\mu_{i},\lambda_{i})$. Multiplying $\mu_{i}^{*}$ on
the left on both side of the equation yields $\mu^{*}\Sigma\mu_{i}=\lambda_{i}\mu_{i}^{*}\mu_{i}.$
Now since $\Sigma$ is symmetric, it follows that $(\mu^{*}\Sigma\mu)^{*}=\mu^{*}\Sigma\mu=\lambda_{i}^{*}\mu_{i}^{*}\mu_{i}.$
Hence, it follows that 
\[
\lambda_{i}^{*}\mu_{i}^{*}\mu_{i}=\lambda_{i}\mu_{i}^{*}\mu_{i}\iff(\lambda_{i}^{*}-\lambda_{i})\mu_{i}^{*}\mu_{i}=0\iff\lambda_{i}^{*}=\lambda_{i},
\]
since we assume $\mu_{i}\neq0.$ Therefore, $\lambda_{i}$ is real.
Since $(\mu_{i},\lambda_{i})$ is chosen to be arbitrary, it follows
that all eigen values in this case are real. 
\item Next, we show that for eigen-pair $(\lambda_{i},\mu_{i}),(\lambda_{j},\mu_{j})$
with $\lambda_{i}\neq\lambda_{j},$ and $\lambda_{i},\lambda_{j}\neq0$,
we have $\left\langle \mu_{i},\mu_{j}\right\rangle =0.$ First, note
the following identity: 
\[
\lambda_{j}\mu_{j}^{*}\mu_{i}=(\Sigma\mu_{j})^{*}\mu_{i}=(\Sigma^{*}\mu_{j})^{*}\mu_{i}=\mu_{j}^{*}\Sigma\mu_{i}=\mu_{j}^{*}\lambda_{i}\mu_{i}=\lambda_{i}\mu_{j}^{*}\mu_{i},
\]
which implies $\lambda_{i}\mu_{j}^{*}\mu_{i}=\lambda_{j}\mu_{j}^{*}\mu_{i}\Leftrightarrow\mu_{j}^{*}\mu_{i}(\lambda_{i}-\lambda_{j})=0.$
Since $\lambda_{i}\neq\lambda_{j}$ by assumption, it follows that
$\left\langle \mu_{j},\mu_{i}\right\rangle =\mu_{j}^{*}\mu_{i}=0$,
as desired. 
\item Now we show that for eigen-pair $(\lambda_{i},\mu_{i}),(\lambda_{j},\mu_{j})$
with $\lambda_{i}=\lambda_{j}$ and $\lambda_{i},\lambda_{j}\neq0,$
we can still have $\left\langle \mu_{i},\mu_{j}\right\rangle =0$.
For better notation, suppose $\lambda_{1}=\lambda_{2}:=\lambda\in\mathbb{R}.$
First, we show that any linear combination of $\mu_{i}$ and $\mu_{j}$
is also an eigen vector with the same eigen value, i.e., $(\lambda,\alpha\mu_{i}+\beta\mu_{2})$
is a valid eigen pair as well for any $\alpha,\beta\in\mathbb{R}-\{0\}.$
Indeed, we have 
\[
\Sigma(\alpha\mu_{i}+\beta\mu_{j})=\alpha\Sigma\mu_{i}+\beta\Sigma\mu_{j}=\lambda\alpha\mu_{i}+\lambda\beta\mu_{j}=\lambda(\alpha\mu_{i}+\beta\mu_{j}).
\]
Therefore, in view of \ref{lem: gram schimt}, we can orthonormalize
$\mu_{i}$ and $\mu_{j}$ to $\widetilde{\mu}_{i}$ and $\widetilde{\mu}_{j}$
such that $(\lambda,\tilde{\mu}_{i})$ and $(\lambda,\widetilde{\mu}_{j})$
are eigen-pairs as well ($\widetilde{\mu}_{1},\widetilde{\mu}_{2}$
are linear combination of $\mu_{1}$ and $\mu_{2}$). 
\item Now we show that for eigen-pair $(\lambda_{i},\mu_{i}),(\lambda_{j},\mu_{j})$
with at least one of $\lambda_{i}$ and $\lambda_{j}$ being equal
to 0, we can still have $\left\langle \mu_{i},\mu_{j}\right\rangle =0$.
Without loss of generality, suppose $\lambda_{i}=0.$ Then, this means
that $\Sigma\mu_{i}=0.$ Now suppose $\lambda_{j}\neq0,$ then we
have that $\Sigma\mu_{j}=\lambda_{j}\mu_{j}\implies\mu_{j}=\Sigma\mu_{j}/\lambda_{j}.$
Then it follows that 
\[
\left\langle \mu_{i},\mu_{j}\right\rangle =\frac{1}{\lambda_{j}}\mu_{i}^{T}\Sigma\mu_{j}=\frac{1}{\lambda_{j}}(\Sigma\mu_{i})^{T}\mu_{j}=0.
\]
Suppose otherwise that $\lambda_{j}=0.$ Then $\mu_{i},\mu_{j}\in\ker(\Sigma),$
which is a subspace. Therefore, we can orthonormalize $\mu_{i},\mu_{j}$
by applying \ref{lem: gram schimt}.\\
\end{enumerate}
\begin{cBoxA}[Problem 2.19 - Characterization of $\Sigma,\Sigma^{-1}$ in Gaussian
distribution]{}
 Show that a real, symmetric matrix $\Sigma$ having the eigenvector
equation (2.45) can be expressed as an expansion in the eigenvectors,
with coefficients given by the eigenvalues, of the form (2.48). Similarly,
show that the inverse matrix $\Sigma^{-1}$ has a representation of
the form (2.49). 
\end{cBoxA}

Note that $\text{Eq.(2.45)}=\sum_{i=1}^{n}\lambda_{i}u_{i}u_{i}^{*}=U\Lambda U^{*},$
where $U=[u_{i}]$ are vertical stack of eigen vectors of $\Sigma.$
On the other hand, note that $\Sigma U=U\Lambda,$ which implies $\Lambda=U^{*}\Sigma U.$
Therefore, substitute back we get $U\Lambda U^{*}=UU^{*}\Sigma UU^{*}=\Sigma$
as desired. \\

\begin{cBoxA}[Problem 2.20 - Positive definite has positive eigenvalues]{}
 A positive definite matrix $\Sigma$ can be defined as one for which
the quadratic form 
\[
a^{T}\Sigma a
\]
is positive for any real value of the vector a. Show that a necessary
and sufficient condition for $\Sigma$ to be positive definite is
that all of the eigenvalues $\lambda_{i}$ of $\Sigma$, defined by
(2.45), are positive.
\end{cBoxA}

This is an important result. Hence, we will prove a stronger version
by extending the matrix to the $\mathbb{C}.$ For later use, we pack
the result in the following lemma.
\begin{lem}
A matrix $M\in\mathrm{Mat}_{\mathbb{C}}(n,n)$ for some $n\in\mathbb{N}$
is symmetric positive definite\footnote{Although there are matrices that are postivie definite but not symmetric.
It suffices for us to assume that $M$ is symmetric in view of Problem
1.14. } if and only if all of the eigen values of $M_{i}$ are positive. 
\end{lem}
\begin{proof}
$\Leftarrow$ Suppose all of the eigen values of $M$, denoted by
$\{\lambda_{i}\}_{i=1}^{n}$ are positive. Note that in view of Problem
2.18, we have $M=V\Lambda V^{*},$ where $V$ is the matrix containing
eigen vectors and $\Lambda$ is the diagonal matrix of eigen values.
So we have for any $x\in\mathbb{C}^{n}$ that $x^{*}Mx=(x^{*}V)\Lambda(V^{*}x)=\sum_{i=1}^{n}s_{i}^{2}/\lambda_{i},$
where $s_{i}$ is the $i$-th term in the vector $V^{*}x.$ Sicne
$\lambda_{i}>0$ for all $i\in\{1,...,n\},$ it follows that $x^{*}Mx>0$
as desired. \medskip\\
$\Rightarrow$ On the other hand, suppose $M$ is positive definite,
i.e. $x^{*}Mx>0$ for all $x\in\mathbb{C}^{n}$.\footnote{One detail we left out here is to show that $x^{*}Mx$ is real for
any $x\in\mathbb{C}^{n}.$ To see this, we observe that $(x^{*}Mx)^{*}=x^{*}M^{*}x=x^{*}Mx,$
whence is real since its complex conjugate is equal to itself. } Let $(\lambda,v)$ be an eigen-pair of $M$. We show that $\lambda>0$
by case analysis. Suppose $\lambda\leq0,$ then we have $v^{*}Mv=v^{*}\lambda v=\lambda\left|v\right|^{2}\leq0,$
which is a contradiction to the fact that $M$ is positive definite.
As a result, $\lambda>0$. Since $\lambda$ is chosen arbitrarily,
we have shown that any eigen value of $M$ is positive as desired.
\\
\end{proof}
%
\begin{cBoxA}[Problem 2.21 - Independent parameter for symmetric matrix ]{}
Show that a real, symmetric matrix of size $D\times D$ has $D(D+1)/2$
independent parameters.
\end{cBoxA}

Clearly, once we have known the upper triangular part of the matrix
plus the diagonal, we will have known the whole matrix. As a result,
the number of independent parameters for symmetric matrix is $\sum_{i=1}^{D}i=D(D+1)/2.$\\

\begin{cBoxA}[Problem 2.22 - Inverse of symmetric matrix is symmetric]{}
 Show that the inverse of a symmetric matrix is itself symmetric.
\end{cBoxA}

First, we fix some notations. Let $M\in\mathrm{GL}_{\mathbb{R}}(n)$
be a symmetric matrix. It suffices to show that $(M^{-1})^{T}=M^{-1}.$
Since $M^{-1}M=MM^{-1}=I,$ it follows that 
\[
(MM^{-1})^{T}=((M^{-1})^{T}M^{T})=(M^{-1})^{T}M=I.
\]
Hence, it follows that $(M^{-1})^{T}M=M^{-1}M.$ Now multiplying both
sides of the previous expression by $M^{-1},$ we get $(M^{-1})^{T}MM^{-1}=M^{-1}MM^{-t},$
which implies that $(M^{-1})^{T}=M^{-1}.$\\

\begin{cBoxA}[Problem 2.23 - Volume of hyperellipsoid in $n$-dimensional space ]{}
 By diagonalizing the coordinate system using the eigenvector expansion
(2.45), show that the volume contained within the hyperellipsoid corresponding
to a constant Mahalanobis distance $\Delta$ is given by 
\[
V_{D}\left|\Sigma\right|^{1/2}\Delta^{D}
\]
where $V_{D}$ is the volume of the unit sphere in $D$ dimensions,
and the Mahalanobis distance is defined by (2.44). 
\end{cBoxA}

The wording of this problem is a bit confusing. Here, we give a clarification:
recall that the unit sphere in $n$-dimensional Euclidean space is
given by 
\[
V_{D}:=\int_{\mathbb{R}^{n}}\mathbbm{1}(\left\Vert x\right\Vert \leq1)dx=\int_{\mathbb{R}^{n}}\mathbbm{1}(x^{T}x\leq1)dx.
\]
Also recall that the solution to this integral has been worked out
in Problem 1.18. In the textbook, the hyperellipsoid with Mahalanobis
distance $\Delta$ is defined to be the set of the form $\{x\in\mathbb{R}^{n}:(x-\mu)^{T}\Sigma^{-1}(x-\mu)\leq\Delta^{2},\Sigma^{-1}\text{ is positive (semi)definite}\}.$
So to show that the volumne of hyperellipsoid with Mahalanobis distance
$\Delta$ is equal to $V_{D}\det\left|\Sigma\right|^{1/2}\Delta^{D}$
is equivalent to showing the following: 
\[
\int_{\mathbb{R}^{n}}\mathbbm{1}((x-\mu)^{T}\Sigma^{-1}(x-\mu))dx=V_{D}\det\left|\Sigma\right|^{1/2}\Delta^{D}.
\]
We show this using a series of change of variables(c.f. \ref{thm: transformation thm})
: note that 
\begin{align*}
\int_{\mathbb{R}^{n}}\mathbbm{1}((x-\mu)^{T}\Sigma^{-1}(x-\mu)\leq\Delta^{2})dx & =\int_{\mathbb{R}^{n}}\mathbbm{1}(y^{T}\Sigma^{-1}y\leq\Delta^{2})dy\tag{by letting \ensuremath{x=y+\mu}}\\
 & =\int_{\mathbb{R}^{n}}\mathbbm{1}(y^{T}\Sigma^{-1/2}\Sigma^{-1/2}y\leq\Delta^{2})dy\tag{since \ensuremath{\Sigma^{-1}} is p.d.}\\
 & =\int_{\mathbb{R}^{n}}\mathbbm{1}(w^{T}w\leq\Delta^{2})\left|\det\Sigma^{1/2}\right|dw\tag{by letting \ensuremath{y=\Sigma^{1/2}w}}\\
 & =\int_{\mathbb{R}^{n}}\mathbbm{1}(z^{T}z\leq1)\left|\det\Sigma^{1/2}\right|\left|\det\Delta I\right|dz\tag{by letting \ensuremath{w=(\Delta I)z}}\\
 & =\left|\det\Sigma^{1/2}\right|\det\left|\Delta I\right|V_{D}\\
 & =\det\left|\Sigma\right|^{1/2}\Delta^{D}V_{D},
\end{align*}
where the last equality follows since $\det(\Sigma^{1/2}\Sigma^{1/2})=\det(\Sigma^{1/2})\det(\Sigma^{1/2})=\det(\Sigma),$
which implies that $\det(\Sigma^{1/2})=(\det\Sigma)^{1/2}$. \\

\begin{cBoxA}[Problem 2.24 - Block matrix inversion formula]{}
 Prove the identity (2.76) by multiplying both sides by the matrix
\[
\begin{bmatrix}A & B\\
C & D
\end{bmatrix}
\]
and making use of the definition $(2.77).$ 
\end{cBoxA}

To facilitate notation, we use $F$ to denote the matrix defined in
the RHS of Eq.(2.76) and $F$ the LHS. To show the identity claimed
in Eq.(2.76) holds, we need to show that $FE=EF=I$(we assume the
dimensions match in all the partitions and parts of the parition in
$F$ are necessarily invertible). Note that the following lemma shows
that it suffices for us to show either $FE=I$ or $EF=I$.\\

\begin{lem}
Let $A\in\mathrm{Mat}_{\mathbb{C}}(n,m)$ and $B\in\mathrm{Mat}_{\mathbb{C}}(m,n).$
Then if $AB=I$, it follows that $BA=I.$\label{lem: matrix left inverse in implies right inverse}
\end{lem}
\begin{proof}
By assumption we have $AB-I=0.$ We multiply on the left by $B$ to
get $BAB-B=(BA-I)B=0.$ Now we let $\{e_{i}\}_{i=1}^{n}$ be the standard
basis of $\mathbb{R}^{n}.$ We claim that $\{Be_{i}\}_{i=1}^{n}$
is also a standard basis. To see this, suppose $\sum_{i=1}^{n}\alpha_{i}Be_{i}=0.$
Multiplying both sides on the left by $A$ and we get $\sum_{i=1}^{n}\alpha_{i}ABe_{i}=0,$
which reduces to $\sum_{i=1}^{n}\alpha_{i}e_{i}=0$ since $AB=I$
by assumption. Since $\{e_{i}\}_{i=1}^{n}$ is the standard basis,
it follows that $\alpha_{i}=0$ for all $i\in\{1,...,n\}.$ As a result
$\{Be_{i}\}_{i=1}^{n}$ is a set of basis in $\mathbb{R}^{n}$ as
desired. Now we come back to the $(BA-I)B=0.$ Note that from which
we can see that $(BA-I)Be_{i}$ for all $i=1,...,n.$ Since $\{Be_{i}\}_{i=1}^{n}$
is a set of basis, it follows that for any $v\in\mathbb{R}^{n},$
we have 
\[
(BA-I)v=(BA-I)\left(\sum_{i=1}^{n}\alpha_{i}Be_{i}\right)=\sum_{i=1}^{n}\alpha_{i}(BA-I)Be_{i}=0,
\]
whence $BA=I$ as desired. 
\end{proof}
In view of previous lemma, we go ahead and show that $FE=I.$ Writing
the terms out explicitly yields: 
\begin{align*}
FE & =\begin{bmatrix}A & B\\
C & D
\end{bmatrix}\begin{bmatrix}M & -MBD^{-1}\\
-D^{-1}CM & D^{-1}+D^{-1}CMBD^{-1}
\end{bmatrix}\\
 & =\begin{bmatrix}AM-B(D^{-1}CM) & -AMBD^{-1}+BD^{-1}+BD^{-1}CMBD^{-1}\\
CM-DD^{-1}CM & -CMBD^{-1}+I+CMBD^{-1}
\end{bmatrix}\\
 & :=\begin{bmatrix}P_{11} & P_{12}\\
P_{21} & P_{22}
\end{bmatrix}.
\end{align*}
Now we evaluate the blocks $P_{ij}$ term by term: 
\begin{align*}
P_{11} & =A(A-BD^{-1}C)^{-1}-BD^{-1}C(A-BD^{-1}C)^{-1}=(A-BD^{-1}C)(A-BD^{-1}C)^{-1}=I;\\
P_{12} & =(-AM+I+BD^{-1}CM)BD^{-1}=(-\underbrace{(A-BD^{-1}C)M}_{=I}+I)BD^{-1}=0;\\
P_{21} & =CM-CM=0;\\
P_{22} & =I.
\end{align*}
Therefore, the results follows. \\

\begin{cBoxA}[Problem 2.25 - Marginal and conditional expectation of multivariate
gaussian ]{}
 In Sections 2.3.1 and 2.3.2, we considered the conditional and marginal
distributions for a multivariate Gaussian. More generally, we can
consider a partitioning of the components of $X$ into three groups
$X_{a}$, $B_{b}$, and $X_{c}$, with a corresponding partitioning
of the mean vector $\mu$ and of the covariance matrix $\Sigma$ in
the form 
\[
\mu=\begin{bmatrix}\mu_{a}\\
\mu_{b}\\
\mu_{c}
\end{bmatrix},\ \ \Sigma=\begin{bmatrix}\Sigma_{aa} & \Sigma_{ab} & \Sigma_{ac}\\
\Sigma_{ba} & \Sigma_{bb} & \Sigma_{bc}\\
\Sigma_{ca} & \Sigma_{cb} & \Sigma_{cc}
\end{bmatrix}
\]
By making use of the results of Section 2.3, find an expression for
the conditional distribution $\P(X_{a}\vert X_{b})$ in which $X_{c}$
has been marginalized out.
\begin{rem}
\textit{Although this textbook has derived the result for marginal
and conditional distribution for multivariate gaussian distributions
using completing squares. It is not done in the most rigiorous manner
and left out many of the calculations. Hence, we rederive the results
here. However, we will be using a slightly different approach in the
derivation, which is more rigorous and slightly more general. Details
of the derivation proposed by the book will also be discussed in the
reading notes. }
\end{rem}
\end{cBoxA}

To begin with, we introduce a few auxiliary lemmas to help proving
later results.
\begin{lem}
Suppose $A\in\mathrm{Mat}_{\mathbb{R}}(n,n)$ is positive definite(symmetric)
and partitioned as $A=\begin{bmatrix}A_{11} & A_{12}\\
A_{21} & A_{22}
\end{bmatrix},$ then $A_{11}$ and $A_{22}$ are positive definite as well. 
\end{lem}
\begin{proof}
It suffices to show that $A_{11}$ and $A_{22}$ are p.d since p.d.
matrices are invertible. Without loss of generality, we assume $A_{11}$
is of dimension $n_{1}\times n_{1}$ and $A_{22}$ of $n_{2}\times n_{2}$
such that $n_{1}+n_{2}=n.$ Note that since $A$ is p.d., $x^{T}Ax>0$
for all $x\in\mathbb{R}^{n}.$ Then for $x=\begin{bmatrix}x_{1}\\
0
\end{bmatrix}$ in block form, where $x_{1}\in\mathbb{R}^{n_{1}}$ is arbitrarily
chosen, it follows that $x^{T}Ax=x_{11}^{T}A_{11}x_{11}>0$. Therefore,
$A_{11}$ is p.d. as well. On the other hand, if we let $x=\begin{bmatrix}0\\
x_{2}
\end{bmatrix}$ for some arbitrarily chosen $x_{2}\in\mathbb{R}^{n_{2}},$ it also
follows that $x^{T}Ax=x_{2}^{T}A_{22}x_{2}>0.$ Since $x_{2}$ is
arbitrarily, $A_{22}$ is p.d. as well. 
\end{proof}
%
\begin{lem}
Let $A\in\mathrm{Mat}_{\mathbb{\mathbb{C}}}(n,n)$, $D\in\mathrm{Mat}_{\mathbb{C}}(m,m),B\in\mathrm{Mat}_{\mathbb{C}}(n,m)$
for arbitrary $n,m\in\mathbb{N}$. Then it follows that \label{lem: block determinant}
\[
\det\begin{bmatrix}A & B\\
0 & D
\end{bmatrix}=\det(A)\det(D).
\]
\end{lem}
\begin{proof}
We layout our proof in several steps as below. 
\begin{enumerate}[leftmargin={*}]
\item We first prove the basic case where $D=I.$ We claim that $\det\begin{bmatrix}A & B\\
0 & I
\end{bmatrix}=\det A.$ To prove this claim, we induct on $I'$s size, denoted as $m$. For
the base case where $m=1$, we have by Laplace expansion on the last
row that 
\[
\det\begin{bmatrix}A & B\\
0 & 1
\end{bmatrix}=(-1)^{(n+1)+(n+1)}\det(A)=\det A.
\]
Now suppose $m=k$ holds. Then we have that 
\[
\det\begin{bmatrix}A & B\\
0 & I_{k+1}
\end{bmatrix}=\det\begin{bmatrix}A & B_{1} & B_{2}\\
0 & I_{k} & 0\\
0 & 0 & 1
\end{bmatrix}=(-1)^{(n+m+1)+(n+m+1)}\det\begin{bmatrix}A & B\\
0 & I_{k}
\end{bmatrix}=\det(A),
\]
where $B_{1}\in\mathrm{Mat}_{\mathbb{C}}(n,m)$ and that $B_{2}\in\mathrm{Mat}_{\mathbb{C}}(n,1)$
and the last equality follows from inductive hypoethesis. We also
note that using the same argument we can also show that 
\[
\det\begin{bmatrix}I & B\\
0 & D
\end{bmatrix}=\det(D).
\]
\item Now we go back to the proof of the lemma. We first deal the case where
$A$ is inverstible, i.e. $\det A\neq0.$ Recall that $\det(MN)=\det(M)\det(N)$
for abitrary compatible matrices $M$ and $N$. Therefore, it follows
that 
\[
\det\begin{bmatrix}A & B\\
0 & D
\end{bmatrix}=\det\left(\begin{bmatrix}A & 0\\
0 & I_{m}
\end{bmatrix}\begin{bmatrix}I_{n} & A^{-1}B\\
0 & D
\end{bmatrix}\right)=\det\begin{bmatrix}A & 0\\
0 & I_{m}
\end{bmatrix}\det\begin{bmatrix}I_{n} & A^{-1}B\\
0 & D
\end{bmatrix}=\det(A)\det(D),
\]
where the last equality follows from step-1. 
\item It remains to deal with the case where $A$ is not invertible. Note
that if $A$ is not invertible, then the columns of $A$ are not linearly
independent. From this we see that the first $n$ columns of $\begin{bmatrix}A & B\\
0 & D
\end{bmatrix}$ are not linearly independent as well since we are stacking $0$'s
under $A$ and as a result the spanning set of first $n$ columns
is thus homeomorphic to that of $A.$ Therefore, $\begin{bmatrix}A & B\\
0 & D
\end{bmatrix}$ is not invertible. 
\end{enumerate}
\end{proof}
%
The following result is a classical theorem, whose proof can be easily
found in any measure theoretic probability books. We state without
proof.
\begin{thm}[Uniqueness of fourier transform]
The Fourier transform of a probability measure on $\mathbb{R}^{n}$
characterizes $\mu,$ that is if two probabilty measures on $\mathbb{R}^{n}$
admit the same Fourier transform, they are equal. \label{thm: uniqueness of cf}
\end{thm}
%
In later results, we will construct some independent variables from
family of distributions decided by random variables that are not necessarily
independent. The proof of this theorem is quite complicated. It's
closely related to the Kolmogrov extension theorem. 
\begin{thm}[Creation of new, independent random variables]
Let $(X_{\alpha})_{\alpha\in A}$ be a family of random variabels(not
necessarily independent or finite), and let $(\mu_{\beta})_{\beta\in B}$
be a collection (not necessarily finite) of probability measures on
measurable spaces $(R_{\beta})_{\beta\in B}.$ Then after extending
the sample spaces if necessary, one can find a family $(Y_{\beta})_{\beta\in B}$
of independent random variables such that each $Y_{\beta}$ has distribution
$\mu_{\beta}$, and the two families $(X_{\alpha})_{\alpha\in A}$
and $(Y_{\beta})_{\beta\in B}$ are independent of each other.\label{thm: construct independent r.v. } 
\end{thm}
One direct application of \ref{thm: uniqueness of cf} is the following
lemma. 
\begin{lem}
Let $X$ be an $\mathbb{R}^{n}$ valued random variable that has partition
of the form $\begin{bmatrix}X_{1};X_{2}\end{bmatrix}$, where $X_{1}\in\mathbb{R}^{k}$
and $X_{2}\in\mathbb{R}^{n-k}.$ Then $X_{1}\perp X_{2}$ iff $\varphi_{X}(t)=\varphi_{X_{1}}(t_{1})\varphi_{X_{2}}(t_{2}).$\label{lem: cf independence}
\end{lem}
\begin{proof}
$\Rightarrow$ Suppose $X_{1}\perp X_{2}.$ Then by a well known result
(cf. \cite[Exercise 4.15]{resnickProbabilityPath2014}), we have that
\[
\varphi_{X}(t)=\E[e^{i\left\langle t,X\right\rangle }]=\E[e^{i\left\langle t_{1},X_{1}\right\rangle }e^{i\left\langle t_{2},X_{2}\right\rangle }]=\E[e^{i\left\langle t_{1},X_{1}\right\rangle }]\E[e^{i\left\langle t_{2},X_{2}\right\rangle }]=\varphi_{X_{1}}(t_{1})\varphi_{X_{2}}(t_{2}).
\]

\noindent $\Leftarrow$ We first construct $\widetilde{X}_{1}$ and
$\widetilde{X}_{2}$ such that $X_{1}=_{d}\widetilde{X}_{1}$ and
$X_{2}=_{d}\widetilde{X}_{1}$, as well as $\widetilde{X}_{1}\perp\widetilde{X}_{2}$
(the existence of $\widetilde{X}_{1}$ and $\widetilde{X}_{2}$ is
guaranteed by \ref{thm: construct independent r.v. }). Then we have
\begin{align*}
\varphi_{(X_{1},X_{2})}((t_{1},t_{2})) & =\varphi_{X_{1}}(t_{1})\varphi_{X_{2}}(t_{2})\\
 & =\varphi_{\tilde{X}_{1}}(t_{1})\varphi_{\tilde{X}_{2}}(t_{2})\tag{by definition of characteristic functions}\\
 & =\varphi_{(\widetilde{X}_{1},\widetilde{X}_{2})}((t_{1},t_{2})).
\end{align*}
Therefore, $\P_{(X_{1},X_{2})}=\P_{(\widetilde{X}_{1},\widetilde{X}_{2})}.$
Hence, as a result, for any $A\in\mathcal{B}(\mathbb{R}^{k}),B\in\mathcal{B}(\mathbb{R}^{n-k}),$
we have 
\begin{align*}
\P_{(X_{1},X_{2})}(X_{1}\in A,X_{2}\in B) & =\P_{(\widetilde{X}_{1},\widetilde{X}_{2})}(\widetilde{X}_{1}\in A,\widetilde{X}_{2}\in B)=\P(\tilde{X}_{1}\in A)\P(\tilde{X}_{2}\in B)\\
 & =\P(X_{1}\in A)\P(X_{2}\in B).
\end{align*}
Hence, $X$ and $Y$ are independent. 
\end{proof}
Now, we state and prove some useful result of multivariate gaussian
distribution for later use. 
\begin{lem}
Let $X$ be a $\mathbb{R}^{n}$ valued random variable such that $X\sim\mathrm{MVN}(\mu,\Sigma)$
and $X$ is paritioned as (for some fixed $k$)
\[
X=\begin{bmatrix}X_{1}\\
X_{2}
\end{bmatrix},\ \ \mu=\begin{bmatrix}\mu_{1}\\
\mu_{2}
\end{bmatrix}\text{ and }\Sigma=\begin{bmatrix}\Sigma_{11} & \Sigma_{12}\\
\Sigma_{21} & \Sigma_{22}
\end{bmatrix},
\]
where $X_{1},\mu_{1}\in\mathbb{R}^{k}$ and $X_{2},\mu_{2}\in\mathbb{R}^{n-k}$,
for $k=1,...,n-1$ ($\Sigma_{11},\Sigma_{12},\Sigma_{21},\Sigma_{22}$
are of dimension $k\times k,k\times(n-k),(n-k)\times k,(n-k)\times(n-k)$).
Then the follow holds:\label{lem: property of gaussian}
\begin{enumerate}
\item $X_{1}\sim\mathrm{MVN}(\mu_{1},\Sigma_{11})$ and $X_{2}\sim\mathrm{MVN}(\mu_{2},\Sigma_{22});$
\item $X_{1}\perp X_{2}$ iff $\Sigma_{12}=0;$ 
\item the conditional distribution of $X_{1}$ given that $X_{2}=x_{2}$
is $\mathrm{MVN}(\mu_{1\cdot2},\Sigma_{11\cdot2}),$ where $\mu_{1\cdot2}=\mu_{1}+\Sigma_{12}\Sigma_{22}^{-1}(x_{2}-\mu_{2})$,
and $\Sigma_{11\cdot2}=\Sigma_{11}-\Sigma_{12}\Sigma_{22}^{-1}\Sigma_{21}.$ 
\end{enumerate}
\end{lem}
%
\begin{proof}
\begin{enumerate}[leftmargin={*}]
\item To prove this, we take the characteristic function approach. We fist
talk about the general case. If we don't make any distributional assumption
about $X$ and only assume that it has some density function which
we denote as $f_{X}(x).$ Then we have that 
\begin{align*}
\varphi_{X_{1}}(t) & =\int_{\mathbb{R}}\cdots\int_{\mathbb{R}}\exp\bigg(i\bigg(\sum_{i=1}^{n_{1}}t_{i}x_{i}\bigg)\bigg)f(x_{1},...,x_{n_{1}})dx_{1}dx_{2}\cdots dx_{n_{1}}\\
 & =\int_{\mathbb{R}}\cdots\int_{\mathbb{R}}\exp\bigg(i\bigg(\sum_{i=1}^{n_{1}}t_{i}x_{i}\bigg)\left(\int_{\mathbb{R}}\cdots\int_{\mathbb{R}}f(x_{1},\ldots,x_{n})dx_{n_{1}+1}\cdots dx_{n}\right)dx_{1}\cdots dx_{n}\\
 & =\int_{\mathbb{R}}\cdots\int_{\mathbb{R}}\exp\bigg(i\bigg(\sum_{i=1}^{n_{1}}t_{i}x_{i}+\sum_{i=n_{1+1}}^{n}0x_{i}\bigg)\bigg)f(x_{1},\ldots,x_{n})dx_{1}\cdots dx_{n}\\
 & =\varphi_{X}(t,0),
\end{align*}
where $t\in\mathbb{R}^{n_{1}}.$ Hence, by applying this fact to the
gaussian case along with \ref{lem: cf of gaussian} we see that 
\begin{align*}
\varphi_{X_{1}}(t) & =\varphi_{X}\bigg(\begin{bmatrix}t\\
0
\end{bmatrix}\bigg)\\
 & =\exp\bigg((i\left\langle \begin{bmatrix}t\\
0
\end{bmatrix},\begin{bmatrix}\mu_{1}\\
\mu_{2}
\end{bmatrix}\right\rangle \bigg)\exp\bigg(-\frac{1}{2}\begin{bmatrix}t\\
0
\end{bmatrix}^{T}\begin{bmatrix}\Sigma_{11} & \Sigma_{12}\\
\Sigma_{21} & \Sigma_{22}
\end{bmatrix}\begin{bmatrix}t\\
0
\end{bmatrix}\bigg)\\
 & =\exp(i\left\langle t,x\right\rangle )\exp\left(-\frac{1}{2}t^{T}\Sigma_{11}t\right).
\end{align*}
Hence, $X_{1}\sim\mathrm{MVN}(\mu_{1},\Sigma_{11}).$ That $X_{2}\sim\mathrm{MVN}(\mu_{2},\Sigma_{22})$
follows from a similar argument. 
\item To prove this, note that 
\[
\Sigma_{12}=0\iff t^{T}\Sigma t\iff t_{1}^{T}\Sigma_{11}t_{1}+t_{2}^{T}\Sigma_{22}t_{2},
\]
for any $t\in\mathbb{R}^{n}$ and $t_{1}\in\mathbb{R}^{k},t_{2}\in\mathbb{R}^{n-k}.$
Then it follows that 
\begin{align*}
\varphi_{X}(t) & =\exp(i\left\langle t,x\right\rangle )\exp\left(-\frac{1}{2}t^{T}\Sigma_{11}t\right)\\
 & =\exp(i\left\langle t_{1},x_{1}\right\rangle )\exp\left(-\frac{1}{2}t_{1}^{T}\Sigma_{11}t_{1}\right)\exp(i\left\langle t_{2},x_{2}\right\rangle )\exp\left(-\frac{1}{2}t_{2}^{T}\Sigma_{22}t_{2}\right)\\
 & =\varphi_{X_{1}}(t_{1})\varphi_{X_{2}}(t_{2}).
\end{align*}
Then the result follows by an application of \ref{lem: cf independence}
\item First, we consider a linear transformation of $X$ in the form as
\[
CX=\begin{bmatrix}I_{k} & -B\\
0 & I_{n-k}
\end{bmatrix}\begin{bmatrix}X_{1}\\
X_{2}
\end{bmatrix}=\begin{bmatrix}X_{1}-BX_{2}\\
X_{2}
\end{bmatrix}:=\begin{bmatrix}Y_{1}\\
Y_{2}
\end{bmatrix}:=Y
\]
Then by \ref{lem: distribution of linear transformation of gaussian},
it follows that $Y\sim\mathrm{MVN}(\mu_{Y},\Sigma_{Y}),$ where 
\begin{align*}
\mu_{Y} & =\begin{bmatrix}I_{k} & -B\\
0 & I_{n-k}
\end{bmatrix}\begin{bmatrix}\mu_{1}\\
\mu_{2}
\end{bmatrix}=\begin{bmatrix}\mu_{1}-B\mu_{2}\\
\mu_{2}
\end{bmatrix};\\
\Sigma_{Y} & =\begin{bmatrix}I_{k} & -B\\
0 & I_{n-k}
\end{bmatrix}\begin{bmatrix}\Sigma_{11} & \Sigma_{12}\\
\Sigma_{21} & \Sigma_{22}
\end{bmatrix}\begin{bmatrix}I_{k} & -B^{T}\\
0 & I_{n-k}
\end{bmatrix}=\begin{bmatrix}\Sigma_{11}-B\Sigma_{12}+\Sigma_{12}B^{T}-B\Sigma_{22}B^{T} & \Sigma_{12}-B\Sigma_{22}\\
\Sigma_{12}-B^{T}\Sigma_{22} & \Sigma_{22}
\end{bmatrix}.
\end{align*}
 In view of part-2, if we let $B=\Sigma_{12}\Sigma_{22}^{-1}$, we
have that 
\begin{align*}
Y & \sim\mathrm{MVN}\left(\begin{bmatrix}\mu_{1}-\Sigma_{12}\Sigma_{22}^{-1}\mu_{2}\\
\mu_{2}
\end{bmatrix},\begin{bmatrix}\Sigma_{11}-\Sigma_{12}\Sigma_{22}^{-1}\Sigma_{21} & 0\\
0 & \Sigma_{22}
\end{bmatrix}\right)=\mathrm{MVN}\left(\widetilde{\mu}=\begin{bmatrix}\mu_{1\cdot2}\\
\mu_{2}
\end{bmatrix},\widetilde{\Sigma}=\begin{bmatrix}\Sigma_{11\cdot2} & 0\\
0 & \Sigma_{22}
\end{bmatrix}\right)
\end{align*}
Then in view of part 1 and 2, it follows that $Y_{1}\sim\mathrm{MVN}(\mu_{11\cdot2},\Sigma_{11\cdot2})\text{ and }Y_{2}\sim\mathrm{MVN}(\mu_{2},\Sigma_{22}),$
and moreover $Y_{1}\perp Y_{2}.$ Therefore, $Y$ has the density
function in the following form 
\begin{align*}
f_{Y}(y) & =f_{(Y_{1},Y_{2})}((y_{1},y_{2}))\\
 & =\frac{1}{(2\pi)^{n/2}(\det\widetilde{\Sigma})^{1/2}}\exp\left(-\frac{1}{2}(y-\mu)^{T}\Sigma^{-1}(y-\mu)\right)\\
 & =\frac{1}{(2\pi)^{k/2}(\det\Sigma_{11\cdot2})}\exp\left(-\frac{1}{2}(y_{1}-\mu_{1\cdot2})^{T}\Sigma_{11\cdot2}^{-1}(y_{1}-\mu_{1\cdot2})\right)\tag{by \ref{lem: block determinant} }\\
 & \ \ \times\frac{1}{(2\pi)^{(n-k)/2}\det(\Sigma_{22})}\exp\left(-\frac{1}{2}(y_{2}-\mu_{2})^{T}\Sigma_{22}^{-1}(y_{2}-\mu_{2})\right)\\
 & =f_{Y_{1}}(y_{1})f_{Y_{2}}(y_{2}).
\end{align*}
Now note that since 
\[
\begin{bmatrix}I_{k} & \Sigma_{12}\Sigma_{22}\\
0 & I_{n-k}
\end{bmatrix}Y=\begin{bmatrix}X_{1}\\
X_{2}
\end{bmatrix},\text{ and }\begin{bmatrix}I_{k} & \Sigma_{12}\Sigma_{22}\\
0 & I_{n-k}
\end{bmatrix}^{-1}=\begin{bmatrix}I_{k} & -\Sigma_{12}\Sigma_{22}^{-1}\\
0 & I_{n-k}
\end{bmatrix}:=M,
\]
it follows from \ref{thm: transformation thm} 
\begin{align*}
f_{(X_{1},X_{2})}(x_{1},x_{2}) & =f_{(Y_{1},Y_{2})}\left(\begin{bmatrix}I_{k} & -\Sigma_{12}\Sigma_{22}^{-1}\\
0 & I_{n-k}
\end{bmatrix}\begin{bmatrix}x_{1}\\
x_{2}
\end{bmatrix}\right)=f_{(Y_{1},Y_{2})}(x_{1}-\Sigma_{12}\Sigma_{22}^{-1}x_{2},x_{2})\\
 & =f_{Y_{1}}(x_{1}-\Sigma_{12}\Sigma_{22}^{-1}x_{2})f_{X_{2}}(x_{2}).
\end{align*}
On the other hand, since $f_{(X_{1},X_{2})}(x_{1},x_{2})=f_{X_{1}\vert X_{2}}(x_{1}\vert x_{2})f_{X_{2}}(x_{2}),$
the follows that $f_{Y_{1}}(x_{1}-\Sigma_{12}\Sigma_{22}^{-1}x_{2})=f_{X_{1}\vert X_{2}}(x_{1}\vert x_{2}).$
Since 
\[
x_{1}-\Sigma_{12}\Sigma_{22}^{-1}x_{2}-\mu_{1}+\Sigma_{12}\Sigma_{22}\mu_{2}=x_{1}-(\mu_{1}+\Sigma_{12}\Sigma_{22}^{-1}(x_{2}-\mu_{2})=\mu_{1\cdot2},
\]
we further expand the expression and get 
\begin{align*}
f_{Y_{1}}(x_{1}-\Sigma_{12}\Sigma_{22}^{-1}x_{2}) & =\frac{1}{(2\pi)^{k/2}\left|\det\Sigma_{11\cdot2}\right|^{1/2}}\exp\bigg(-\frac{1}{2}(x_{1}-\mu_{1\cdot2})^{T}\Sigma_{11\cdot2}^{-1}(x_{1}-\mu_{1\cdot2})\bigg)\\
 & =f_{\mathrm{MVN}(\mu_{1\cdot2},\Sigma_{11\cdot2})}(x_{1}).
\end{align*}
Hence, it follows that $X_{1}\vert X_{2}=x_{2}\sim\mathrm{MVN}(\mu_{1\cdot2},\Sigma_{11\cdot2})$
as desired. 
\end{enumerate}
\end{proof}
Now we go back to the solution to the problem. Since $(X_{a},X_{b},X_{c})$
and $((X_{a},X_{b}),X_{c})$ are homeomorphic, it follows from \ref{lem: property of gaussian}
that 
\[
\begin{bmatrix}X_{a}\\
X_{b}
\end{bmatrix}\sim\mathrm{MVN}\left(\mu_{a\cdot b},\Sigma_{a\cdot b}\right),\text{where }\mu_{a\cdot b}=\begin{bmatrix}\mu_{a}\\
\mu_{b}
\end{bmatrix},\Sigma_{a\cdot b}=\begin{bmatrix}\Sigma_{aa} & \Sigma_{ab}\\
\Sigma_{ba} & \Sigma_{bb}
\end{bmatrix}.
\]
Another application of \ref{lem: property of gaussian}-(3), yields
that 
\[
f_{X_{a}\vert X_{b}}(x_{a}\vert x_{b})=\frac{1}{(2\pi)^{(\dim X_{a}+\dim X_{b})/2}(\det\Sigma_{aa\cdot2})}\exp\left(-\frac{1}{2}(x-\mu_{a\cdot2})^{T}\Sigma_{aa\cdot2}^{-1}(x-\mu_{a\cdot2})\right),
\]
where $\mu_{a\cdot2}=\mu_{a}+\Sigma_{ab}\Sigma_{bb}^{-1}(x_{b}-\mu_{b}),\text{ and }\Sigma_{aa\cdot2}=\Sigma_{aa}-\Sigma_{ab}\Sigma_{bb}^{-1}\Sigma_{ba}.$\\

\begin{cBoxA}[Problem 2.26 - Woodbury matrix inversion formula]{}
 A very useful result from linear algebra is the Woodbury matrix
inversion formula given by 
\[
(A+BCD)^{-1}=A^{-1}-A^{-1}B(C^{-1}+DA^{-1}B)^{-1}DA^{-1}.
\]
By multiplying both sides by $(A+BCD)$ prove the correctness of this
result.
\end{cBoxA}

In view of \ref{lem: matrix left inverse in implies right inverse},
it suffices to show one of the left and right inverse. We show the
left inverse here. Note

\begin{align*}
 & (A^{-1}-A^{-1}B(C^{-1}+DA^{-1}B)^{-1}DA^{-1})(A+BCD)\\
=\ \  & I+A^{-1}B(C^{-1}+DA^{-1}B)^{-1}D-A^{-1}BCD-A^{-1}B(C^{-1}+DA^{-1}B)^{-1}DA^{-1}BCD\\
=\ \  & I+A^{-1}BCD-A^{-1}B(C^{-1}+DA^{-1}B)(DA^{-1}BC+I)D\\
=\ \  & I+A^{-1}BCD-A^{-1}B(C^{-1}+DA^{-1}B)(DA^{-1}B+C^{-1})CD\\
=\ \  & I+A^{-1}BCD-A^{-1}BCD\\
=\ \  & I.
\end{align*}
\\

\begin{cBoxA}[Problem 2.27 - Linearity of expectation and covariance (multivariate
case)]{}
 Let $X$ and $Z$ be two independent random vectors, so that $f(x,z)=f(x)f(z)$.
Show that the mean of their sum $Y=X+Z$ is given by the sum of the
means of each of the variable separately. Similarly, show that the
covariance matrix of $Y$ is given by the sum of the covariance matrices
of $X$ and $Z$. Confirm that this result agrees with that of Exercise
1.10.
\end{cBoxA}

\begin{enumerate}[leftmargin={*}]
\item For expectation, we note that 
\begin{align*}
\E[X+Y] & =\int\int(x+y)f_{(X,Y)}(x,y)dxdy\\
 & =\int\int(x+y)f_{X}(x)f_{Y}(y)dxdy\\
 & =\int_{\mathrm{supp}(X)}xf(x)\left(\int_{\mathrm{supp}(Y)}f(y)dy\right)dx+\int_{\mathrm{supp}(Y)}yf(y)\left(\int_{\mathrm{supp}(X)}f(x)dx\right)dy\\
 & =\E[X]+\E[Y].
\end{align*}
\item For covariance, note that 
\begin{align*}
\mathrm{Cov}[X+Y] & =\E[(X+Y-\E[X+Y])(X+Y-\E[X+Y])^{T}]\\
 & =\E[(X-\E[X]+Y-\E[Y])(X-\E[X]+Y-\E[Y])^{T}]\\
 & =\mathrm{Cov}[X]+\mathrm{Cov}[Y]+\E[(X-\E[X])(Y-\E[Y])^{T}]+\E[(Y-\E[Y])(X-\E[X])^{T}]\\
 & =\mathrm{Cov}[X]+\mathrm{Cov}[Y]+\E[(X-\E[X])\E[(Y-\E[Y])^{T}]+\E[(Y-\E[Y]]\E[X-\E[X]]^{T}\tag{since \ensuremath{X\perp Y}}\\
 & =\mathrm{Cov}[X]+\mathrm{Cov}[Y].
\end{align*}
\\
\end{enumerate}
\begin{cBoxA}[Problem 2.28 - Conditional distribution from joint gaussian]{}
 Consider a joint distribution over the variable 
\[
z=\begin{bmatrix}x\\
y
\end{bmatrix}
\]
whose mean and covariance are given by (2.108) and (2.105) respectively.
By making use of the results (2.92) and (2.93) show that the marginal
distribution $f(x)$ is given (2.99). Similarly, by making use of
the results (2.81) and (2.82) show that the conditional distribution
$f(y\vert x)$ is given by (2.100).
\end{cBoxA}

\textit{This problem can be solved using the hint provided by the
book which adopts the technique of completing squares. However, here
we will solve it using the theory we have developed in a few previous
exercises. But first, we would like to rephrase this problem to make
things clearer: given a joint normal variable $Z$ such that 
\[
Z=\begin{bmatrix}X\\
Y
\end{bmatrix}\sim\mathrm{MVN}\left(\mu_{Z}:=\begin{bmatrix}\mu\\
A\mu+b
\end{bmatrix},\Sigma_{Z}:=\begin{bmatrix}\Lambda^{-1} & \Lambda^{-1}A^{T}\\
A\Lambda^{-1} & L^{-1}+A\Lambda^{-1}A^{T}
\end{bmatrix}\right),
\]
find the conditional distribution of $Y\vert X,$ which we denote
as $f_{Y\vert X}(y\vert x).$\medskip}\\
Without loss generality, assume that $X\in\mathbb{R}^{n}$ and $Y\in\mathbb{R}^{m}$.
Since $\begin{bmatrix}0 & I_{m}\\
I_{n} & 0
\end{bmatrix}\begin{bmatrix}X\\
Y
\end{bmatrix}=\begin{bmatrix}Y\\
X
\end{bmatrix},$ it follows from \ref{lem: distribution of linear transformation of gaussian}
that 
\begin{align*}
\begin{bmatrix}Y\\
X
\end{bmatrix} & \sim\mathrm{MVN}\left(\widehat{\mu}=\begin{bmatrix}0 & I_{m}\\
I_{n} & 0
\end{bmatrix}\begin{bmatrix}\mu\\
A\mu+b
\end{bmatrix},\widetilde{\Sigma}=\begin{bmatrix}0 & I_{m}\\
I_{n} & 0
\end{bmatrix}\begin{bmatrix}\Lambda^{-1} & \Lambda^{-1}A^{T}\\
A\Lambda^{-1} & L^{-1}+A\Lambda^{-1}A^{T}
\end{bmatrix}\begin{bmatrix}0 & I_{m}\\
I_{n} & 0
\end{bmatrix}^{T}\right)\\
 & =\mathrm{MVN}\left(\widetilde{\mu}=\begin{bmatrix}A\mu+b\\
\mu
\end{bmatrix},\widetilde{\Sigma}=\begin{bmatrix}A\Lambda^{-1} & L^{-1}+A\Lambda^{-1}A^{T}\\
\Lambda^{-1} & \Lambda^{-1}A^{T}
\end{bmatrix}\begin{bmatrix}0 & I_{n}\\
I_{m} & 0
\end{bmatrix}\right)\\
 & =\mathrm{MVN}\left(\widetilde{\mu}=\begin{bmatrix}A\mu+b\\
\mu
\end{bmatrix},\widetilde{\Sigma}=\begin{bmatrix}L^{-1}+A\Lambda^{-1}A^{T} & A\Lambda^{-1}\\
\Lambda^{-1}A^{T} & \Lambda^{-1}
\end{bmatrix}\right).
\end{align*}
Then we can apply \ref{lem: property of gaussian}-(3) and get 
\begin{align*}
Y\vert X=x & \sim\mathrm{MVN}(\mu_{Y\vert X=x}=A\mu_{1}+b+(A\Lambda^{-1})\Lambda(x-\mu),\Sigma_{Y\vert X}=L^{-1}+A\Lambda^{-1}A^{T}-A\Lambda^{-1}\Lambda\Lambda^{-1}A^{T})\\
 & =\mathrm{MVN}(\mu_{Y\vert X=x}=Ax+b,\Sigma_{Y\vert X}=L^{-1}.
\end{align*}
as desired. \\

\begin{cBoxA}[Problem 2.29 - Verify Eq.(2.105) ]{}
 Using the partitioned matrix inversion formula (2.76), show that
the inverse of the precision matrix (2.104) is given by the covariance
matrix (2.105).
\end{cBoxA}

Recall from Problem 2.24, for an arbitrary block matrix of the form
$\begin{bmatrix}A & B\\
C & D
\end{bmatrix}$ has inverse of form 
\[
\begin{bmatrix}A & B\\
C & D
\end{bmatrix}^{-1}=\begin{bmatrix}M & -MBD^{-1}\\
-D^{-1}CM & D^{-1}+D^{-1}CMBD^{-1}
\end{bmatrix},
\]
where $M=(A-BDC^{-1})^{-1}.$ Therefore, we have 
\[
\begin{bmatrix}\Lambda+A^{T}LA & -A^{T}L\\
-LA & L
\end{bmatrix}^{-1}=\begin{bmatrix}\Sigma_{11} & \Sigma_{12}\\
\Sigma_{21} & \Sigma_{22}
\end{bmatrix},
\]
where 
\begin{align*}
\Sigma_{11} & =(\Lambda+A^{T}LA-A^{T}LL^{-1}LA)=\Lambda;\\
\Sigma_{12} & =-\Lambda^{-1}A^{T}LL^{-1}=-\Lambda^{-1}A^{T};\\
\Sigma_{21} & =-L^{-1}LA\Lambda=A\Lambda;\\
\Sigma_{22} & =L^{-1}+L^{-1}(-LA)\Lambda(-A^{T}L)L^{-1}=L^{-1}+A\Lambda A^{T}.
\end{align*}
\\

\begin{cBoxA}[Problem 2.30 - Verify Eq.(2.108) ]{}
 By starting from (2.107) and making use of the result (2.105), verify
the result (2.108).
\end{cBoxA}

Note that we have 
\begin{align*}
\E[Z] & =R^{-1}\begin{bmatrix}\Lambda\mu-A^{T}Lb\\
Lb
\end{bmatrix}=\begin{bmatrix}\Lambda^{-1} & \Lambda^{-1}A^{T}\\
A\Lambda^{-1} & L^{-1}+A\Lambda^{-1}A
\end{bmatrix}\begin{bmatrix}\Lambda\mu-A^{T}Lb\\
Lb
\end{bmatrix}\\
 & =\begin{bmatrix}\Lambda^{-1}\Lambda\mu-\Lambda^{-1}A^{T}Lb+\Lambda^{-1}A^{T}Lb\\
A\Lambda^{-1}\Lambda\mu-A\Lambda^{-1}A^{T}Lb+L^{-1}Lb+A\Lambda^{-1}ALb
\end{bmatrix}\\
 & =\begin{bmatrix}\mu\\
A\mu+b
\end{bmatrix}
\end{align*}
as desired. \\

\begin{cBoxA}[Problem 2.31 - Sum of multivariate gaussian]{}
 Consider two multidimensional random vectors x and z having Gaussian
distributions $f(x)=\mathrm{N}(x\vert\mu_{x},\Sigma_{x})$ and $f(z)=\mathrm{N}(z\vert\mu_{z},\Sigma_{z})$
respectively, together with their sum $y=x+z$. Use the results (2.109)
and (2.110) to find an expression for the marginal distribution $f(y)$
by considering the linear-Gaussian model comprising the product of
the marginal distribution $f(x)$ and the conditional distribution
$f(y\vert x)$.
\end{cBoxA}

First, we give a lemma. 
\begin{lem}
Let $X_{1},X_{2}$ be two independent $\mathbb{R}^{n}$-valued random
variables. The characteristic function of $X_{1}+X_{2}$ can be represented
as \label{lem: cf of two independent r.v.}
\[
\varphi_{X_{1}+X_{2}}(t)=\varphi_{X_{1}}(t)\varphi_{X_{2}}(t).
\]
\end{lem}
\begin{proof}
We follow the definition and write 
\[
\varphi_{X_{1}+X_{2}}(t)=\E[e^{i\left\langle t,X_{1}+X_{2}\right\rangle }]=\E[e^{i\left\langle t,X_{1}\right\rangle }e^{i\left\langle t,X_{2}\right\rangle }]=\E[e^{i\left\langle t,X_{1}\right\rangle }]\E[e^{i\left\langle t,X_{2}\right\rangle }]=\varphi_{X_{1}}(t)\varphi_{X_{2}}(t)
\]
as desired. 
\end{proof}
The solution to this problem is an direct application of this lemma.
Since this is an important property, we pack it in a lemma below.
In addition to having only two r.v., we also generalize it to an arbitrary
number of r.v.s. 
\begin{lem}
Let $\{X_{i}\}_{i=1}^{m}$ be $\mathbb{R}^{n}$-valued independent
random variables with $X_{i}\sim\mathrm{MVN}(\mu_{i},\Sigma_{i})$.
Then it follows that $\sum_{i=1}^{m}X_{i}\sim\mathrm{MVN}(\sum_{i=1}^{m}\mu_{i},\sum_{i=1}^{m}\Sigma_{i})$.
\label{lem: sum of independent MVN}
\end{lem}
\begin{proof}
To prove the claim, we induct on $m.$ For the base case $m=2,$ we
note that since $X_{1},X_{2}$ are independent, it follows from \ref{lem: cf of two independent r.v.}
that 
\begin{align*}
\varphi_{X_{1}+X_{2}}(t) & =\varphi_{X_{1}}(t)\varphi_{X_{2}}(t)\\
 & =\exp(i\left\langle t,\mu_{1}\right\rangle )\exp\left(-\frac{1}{2}\left\langle \Sigma_{1}t,t\right\rangle \right)\exp(i\left\langle t,\mu_{2}\right\rangle )\exp\left(-\frac{1}{2}\left\langle \Sigma_{2}t,t\right\rangle \right)\\
 & =\exp(i\left\langle t,\mu_{1}+\mu_{2}\right\rangle )\exp\left(-\frac{1}{2}\left\langle (\Sigma_{1}+\Sigma_{2})t,t\right\rangle \right).
\end{align*}
Hence, it follows that $X_{1}+X_{2}\sim\mathrm{MVN}(\mu_{1}+\mu_{2},\Sigma_{1}+\Sigma_{2}).$
Now suppose the claim holds for $m=k.$ Then for $m=k+1,$ we first
note that by inductive hypothesis $\sum_{i=1}^{k}X_{i}\sim\mathrm{MVN}(\sum_{i=1}^{k}\mu_{i},\sum_{i=1}^{k}\Sigma_{i}).$
By (??QUOTE), $(\sum_{i=1}^{k}X_{i})\perp X_{k+1},$ and as a result
\begin{align*}
\varphi_{\sum_{i=1}^{k+1}X_{i}}(t) & =\varphi_{\sum_{i=1}^{k}X_{i}}(t)\varphi_{X_{k+1}}(t)\\
 & =\exp\left(i\left\langle t,\sum_{i=1}^{k}\mu_{i}\right\rangle \right)\exp\left(-\frac{1}{2}\left\langle \left(\sum_{i=1}^{k}\Sigma_{k}\right)t,t\right\rangle \right)\exp(i\left\langle t,\mu_{k+1}\right\rangle )\exp\left(-\frac{1}{2}\left\langle \Sigma_{k+1}t,t\right\rangle \right)\\
 & =\exp\left(i\left\langle t,\sum_{i=1}^{k+1}\mu_{i}\right\rangle \right)\exp\left(-\frac{1}{2}\left\langle \left(\sum_{i=1}^{k+1}\Sigma_{k}t,t\right)\right\rangle \right)
\end{align*}
and as a result $\sum_{i=1}^{k+1}X_{i}\sim\mathrm{MVN}(\sum_{i=1}^{k+1}\mu_{i},\sum_{i=1}^{k+1}\Sigma_{i}).$
By the principle of mathematical induction, we conclude that the desired
result follows.

Therefore, by \ref{lem: sum of independent MVN}, $X+Z\sim\mathrm{MVN}(\mu_{x}+\mu_{z},\Sigma_{x}+\Sigma_{z}).$ 
\end{proof}

\subsection*{Extensions }

From this problem, we have shown that sum of independent Gaussians
is also Gaussian. There are some further implications can be derived,
which are also useful. The first result is a necessary and sufficient
condition for a random vector to be multivariate gaussian. 
\begin{lem}
Let $X=(X_{1},X_{2},\ldots,X_{n})$ be a $\mathbb{R}^{n}$-valued
random variable. Then $X$ is multivariate gaussian if and only if
$\left\langle t,X\right\rangle $ is univariate gaussian for any $t\in\mathbb{R}^{n}-\{0\}.$
\label{lem: necessary and sufficent condition to be MVN}
\end{lem}
\begin{proof}
$\Rightarrow$ Suppose that $X$ is multivariate gaussian. Then by
\ref{lem: distribution of linear transformation of gaussian}, $\left\langle t,X\right\rangle =t^{T}X$,
being a linear transformation of $X,$ is also gaussian. Moreover,
it is univariate gaussian since $t^{T}X\in\mathbb{R}$. \medskip\\
$\Leftarrow$ On the other hand, suppose $\left\langle t,X\right\rangle $
is univariate gaussian for any $t\in\mathbb{R}^{n}-\{0\}.$ Then it
follows that $X_{i}\sim\mathrm{N}(\mu_{i},\Sigma_{i})$ by letting
$t$ be $e_{i}$ for $i=1,...,n,$ where $e_{i}$ is the standard
basis vector in $\mathbb{R}^{n}.$ Therefore, $\E\left|X_{i}\right|<\infty,\E[X_{i}^{2}]<\infty$
and by Cauchy Schwartz $\E[X_{i}X_{j}]\leq\E\left|X_{i}X_{j}\right|\leq\E\left|X_{i}\right|^{1/2}\E\left|X_{j}\right|^{1/2}<\infty$
for any $i,j=1,...,n$. Therefore, $\E[X]$ and $\mathrm{Cov}[X]$
(the variance covariance matrix) is finite. We denote $\E[X]=\mu_{X}$
and $\mathrm{Cov}[X]=\Sigma_{X}.$ Note that with the notation defined
above, $t^{T}X\sim\mathrm{N}(t^{T}\mu_{X},t^{T}\Sigma_{X}t)$ for
any $t\in\mathbb{R}^{n}-\{0\}$
\[
\varphi_{t^{T}X}(r)=\E[\exp(irt^{T}X)]=\exp(irt^{T}\mu_{X})\exp\left(-\frac{r^{2}}{2}t^{T}\Sigma_{X}t\right).
\]
Also note that $\varphi_{X}(t)=\E[\exp(i\left\langle t,X\right\rangle )]=\E[\exp(it^{T}X)]=\varphi_{t^{T}X}(1).$
It follows that 
\[
\varphi_{X}(t)=\exp(it^{T}\mu_{X})\exp\left(-\frac{1}{2}t^{T}\Sigma_{X}t\right).
\]
Hence, $X\sim\mathrm{MVN}(\mu_{X},\Sigma_{X})$ in view of \ref{lem: cf of gaussian}. 
\end{proof}
An direct application of the previous lemma is that fact that if we
stack up independent multi dimensional gaussians, the stacked up vector
will also be random gaussian, which we state formally as a corollary
below. It should be noted as this result can also be shown using characteristic
functions directly. 
\begin{lem}
Let $\{X_{i}\}_{i=1}^{m}$ be a collection of independent random variables
such that $X_{i}\in\mathbb{R}^{n_{i}}$ and $X_{i}\sim\mathrm{MVN}(\mu_{i},\Sigma_{i}).$
Then 
\[
X=(X_{1},...,X_{m})\sim\mathrm{MVN}\left(\mu_{X}=\begin{bmatrix}\mu_{1}\\
\vdots\\
\mu_{m}
\end{bmatrix},\Sigma_{X}=\begin{bmatrix}\Sigma_{1} & 0 & 0 & 0\\
0 & \Sigma_{2} & 0 & 0\\
0 & 0 & \ddots & 0\\
0 & 0 & 0 & \Sigma_{n}
\end{bmatrix}\right).
\]
\end{lem}
\begin{proof}
For any $t\in\mathbb{R}^{\sum_{i=1}^{m}n_{i}}$, we note that by triangularizing
it follows that 
\[
\left\langle t,X\right\rangle =\sum_{i=1}^{\sum_{i=1}^{m}n_{i}}t_{i}X_{i}=\sum_{i=1}^{m}\bigg(\sum_{j=1}^{n_{i}}t_{ij}X_{ij}\bigg).
\]
Since by assumption $X_{i}=(X_{i1},X_{i2},...,X_{in_{i}})$ is gaussian,
by \ref{lem: necessary and sufficent condition to be MVN} it follow
that $\sum_{j=1}^{n_{i}}t_{ij}X_{ij}$ is univariate gaussian. By
well a well known result (for example c.f. QUOTE independence), $Y_{i}:=\sum_{j=1}^{n_{i}}t_{ij}X_{ij}\perp Y_{i'}=\sum_{j=1}^{n_{i'}}t_{i'j}X_{i'j}$
for $i\neq i'.$ Then by \ref{lem: sum of independent MVN} it follows
that $\left\langle t,X\right\rangle =\sum_{i=1}^{m}(\sum_{j=1}^{n_{i}}t_{ij}X_{ij})$
is gaussian. Therefore, \ref{lem: necessary and sufficent condition to be MVN}
implies that $X$ is multivariate gaussian. That $\mu_{X}$ and $\Sigma_{X}$
are of the form claimed can be verified by direct computation, which
we left an easy exercise. 
\end{proof}
%
\begin{cBoxA}[Problem 2.32 - Completing the squares trick for gaussian - 1]{}
 This exercise and the next provide practice at manipulating the
quadratic forms that arise in linear-Gaussian models, as well as giving
an independent check of results derived in the main text. Consider
a joint distribution $f(x,y)$ defined by the marginal and conditional
distributions given by (2.99) and (2.100). By examining the quadratic
form in the exponent of the joint distribution, and using the technique
of \textquotedbl completing the square\textquotedbl{} discussed in
Section 2.3, find expressions for the mean and covariance of the marginal
distribution $f(y)$ in which the variable $x$ has been integrated
out. To do this, make use of the Woodbury matrix inversion formula
(2.289). Verify that these results agree with (2.109) and (2.110)
obtained using the results of Chapter 2.
\end{cBoxA}

First, we give a lemma that formalize the process of \textquotedbl completing
the squares\textquotedbl{} for gaussian distribution. Note that most
of the derivation of the results about Gaussian distribution in the
book can be attributed to this lemma, although it is not mentioned
explicitly in the book. \\

\begin{lem}
Let $X$ be a $\mathbb{R}^{n}$-valued absolutely continuous random
variable with density function denoted as $f_{X}.$ If $f_{X}(x)\propto\exp(-\frac{1}{2}q(x)),$
where $q(x)=x^{T}\Lambda x-2b^{T}x+c$, in which $\Lambda\in\mathrm{Mat}_{\mathbb{R}}(n,n)$
and $\Lambda$ is positive definite, $b\in\mathbb{R}^{n},c\in\mathbb{R},$
then $X\sim\mathrm{MVN}(\Lambda^{-1}b,\Lambda^{-1}).$ \label{lem: gaussian completion of squares}
\end{lem}
\begin{proof}
First, note that by completing square we have that 
\[
q(x)=x^{T}\Lambda x-2b^{T}x+c=(x-\Lambda^{-1}b)^{T}\Lambda(x-\Lambda^{-1}b)-b^{T}\Lambda^{-1}b+c.
\]
Then it follows that 
\begin{align*}
\exp\left(-\frac{1}{2}q(x)\right) & =\exp\left(-\frac{1}{2}(x-\Lambda^{-1}b)^{T}\Lambda(x-\Lambda^{-1}b)\right)\exp\left(-\frac{1}{2}(b^{T}\Lambda^{-1}b-c)\right)\\
 & \propto\exp\left(-\frac{1}{2}(x-\Lambda^{-1}b)\Lambda(x-\Lambda^{-1}b)\right).
\end{align*}
Now note that $f_{X}(x)$, being a density should integrate to $1$
and 
\[
\int\frac{1}{(2\pi)^{n/2}\left|\det\Lambda^{-1}\right|^{1/2}}\exp\left(-\frac{1}{2}(x-\Lambda^{-1}b)\Lambda(x-\Lambda^{-1}b)\right)dx=1,\tag{1}
\]
it follows that $f_{X}(x)$ should be the gaussian density proposed
in Eq.(1). Hence, it follows that $X\sim\mathrm{MVN}(\Lambda^{-1}b,\Lambda^{-1}).$
\end{proof}
%
With this lemma, we now solve this problem. Note that 
\begin{align*}
f_{(X,Y)}(x,y) & =f_{X}(x)f_{Y\vert X}(y\vert x)\\
 & =C\exp\bigg(\underbrace{-\frac{1}{2}(x-\mu)^{T}\Lambda(x-\mu)-\frac{1}{2}(y-(Ax+b))L^{T}(y-(Ax+b))}_{:=Q(x,y)}\bigg),
\end{align*}
where $C$ is some normalizing constant. Now we message $Q(x,y)$
a little bit by grouping all terms related to $x$ together: 
\begin{align*}
Q(x,y) & =-\frac{1}{2}\left[(x-\mu)^{T}\Lambda(x-\mu)+(y-(Ax+b))^{T}L(y-(Ax+b))\right]\\
 & =-\frac{1}{2}\left[x^{T}\Lambda x-2x^{T}\Lambda\mu+\mu^{T}\Lambda\mu+y^{T}Ly-2y^{T}L(Ax+b)+(Ax+b)^{T}L(Ax+b)\right]\\
 & =-\frac{1}{2}\left[x^{T}\Lambda x+x^{T}A^{T}LAx+2x^{T}A^{T}Lb-2x^{T}A^{T}Ly+2x^{T}\Lambda\mu-2y^{T}Lb+y^{T}Ly+\mu^{T}\Lambda\mu+b^{T}Lb\right]\\
 & =-\frac{1}{2}\left[x^{T}(\Lambda+A^{T}LA)x+2x^{T}(\Lambda\mu+A^{T}Lb-A^{T}Ly)\right]+R(y)\\
 & =-\frac{1}{2}\left[x^{T}(\Lambda+A^{T}LA)x-2x^{T}(\Lambda\mu+A^{T}L(y-b))\right]+R(y)\tag{1}\\
 & =-\frac{1}{2}\left[(x-\widetilde{\mu})^{T}(\Lambda+A^{T}LA)(x-\widetilde{\mu})+(\Lambda\mu+A^{T}L(y-b))^{T}(\Lambda+A^{T}LA)^{-1}(\Lambda\mu+A^{T}L(y-b))\right]+R(y)\\
 & =-\frac{1}{2}(x-\widetilde{\mu})^{T}(\Lambda+A^{T}LA)(x-\widetilde{\mu})\\
 & \ \ +\frac{1}{2}(\Lambda\mu+A^{T}L(y-b))^{T}(\Lambda+A^{T}LA)^{-1}(\Lambda+A^{T}LA)(\Lambda+A^{T}LA)^{-1}(\Lambda\mu+A^{T}L(y-b))+R(y)\\
 & =\underbrace{-\frac{1}{2}(x-\widetilde{\mu})^{T}(\Lambda+A^{T}LA)(x-\widetilde{\mu})}_{:=\mathcal{H}_{1}(x)}+\underbrace{\frac{1}{2}\widetilde{\mu}^{T}(\Lambda+A^{T}LA)\widetilde{\mu}+R(y)}_{:=\mathcal{H}_{2}(y)}.
\end{align*}
where $\widetilde{\mu}=(\Lambda+A^{T}LA)^{-1}(\Lambda\mu+A^{T}L(y-b))$
and $R(y)=-\frac{1}{2}\mu^{T}\Lambda\mu-\frac{1}{2}b^{T}Lb-\frac{1}{2}y^{T}Ly+y^{T}Lb.$
Therefore, we have that 
\begin{align*}
f_{Y}(y) & =\int f_{(X,Y)}(x,y)dx=\int\frac{1}{(2\pi)^{n/2}\left|\det(\Lambda+A^{T}LA)\right|^{1/2}}\exp\left(\mathcal{H}_{1}(x)\right)dx\cdot\widetilde{C}\exp(\mathcal{H}_{2}(y))\\
 & =\widetilde{C}\exp(\mathcal{H}_{2}(y)).
\end{align*}
where $\widetilde{C}$ a new normalizing constant. Now we do the completion
of the square again for $f_{Y}(y)$ as follows: 
\begin{align*}
 & f_{Y}(y)\\
=\ \  & \widetilde{C}\exp(\widetilde{\mu}^{T}(\Lambda+A^{T}LA)\widetilde{\mu}+R(y))\\
\propto\ \  & \exp\left(\frac{1}{2}[(\Lambda+A^{T}LA)^{-1}(\Lambda\mu+A^{T}L(y-b))]^{T}(\Lambda+A^{T}LA)[(\Lambda+A^{T}LA)^{-1}(\Lambda\mu+A^{T}L(y-b))]\right)\\
 & \times\exp\left(-\frac{1}{2}\mu^{T}\Lambda\mu-\frac{1}{2}b^{T}Lb-\frac{1}{2}y^{T}Ly+2y^{T}Lb\right)\\
=\ \  & \exp\left(\frac{1}{2}(\Lambda\mu+A^{T}L(y-b)){}^{T}(\Lambda+A^{T}LA)^{-1}(\Lambda\mu+A^{T}L(y-b))\right)\\
 & \times\exp\left(-\frac{1}{2}\mu^{T}\Lambda\mu-\frac{1}{2}b^{T}Lb-\frac{1}{2}y^{T}Ly+y^{T}Lb\right)\\
\propto\ \  & \exp\bigg(-\frac{1}{2}y^{T}Ly+y^{T}Lb+\frac{1}{2}y^{T}LA(\Lambda+A^{T}LA)^{-1}A^{T}Ly+(\Lambda\mu)^{T}(\Lambda+A^{T}LA)^{-1}A^{T}Ly\\
 & \ \ \ \ \ \ -y^{T}LA(\Lambda+A^{T}LA)^{-1}A^{T}Lb\bigg)\\
=\ \  & \exp\left(-\frac{1}{2}y^{T}(L-LA(\Lambda+A^{T}LA)^{-1}A^{T}L)y+y^{T}[(L-LA(\Lambda+A^{T}LA)^{-1}A^{T}L)b+LA(\Lambda+A^{T}LA)^{-1}\Lambda\mu]\right).
\end{align*}
Therefore, by \ref{lem: gaussian completion of squares} it follows
that $Y\sim\mathrm{MVN}(\mu_{Y},\Sigma_{Y})$, where 
\begin{align*}
\Sigma_{Y} & =(L-LA(\Lambda+A^{T}LA)^{-1}A^{T}L)^{-1}\\
 & =L^{-1}-L^{-1}(-LA)[(\Lambda+A^{T}LA)+A^{T}LL^{-1}(-LA)]^{-1}A^{T}LL^{-1}\tag{by Woodbury inversion formula}\\
 & =L^{-1}+A[\Lambda+A^{T}LA-A^{T}LA]^{-1}A^{T}\\
 & =L^{-1}+A\Lambda^{-1}A^{T},
\end{align*}
and 
\begin{align*}
\mu_{Y} & =(L-LA(\Lambda+A^{T}LA)^{-1}A^{T}L)^{-1}[(L-LA(\Lambda+A^{T}LA)^{-1}A^{T}L)b+LA(\Lambda+A^{T}LA)^{-1}\Lambda\mu]\\
 & =(L^{-1}+A\Lambda^{-1}A^{T})[(L^{-1}+A\Lambda^{-1}A^{T})^{-1}b+LA(\Lambda+A^{T}LA)^{-1}\Lambda\mu]\\
 & =b+(L^{-1}+A\Lambda^{-1}A^{T})LA(\Lambda+A^{T}LA)^{-1}\Lambda\mu\\
 & =b+(A+A\Lambda^{-1}A^{T}LA)(\Lambda+A^{T}LA)^{-1}\Lambda\mu=b+A(I+\Lambda^{-1}A^{T}LA)(I+\Lambda^{-1}A^{T}LA)^{-1}\Lambda^{-1}\Lambda\mu\\
 & =A\mu+b.
\end{align*}
\\
\begin{cBoxA}[Problem 2.33 - Completing the squares trick for gaussian - 2 ]{}
 Consider the same joint distribution as in Exercise 2.32, but now
use the technique of completing the square to find expressions for
the mean and covariance of the conditional distribution $f(x\vert y)$.
Again, verify that these agree with the corresponding expressions
(2.111) and (2.112).
\end{cBoxA}

First, using the same argument as in Problem 2.32 we see that 
\[
f_{(X,Y)}(x,y)=C\exp\bigg(\underbrace{-\frac{1}{2}(x-\mu)^{T}\Lambda(x-\mu)-\frac{1}{2}(y-(Ax+b))L^{T}(y-(Ax+b))}_{:=Q(x,y)}\bigg).
\]
And following Eq.(1) in Problem 2.33, we have 
\begin{align*}
Q(x,y) & =-\frac{1}{2}\left[x^{T}(\Lambda+A^{T}LA)x-2x^{T}(\Lambda\mu+A^{T}L(y-b))\right]+R(y).
\end{align*}
Then it follows from \ref{lem: gaussian completion of squares} that
\[
X\vert Y=y\sim\mathrm{MVN}(\mu_{X|Y}=(\Lambda+A^{T}LA)^{-1}(\Lambda\mu+A^{T}L(y-b)),\Sigma_{X|Y}=(\Lambda+A^{T}LA)^{-1}.
\]
\\

\begin{cBoxA}[Problem 2.34 - MLE of covariance matrix for multivariate gaussian ]{}
 To find the maximum likelihood solution for the covariance matrix
of a multivariate Gaussian, we need to maximize the log likelihood
function (2.118) with respect to $\Sigma$, noting that the covariance
matrix must be symmetric and positive definite. Here we proceed by
ignoring these constraints and doing a straightforward maximization.
Using the results (C.21), (C.26), and (C.28) from Appendix C, show
that the covariance matrix $\Sigma$ that maximizes the log likelihood
function (2.118) is given by the sample covariance (2.122). We note
that the final result is necessarily symmetric and positive definite
(provided the sample covariance is nonsingular).
\end{cBoxA}

\textit{This problem involves differentiating the log-determinant
of a positive definite matrix. Although relevant formulas have been
provided in the back of the book, they are somehow not very rigorous
in its presentation. Hence, through the solution to this problem we
would develop some results w.r.t to the determinant of the matrix
determinants.}

\medskip Before we start proving results, we stop and recall the
definition of determinant. The most generalized definition of determinant
involves the notion of manifold and Lie groups. For practical purposes
it suffices to define determinant in the way that characterizes it
as a function as follows:
\begin{thm}
There exists a unique function $D:(\mathbb{R}^{n})^{n}\rightarrow\mathbb{R}$
such that 
\begin{enumerate}
\item $D$ is multilinear, i.e., $D$ is linear w.r.t to each of its arguments.
\item $D$ is anti-symmetric: Exchanging any two arguments changes its sign.
\item $D$ is normalized: $D(e_{1},\ldots,e_{n})=1$ the set of the standard
basis vectors $\{e_{i}\}_{i=1}^{n}$ in $\mathbb{R}^{n}.$
\end{enumerate}
The determinant of an $n\times n$ matrix $A=[a_{1},a_{2},\ldots,a_{n}]$
is defined as $\det A=D(a_{1},\ldots,a_{n})$ where $a_{1},\ldots a_{n}\in\mathbb{R}^{n}$
are the columns of the matrix $A$. 
\end{thm}
In order the find the Frechet derivative of the determinant function,
we need a specific characterization of the determinant function that
is easy to work with. 
\begin{lem}
Let $A=[a_{ij}]_{ij}\in\mathrm{Mat}_{\mathbb{R}}(n,n).$ Then \label{lem: characterization of determinant}
\[
\det A=\sum_{\sigma\in S_{n}}\mathrm{sgn}(\sigma)a_{1,\sigma(1)}\ldots a_{n,\sigma(n)},
\]
where $S_{n}$ is the permutation group. 
\end{lem}
%
With the previous two well-known results, we are ready the calculate
the derivative of the determinant function. 
\begin{lem}
The following results about the determinant function are true:\label{lem: log-det differential}
\begin{enumerate}
\item The determinant function is Frechet differentiable.
\item The derivative of the determinant at the identity is given by $[\mathrm{D}\det(I)]B=\mathrm{tr}(B).$ 
\item If $\det A\neq0,$ then $[\mathrm{D}\det(A)]B=\det A\cdot\mathrm{tr}(A^{-1}B).$ 
\end{enumerate}
\end{lem}
\begin{proof}
We first find the Frechet differential of the determinant function
at the identity matrix, $I$. Since all norms are equivalent, it suffices
to find the Frechet derivative w.r.t to an arbitrarily chosen norm.
For convenience, we choose this norm to be the max norm, $\left\Vert A\right\Vert _{\infty}=\max_{i,j}|a_{i,j}|.$
Now consider an arbitrary matrix $H,$ we see that 
\begin{align*}
\det(I+H) & =\det\begin{bmatrix}1+h_{1,1} & h_{1,2} & \cdots & h_{1,n}\\
h_{2,1} & 1+h_{2,2} & \cdots & h_{2,n}\\
\vdots & \cdots & \ddots & \vdots\\
h_{n,1} & h_{n,2} & \cdots & 1+h_{n,n}
\end{bmatrix}\\
 & =\sum_{\sigma\in S_{n}}\mathrm{sgn}(\sigma)\prod_{i=1}^{n}(I+H)_{i,\sigma(i)}\\
 & =\mathrm{sgn}(\mathrm{id})\prod_{i=1}^{n}(I+H)_{i,i}+\sum_{\sigma\in S_{n},\sigma\neq\mathrm{id}}\prod_{i=1}^{n}(I+H)_{i,\sigma(i)}\\
 & =\prod_{i=1}^{n}(1+h_{i,i})+\sum_{\sigma\in S_{n},\sigma\neq\mathrm{id}}\prod_{i=1}^{n}(I+H)_{i,\sigma(i).}\tag{1}
\end{align*}
Now we analysis the terms in Eq.(1) term by term: 
\begin{align*}
\prod_{i=1}^{n}(1+h_{i,i}) & =1+\sum_{i=1}^{n}h_{i,i}+\underbrace{\sum_{i_{1},i_{2}\in\{1,...,n\},i_{1}\neq i_{2}}h_{i_{1}}h_{i_{2}}}_{\leq\binom{n}{2}\left\Vert H\right\Vert _{\infty}^{2}=o(\left\Vert H\right\Vert _{\infty})}+\cdots+\underbrace{h_{i_{1}}h_{i_{2}}\ldots h_{i_{n}}}_{\leq\left\Vert H\right\Vert ^{n}=o(\left\Vert H\right\Vert _{\infty})}\\
 & =\det I+\mathrm{tr}(H)+o(\left\Vert H\right\Vert _{\infty}).
\end{align*}
On the other hand, note that for $\sigma\notin\mathrm{id}$, there
are at least two terms of $(I+H)_{i,\sigma(i)}$ are off diagonal
since every permutation can be written as a product of disjoint cycles
and identity element is the only element in the permutation group
that has a representation as product of $1$-cycles\footnote{this is because group identity is unique. };
so $\sigma$ has at least one $2$ cycle in its product representation,
which means there exists at least one pair $(i,j)$ with $i\neq j$
such that $(I+H)_{i,\sigma(i)}=(I+H)_{i,j}=h_{i,j}$ and similarly
$(I+H)_{j,\sigma(j)}=h_{j,i}$. As a result, 
\[
\prod_{i=1}^{n}(I+H)_{i,\sigma(i)}=h_{i}h_{j}\prod_{k\in\{1,...,n\}-\{i,j\}}(I+H)_{i,\sigma(i)}\leq\left\Vert H\right\Vert _{\infty}^{2}\prod_{k\in\{1,...,n\}-\{i,j\}}(I+H)_{i,\sigma(i)}=o(\left\Vert H\right\Vert _{\infty})
\]
Therefore, it follows that 
\[
\mathrm{Eq}.(1)=\det I+\mathrm{tr}(H)+o(\left\Vert H\right\Vert _{\infty}).
\]
Hence, by definition we have $[\mathrm{D}\det I](H)=\mathrm{tr}(H)$.
\medskip\\
Now we find the Frechet differential of the determinant function at
any invertible matrix $A$.To start with, we note that 
\begin{align*}
\det(A+H) & =\det A\det(I+A^{-1}H)=\det A(\det I+\mathrm{tr}(A^{-1}H)+o(\left\Vert A^{-1}H\right\Vert ))\\
 & =\det A+\det A\cdot\mathrm{tr}(A^{-1}H)+o(\left\Vert A^{-1}H\right\Vert )\\
 & =\det A+\det(A)\cdot\mathrm{tr}(A^{-1}H)+o(\left\Vert H\right\Vert ),
\end{align*}
where $\left\Vert \cdot\right\Vert $ is any consistent norm of our
choosing. Since $\det(A)\cdot\mathrm{tr}(A^{-1}H)=\left\langle \det A\cdot A^{-1},H\right\rangle ,$
it follow that $[\mathrm{D}\det(A)](H)=\det$$\nabla\det(A)=\det A\cdot A^{-1}.$ 
\end{proof}
We have an intermediate corollary. 
\begin{cor}
Let $A\in\mathrm{GL}_{\mathbb{R}}(n,n).$ Then $\nabla\log\det A=A^{-1}.$ 
\end{cor}
\begin{proof}
By the chain rule, 
\[
\mathrm{D}(\log\det A)(H)=\frac{1}{\det A}\circ\mathrm{D}(\det A)=\frac{1}{\det A}\det A\left\langle A^{-1},H\right\rangle =\left\langle A^{-1},H\right\rangle .
\]
Therefore, it follows that $\nabla\log\det A=A^{-1}.$
\end{proof}
We still need one more result to reach fully rigorous solution to
this problem; that is the differential of the function $\mathrm{GL}_{\mathbb{R}}(n)\rightarrow\mathrm{GL}_{\mathbb{R}}(n)$
defined by $A\mapsto A^{-1}.$ First, a lemma coupled with a definition. 
\begin{defn}
A Banach algebra is an associative algebra with unit $1$ over complex
(or real) numbers that is at the same time a Banach space, and so
that the norm is sub-multiplicative and $\left\Vert 1\right\Vert =1.$
\end{defn}
\begin{lem}
\label{lem: neumann series}Let $\mathcal{A}$ be a unital Banach
algebra, and $a\in\mathcal{A}$ with $\left\Vert a\right\Vert <1.$
Then $1-a$ is invertible with inverse $(1-a)^{-1}=\sum_{i=0}^{\infty}a^{n}.$
\end{lem}
\begin{proof}
For each $n\in\mathbb{N},$ define the partial sums $S_{n}=\sum_{i=0}^{n}a_{i}.$
Note that $\left\Vert a\right\Vert <1$; so $\left\Vert a^{n}\right\Vert \leq\left\Vert a\right\Vert ^{n}<1$
and it follows that $\sum_{i=0}^{\infty}\left\Vert a^{n}\right\Vert \leq\sum_{i=0}^{\infty}\left\Vert a\right\Vert ^{n}<\infty,$
whence $\sum_{i=0}^{\infty}a^{n}$ absolutely convergent and thus
convergent. We then denote $S=\lim_{i=0}^{\infty}a_{i}=\lim_{n}S_{n}$.
Now we note that 
\[
(1-a)S_{n}=(1-a)\left(\sum_{i=0}^{n}a^{i}\right)=1-a^{n+1}.
\]
Now we take the limit, it follows that 
\[
\lim_{n\rightarrow\infty}(1-a)S_{n}=(1-a)\lim_{n\rightarrow\infty}S_{n}=(1-a)S=1-\lim_{n\rightarrow\infty}a^{n+1}=1.
\]
And similarly, 
\[
\lim_{n\rightarrow\infty}[S_{n}(1-a)]=(\lim_{n\rightarrow}S_{n})(1-a)=S(1-a)=1-\lim_{n\rightarrow\infty}a^{n+1}=1.
\]
As a result, $S=a^{-1}.$ 
\end{proof}
\begin{rem}
A direct application of the this lemma is that in the context of matrix
inversion. Let $A\in\mathrm{GL}_{\mathbb{R}}(n);$ if $\left\Vert A\right\Vert <1$,
then it follows that $(I-A)$ is invertible and $(I-A)^{-1}=\sum_{i=0}^{\infty}A^{i}.$ 
\end{rem}
Now with the Neumann series, have the following lemma. 
\begin{lem}
Let $f$ be a function $\mathrm{GL}_{\mathbb{R}}(n)\rightarrow\mathrm{GL}_{\mathbb{R}}(n):A\mapsto A^{-1}.$
Then $(\mathrm{D}f(A))(H)=-A^{-1}HA^{-1}.$
\end{lem}
\begin{proof}
We fix $\left\Vert \cdot\right\Vert $ to be any consistent matrix
norm. First, we note that for $H\in\mathrm{GL}_{\mathbb{R}}(n)$ such
that $\left\Vert A^{-1}H\right\Vert <\frac{1}{2}.$
\begin{align*}
f(A+H) & =(A+H)^{-1}-A^{-1}=A(I-(-(A^{-1}H)^{-1})\\
 & =\left[A\left(\sum_{i=0}^{\infty}(-A^{-1}H)^{-i}\right)\right]^{-1}\\
 & =\left[\left(I-A^{-1}H+\sum_{i=2}^{\infty}(-1)^{i}(A^{-1}H)^{i}\right)A^{-1}\right]\\
 & =A^{-1}+(-A^{-1}HA^{-1})+\sum_{i=2}^{\infty}(-1)^{i}(A^{-1}H)^{i}A^{-1}.\tag{1}
\end{align*}
Next, note that 
\begin{align*}
\left\Vert \sum_{i=2}^{\infty}(-1)^{i}(A^{-1}H)^{i}A^{-1}\right\Vert  & \leq\sum_{i=2}^{\infty}\left\Vert (A^{-1}H)^{i}A^{-1}\right\Vert \leq\left\Vert A^{-1}\right\Vert \sum_{i=2}^{\infty}\left\Vert (A^{-1}H)^{i}\right\Vert \leq\left\Vert A^{-1}\right\Vert \sum_{i=2}^{\infty}\left\Vert A^{-1}H\right\Vert ^{i}\\
 & =\left\Vert A^{-1}\right\Vert \left\Vert A^{-1}H\right\Vert \sum_{i=1}^{\infty}\left\Vert A^{-1}H\right\Vert ^{i}=\frac{\left\Vert A^{-1}H\right\Vert \left\Vert A^{-1}\right\Vert \left\Vert A^{-1}H\right\Vert }{1-\left\Vert A^{-1}H\right\Vert }\leq\frac{\left\Vert A^{-1}\right\Vert ^{3}\left\Vert H\right\Vert ^{2}}{1-\left\Vert A^{-1}H\right\Vert }\\
 & \leq2\left\Vert A^{-1}\right\Vert ^{3}\left\Vert H\right\Vert ^{2}\tag{since \ensuremath{\left\Vert A^{-1}H\right\Vert \leq\frac{1}{2}} by assumption}.
\end{align*}
Hence, it follows that 
\[
\frac{\left\Vert \sum_{i=2}^{\infty}(-1)^{i}(A^{-1}H)^{i}A^{-1}\right\Vert }{\left\Vert H\right\Vert }\leq\frac{2\left\Vert A^{-1}\right\Vert ^{3}\left\Vert H\right\Vert ^{2}}{\left\Vert H\right\Vert }=2\left\Vert A^{-1}\right\Vert ^{3}\left\Vert H\right\Vert \xrightarrow{\left\Vert H\right\Vert \rightarrow0}0.
\]
It follows that 
\[
f(A+H)=A^{-1}+(-A^{-1}HA^{-1})+o(\left\Vert H\right\Vert ).
\]
Since the function $\mathrm{GL}_{\mathbb{R}}(n)\ni H\mapsto-A^{-1}HA^{-1}$
is in $\mathrm{Hom}(\mathrm{GL}_{\mathbb{R}}(n),\mathrm{GL}_{\mathbb{R}}(n)),$
it follows that $[\mathrm{D}f(A)]\circ H=-A^{-1}HA^{-1}.$ 
\end{proof}
Now we come back to solve this problem. First, we write out the likelihood
equation as follows: 
\[
\mathcal{L}(\mu,\Sigma)=\prod_{i=1}^{d}\frac{1}{(2\pi)^{n/2}\det\Sigma^{1/2}}\exp\left(-\frac{1}{2}(x_{i}-\mu)^{T}\Sigma^{-1}(x_{i}-\mu)\right).
\]
Take the logarithm and we get 
\begin{align*}
\ell(\mu,\Sigma) & =\sum_{i=1}^{d}\left[\log\frac{1}{(2\pi)^{n/2}\det\Sigma^{1/2}}-\frac{1}{2}(x_{i}-\mu)^{T}\Sigma^{-1}(x_{i}-\mu)\right]\\
 & =-\sum_{i=1}^{d}\log\frac{1}{(2\pi)^{n/2}}-\sum_{i=1}^{d}\log\frac{1}{\det\Sigma^{1/2}}-\sum_{i=1}^{d}(x_{i}-\mu)^{T}\Sigma^{-1}(x_{i}-\mu)\\
 & =-\frac{nd}{2}\log2\pi-\frac{d}{2}\log\det\Sigma-\sum_{i=1}^{d}(x_{i}-\mu)^{T}\Sigma^{-1}(x_{i}-\mu).
\end{align*}
To find the critical points, we set the gradient to zero. First, we
find the gradient. Note that by the product rule, we have that 
\[
\mathrm{D}_{\Sigma}\ell(\mu,\Sigma)=-\frac{d}{2}(\mathrm{D}_{\Sigma^{-1}}\log\det\Sigma)-\frac{1}{2}\sum_{i=1}^{d}\mathrm{D}_{\Sigma^{-1}}(x_{i}-\mu)^{T}\Sigma^{-1}(x_{i}-\mu).\tag{2 }
\]
By \ref{lem: log-det differential}, it follows that $\mathrm{D}\log\det\Sigma=\Sigma^{-1}.$
For the term $(x_{i}-\mu)^{T}\Sigma^{-1}(x_{i}-\mu),$ we see that
by \ref{lem: prod rule}
\begin{align*}
\mathrm{\mathrm{D}}_{\Sigma}[(x_{i}-\mu)^{T}\Sigma^{-1}(x_{i}-\mu)]\circ H & =\mathrm{D}_{\Sigma}\left\langle \Sigma^{-1}(x_{i}-\mu),(x_{i}-\mu)\right\rangle \circ H\\
 & =\mathrm{D}_{\Sigma}\left\langle \Sigma^{-1},(x_{i}-\mu)(x_{i}-\mu)^{T}\right\rangle \circ H\\
 & =\left\langle (\mathrm{D}_{\Sigma}\Sigma^{-1})\circ H,(x_{i}-\mu)(x_{i}-\mu)^{T}\right\rangle \\
 & =\left\langle (x_{i}-\mu)(x_{i}-\mu)^{T},-\Sigma^{-1}H\Sigma^{-1}\right\rangle \\
 & =\left\langle -\Sigma^{-1}(x_{i}-\mu)(x_{i}-\mu)^{T}\Sigma^{-1},H\right\rangle .
\end{align*}
Therefore, it follows that $\mathrm{D}_{\Sigma}[(x_{i}-\mu)^{T}\Sigma^{-1}(x_{i}-\mu)]=\Sigma^{-1}(x_{i}-\mu)(x_{i}-\mu)^{T}\Sigma^{-1}.$
Now substitute these result back into Eq.(2) and we get 
\[
\nabla_{\Sigma}\ell(\mu,\Sigma)=-\frac{d}{2}\Sigma^{-1}+\frac{1}{2}\sum_{i=1}^{d}\Sigma^{-1}(x_{i}-\mu)(x_{i}-\mu)^{T}\Sigma^{-1}.
\]
 Setting it to zero yields 
\begin{align*}
\frac{1}{2}\sum_{i=1}^{d}\Sigma^{-1}(x_{i}-\mu)(x_{i}-\mu)^{T}\Sigma^{-1}=\frac{d}{2}\Sigma^{-1} & \implies I=\frac{1}{d}\Sigma^{-1}\sum_{i=1}^{d}(x_{i}-\mu)(x_{i}-\mu)^{T}\\
 & \implies\Sigma=\frac{1}{d}\sum_{i=1}^{d}(x_{i}-\mu)(x_{i}-\mu)^{T}.
\end{align*}

\begin{cBoxA}[Problem 2.35 - Expectation of $\Sigma_{MLE}$ in multivariate gaussian]{}
 Use the result (2.59) to prove (2.62). Now, using the results (2.59),
and (2.62), show that 
\[
\E[x_{n}x_{m}]=\mu\mu^{T}+I_{nm}\Sigma
\]
where $x_{n}$ denotes a data point sampled from a Gaussian distribution
with mean $\mu$ and covariance $\Sigma$ and $I_{nm}$ denotes the
$(n,m)$ element of the identity matrix. Hence, prove the result (2.124).
\end{cBoxA}

Recall in Problem 3.34, we have that $\Sigma_{MLE}=\frac{1}{n}\sum_{i=1}^{n}(X_{i}-\mu_{MLE})(X_{i}-\mu_{MLE})^{T}.$
Then we have 
\begin{align*}
\E[\Sigma_{MLE}] & =\frac{1}{n}\sum_{i=1}^{n}\E[(X_{i}-\mu_{MLE})(X_{i}-\mu_{MLE})^{T}]\\
 & =\frac{1}{n}\sum_{i=1}^{n}\underbrace{\E[X_{i}X_{i}^{T}-X_{i}\mu_{MLE}^{T}-\mu_{MLE}X_{i}^{T}+\mu_{MLE}\mu_{MLE}^{T}]}_{:=\mathcal{H}(X_{i})}.
\end{align*}
Now we note that 
\begin{align*}
\mathcal{H}(X_{i}) & =\E\bigg[X_{i}X_{i}^{T}-X_{i}\bigg(\frac{1}{n}\sum_{j=1}^{n}X_{j}\bigg)^{T}-\bigg(\sum_{j=1}^{n}X_{i}\bigg)X_{i}^{T}+\bigg(\sum_{i=1}^{n}X_{i}\bigg)\bigg(\sum_{i=1}^{n}X_{i}\bigg)^{T}\bigg]\\
 & =\E\bigg[X_{i}X_{i}^{T}-\frac{2}{n}\sum_{j=1}^{n}X_{i}X_{j}+\bigg(\sum_{i=1}^{n}X_{i}\bigg)\bigg(\sum_{i=1}^{n}X_{i}\bigg)^{T}\bigg]\\
 & =\mu\mu^{T}+\Sigma-\frac{2}{n}\bigg(\sum_{j=1}^{n}\mu\mu^{T}+\Sigma\bigg)+\frac{1}{n^{2}}\bigg[\sum_{i,j=1}^{n}\mu\mu^{T}+\sum_{i=1}^{n}\Sigma\bigg]\\
 & =\mu\mu^{T}+\Sigma-2\mu\mu^{T}-\frac{2}{n}\Sigma+\mu\mu^{T}+\frac{1}{n}\Sigma\\
 & =\left(1-\frac{1}{n}\right)\Sigma.
\end{align*}
\\

\begin{cBoxA}[Problem 2.36 - Sequential estimation of gaussian covariance - univariate
case]{}
 Using an analogous procedure to that used to obtain (2.126), derive
an expression for the sequential estimation of the variance of a univariate
Gaussian distribution, by starting with the maximum likelihood expression
\[
\sigma_{ML}^{2}=\frac{1}{n}\sum_{i=1}^{n}(x_{n}-\mu)^{2}
\]
Verify that substituting the expression for a Gaussian distribution
into the Robbins-Monro sequential estimation formula (2.135) gives
a result of the same form, and hence obtain an expression for the
corresponding coefficients $a_{n}$.
\end{cBoxA}

First, we note that 
\begin{align*}
\sigma_{n}^{2} & =\frac{1}{n}\sum_{i=1}^{n}(x_{i}-\mu)^{2}=\frac{n-1}{n}\frac{1}{n-1}\bigg[\sum_{i=1}^{n-1}(x_{i}-\mu)^{2}+(x_{n}-\mu)^{2}\bigg]\\
 & =\left(1-\frac{1}{n}\right)\sigma_{n-1}^{2}+\frac{1}{n}(x_{n}-\mu)^{2}\\
 & =\sigma_{n-1}^{2}+\frac{1}{n}[(x_{n}-\mu)^{2}-\sigma_{n-1}^{2}].\tag{1}
\end{align*}
On the other hand, using the sequential estimation formula, Eq.(2.135),
we have that 
\begin{align*}
\sigma_{n}^{2} & =\sigma_{n-1}^{2}+a_{n-1}\frac{\partial}{\partial\sigma_{n-1}^{2}}\log\left[\frac{1}{(2\pi\sigma_{n-1}^{2})^{1/2}}\exp\left\{ -\frac{(x_{n}-\mu)^{2}}{2\sigma_{n-1}^{2}}\right\} \right]\\
 & =\sigma_{n-1}^{2}+a_{n-1}\left[\frac{(x_{n}-\mu)^{2}}{2\sigma_{n-1}^{4}}-\frac{1}{2\sigma_{n-1}^{2}}\right]\\
 & =\sigma_{n-1}^{2}+\frac{a_{n-1}}{2\sigma_{n-1}^{4}}[(x_{n}-\mu)^{2}-\sigma_{n-1}^{2}].\tag{2}
\end{align*}
Compared Eq.(1) and Eq.(2), we see that 
\[
\frac{1}{n}=\frac{a_{n-1}}{2\sigma_{n-1}^{4}}\implies a_{n-1}=\frac{2\sigma_{n-1}^{4}}{n}.
\]
\\

\begin{cBoxA}[Problem 2.37 - Sequential estimation of gaussian covariance - multivariate
case]{}
 Using an analogous procedure to that used to obtain (2.126), derive
an expression for the sequential estimation of the covariance of a
multivariate Gaussian distribution, by starting with the maximum likelihood
expression (2.122). Verify that substituting the expression for a
Gaussian distribution into the Robbins-Monro sequential estimation
formula (2.135) gives a result of the same form, and hence obtain
an expression for the corresponding coefficients $a_{n}$.
\end{cBoxA}

Using the result in Problem 2.34 we have 
\begin{align*}
\Sigma_{n} & =\frac{1}{n}\sum_{i=1}^{n}(x_{i}-\mu)(x_{i}-\mu)^{T}=\frac{n-1}{n}\frac{1}{n-1}\left[\sum_{i=1}^{n}(x_{i}-\mu)(x_{i}-\mu)^{T}\right]\\
 & =\left(1-\frac{1}{n}\right)\frac{1}{n-1}\left[\sum_{i=1}^{n-1}(x_{i}-\mu)(x_{i}-\mu)^{T}+(x_{n}-\mu)(x_{n}-\mu)^{T}\right]\\
 & =\left(1-\frac{1}{n}\right)\Sigma_{n-1}+\frac{1}{n}(x_{n}-\mu)(x_{n}-\mu)^{T}\\
 & =\Sigma_{n-1}+\frac{1}{n}\left[(x_{n}-\mu)(x_{n}-\mu)^{T}-\Sigma_{n-1}\right].\tag{1 }
\end{align*}
On the other hand, using the sequential estimation formula, Eq.(2.135)
, we see that 
\begin{align*}
\Sigma_{n} & =\Sigma_{n-1}+a_{n-1}\frac{\partial}{\partial\Sigma_{n-1}}\log\left[\frac{1}{(2\pi)^{d/2}(\det\Sigma_{n-1})^{1/2}}\exp\left(-\frac{1}{2}(x_{n}-\mu)^{T}\Sigma_{n-1}^{-1}(x_{n}-\mu)\right)\right]\\
 & =\Sigma_{n-1}+a_{n-1}\mathrm{D}_{\Sigma_{n-1}}\left(-\frac{d}{2}\log2\pi-\frac{1}{2}\log\det\Sigma_{n-1}-\frac{1}{2}(x_{n}-\mu)^{T}\Sigma_{n-1}^{-1}(x_{n}-\mu)\right)\\
 & =\Sigma_{n-1}-\frac{1}{2}a_{n-1}(\mathrm{D}_{\Sigma_{n-1}}\log\det\Sigma_{n-1}+\mathrm{D}_{\Sigma_{n-1}}\left\langle \Sigma_{n-1}^{-1}(x_{n}-\mu)^{T},(x_{n}-\mu)\right\rangle )\\
 & =\Sigma_{n-1}-\frac{1}{2}a_{n-1}(\Sigma_{n-1}^{-1}-\mathrm{D}_{\Sigma_{n-1}}\left\langle \Sigma_{n-1}^{-1},(x_{n}-\mu)(x_{n}-\mu)^{T}\right\rangle )\\
 & =\Sigma_{n-1}-\frac{1}{2}a_{n-1}(\Sigma_{n-1}^{-1}-\Sigma_{n-1}^{-1}(x_{n}-\mu)(x_{n}-\mu)^{T}\Sigma_{n-1}^{-1})\\
 & =\Sigma_{n-1}+\frac{1}{2}a_{n-1}(\Sigma_{n-1}^{-2}(x_{n}-\mu)(x_{n}-\mu)^{T}-\Sigma_{n-1}^{-1})\\
 & =\Sigma_{n-1}+\frac{a_{n-1}}{2}\Sigma_{n-1}^{-2}((x_{n}-\mu)(x_{n}-\mu)^{T}-\Sigma_{n-1}).\tag{2 }
\end{align*}
Compare Eq.(1) and Eq.(2) and we get that 
\[
\frac{a_{n-1}}{2}\Sigma_{n-1}^{-2}=\frac{1}{n}\implies a_{n-1}=\frac{2\Sigma_{n-1}^{2}}{n}.
\]
\\

\begin{cBoxA}[Problem 2.38 - Completion the square for Gaussian bayesian update]{}
 Use the technique of completing the square for the quadratic form
in the expo- nent to derive the results (2.141) and (2.142). 
\end{cBoxA}

Recall that we are given 
\begin{align*}
f(x_{1},\dots,x_{n}\vert\mu) & =\prod_{i=1}^{n}f(x_{n}\vert\mu)=\frac{1}{(2\pi)^{n/2}\sigma^{2}}\exp\left(-\frac{1}{2\sigma^{2}}\sum_{i=1}^{n}(x_{i}-\mu)^{2}\right);\\
f(\mu\vert\mu_{0},\sigma_{0}) & =\frac{1}{(2\pi)^{1/2}\sigma_{0}^{2}}\exp\left(-\frac{1}{2\sigma_{0}^{2}}(\mu-\mu_{0})^{2}\right).
\end{align*}
Then it follows that 
\begin{align*}
f(\mu\vert x_{1},\ldots,x_{n}) & \propto f(x_{1},\ldots,x_{n}\vert\mu)f(\mu\vert\mu_{0},\sigma_{0})\\
 & \propto\exp\left(-\frac{1}{2}\left(\frac{1}{\sigma^{2}}\sum_{i=1}^{n}(x_{i}-\mu)^{2}+\frac{1}{\sigma_{0}^{2}}(\mu-\mu_{0})^{2}\right)\right)\\
 & =\exp\left(-\frac{1}{2}\left(\frac{\mu^{2}}{\sigma_{0}^{2}}-\frac{2\mu\mu_{0}}{\sigma_{0}^{2}}+\frac{\mu_{0}^{2}}{\sigma_{0}^{2}}+\frac{1}{\sigma^{2}}\sum_{i=1}^{n}x_{i}^{2}-\frac{2}{\sigma^{2}}\sum_{i=1}^{n}x_{i}\mu+\frac{1}{\sigma^{2}}\sum_{i=1}^{n}\mu^{2}\right)\right)\\
 & =\exp\left(-\frac{1}{2}\left[\mu^{2}\left(\frac{1}{\sigma_{0}^{2}}+\frac{n}{\sigma^{2}}\right)-2\mu\left(\frac{\mu_{0}}{\sigma_{0}^{2}}+\frac{1}{\sigma^{2}}\sum_{i=1}^{n}x_{i}\right)+\frac{\mu_{0}^{2}}{\sigma_{0}^{2}}+\frac{1}{\sigma^{2}}\sum_{i=1}^{n}x_{i}^{2}\right]\right)\\
 & \propto\exp\left(-\frac{1}{2}\left[\mu^{2}\left(\frac{1}{\sigma_{0}^{2}}+\frac{n}{\sigma^{2}}\right)-2\mu\left(\frac{\mu_{0}}{\sigma_{0}^{2}}+\frac{1}{\sigma^{2}}\sum_{i=1}^{n}x_{i}\right)\right]\right).
\end{align*}
Therefore, by \ref{lem: gaussian completion of squares} it follows
that $\mu\sim\mathrm{N}((\frac{1}{\sigma_{0}^{2}}+\frac{n}{\sigma^{2}})^{-1}(\frac{\mu_{0}}{\sigma_{0}^{2}}+\frac{1}{\sigma^{2}}\sum_{i=1}^{n}x_{i}),(\frac{1}{\sigma_{0}^{2}}+\frac{n}{\sigma^{2}})^{-1}).$
And we can get the desired result by noticing that 
\begin{align*}
\left(\frac{1}{\sigma_{0}^{2}}+\frac{n}{\sigma^{2}}\right)^{-1}\left(\frac{\mu_{0}}{\sigma_{0}^{2}}+\frac{1}{\sigma^{2}}\sum_{i=1}^{n}x_{i}\right) & =\frac{\sigma_{0}^{2}\sigma^{2}}{\sigma^{2}+n\sigma_{0}^{2}}\frac{\mu_{0}\sigma^{2}+\sigma_{0}\sum_{i=1}^{n}x_{i}}{\sigma_{0}^{2}\sigma^{2}}=\frac{\mu_{0}\sigma^{2}}{\sigma^{2}+n\sigma_{0}^{2}}+\frac{\sigma_{0}\sum_{i=1}^{n}x_{i}}{\sigma^{2}+n\sigma_{0}^{2}}\\
 & =\frac{\mu_{0}\sigma^{2}}{\sigma^{2}+n\sigma_{0}^{2}}+\frac{n\sigma_{0}}{\sigma^{2}+n\sigma_{0}^{2}}\mu_{ML},
\end{align*}
and that 
\[
\left(\frac{1}{\sigma_{0}^{2}}+\frac{n}{\sigma^{2}}\right)^{-1}=\frac{\sigma_{0}^{2}\sigma^{2}}{\sigma^{2}+n\sigma_{0}^{2}}.
\]
\\

\begin{cBoxA}[Problem 2.39 - Sequential bayesian for univariate gaussian]{}
Starting from the results (2.141) and (2.142) for the posterior distribution
of the mean of a Gaussian random variable, dissect out the contributions
from the first $n-1$ data points and hence obtain expressions for
the sequential update of $\mu_{n}$ and $\sigma_{n}^{2}$ . Now derive
the same results starting from the posterior distribution $f(\mu\vert x_{1},\ldots,x_{n-1})=\mathrm{N}(\mu\vert\mu_{n-1},\sigma_{n-1}^{2})$
and multiplying by the likelihood function $f(x_{n}\vert\mu)=\mathrm{N}(x_{n}\vert\mu,\sigma^{2})$
and then completing the square and normalizing to obtain the posterior
distribution after $n$ observations.
\end{cBoxA}

Recall that by Eq.(2.141) and Eq.(2.142), we have 
\[
\mu_{n}=\frac{\sigma^{2}}{n\sigma_{0}^{2}+\sigma^{2}}\mu_{0}+\frac{n\sigma_{0}^{2}}{n\sigma_{0}^{2}+\sigma^{2}}\mu_{ML}\ \text{and }\frac{1}{\sigma_{n}^{2}}=\frac{1}{\sigma_{0}^{2}}+\frac{n}{\sigma^{2}}=\frac{\sigma^{2}+n\sigma_{0}^{2}}{\sigma_{0}^{2}\sigma^{2}}.
\]
First, we note that since for any $i=1,...,n$, $\frac{1}{\sigma_{i}^{2}}=\frac{1}{\sigma_{i}^{2}}+\frac{i}{\sigma^{2}},$
then it follows that 
\[
\frac{1}{\sigma_{n}^{2}}=\frac{1}{\sigma_{0}^{2}}+\frac{n-1}{\sigma^{2}}+\frac{1}{\sigma^{2}}=\frac{1}{\sigma_{n-1}^{2}}+\frac{1}{\sigma^{2}}=\frac{\sigma^{2}+\sigma_{n-1}^{2}}{\sigma_{n-1}^{2}\sigma^{2}}.
\]
Next, note that 
\[
\frac{\sigma_{n}^{2}}{\sigma_{n-1}^{2}}=\frac{\sigma^{2}+(n-1)\sigma_{0}^{2}}{\sigma_{0}^{2}\sigma^{2}}\frac{\sigma_{0}^{2}\sigma^{2}}{\sigma^{2}+n\sigma_{0}^{2}}=\frac{\sigma^{2}+(n-1)\sigma_{0}^{2}}{\sigma^{2}+n\sigma_{0}^{2}}.
\]
Hence, it follows that 
\begin{align*}
\mu_{n-1} & =\frac{1}{n\sigma_{0}^{2}+\sigma^{2}}(\sigma^{2}\mu_{0}+n\sigma_{0}^{2}\mu_{ML})=\frac{1}{n\sigma_{0}^{2}+\sigma^{2}}\left(\sigma^{2}\mu_{0}+\sigma_{0}^{2}\sum_{i=1}^{n}x_{i}\right)\\
 & =\frac{1}{n\sigma_{0}^{2}+\sigma^{2}}\left(\sigma^{2}\mu_{0}+\sigma_{0}^{2}\sum_{i=1}^{n-1}x_{i}+\sigma_{0}^{2}x_{n}\right)\\
 & =\frac{(n-1)\sigma_{0}^{2}+\sigma^{2}}{n\sigma_{0}^{2}+\sigma^{2}}\frac{1}{(n-1)\sigma_{0}^{2}+\sigma^{2}}\left(\sigma^{2}\mu_{0}+\sigma_{0}^{2}\sum_{i=1}^{n-1}x_{i}\right)+\frac{\sigma_{0}^{2}x_{n}}{n\sigma_{0}^{2}+\sigma^{2}}\\
 & =\frac{\sigma_{n}^{2}}{\sigma_{n-1}^{2}}\underbrace{\frac{1}{(n-1)\sigma_{0}^{2}+\sigma^{2}}\left(\sigma^{2}\mu_{0}+\sigma_{0}^{2}\sum_{i=1}^{n-1}x_{i}\right)}_{:=\mu_{n-1}}+\frac{\sigma_{0}^{2}x_{n}}{n\sigma_{0}^{2}+\sigma^{2}}\\
 & =\frac{\sigma_{n}^{2}}{\sigma_{n-1}^{2}}\mu_{n-1}+\frac{\sigma^{2}\sigma_{0}^{2}}{n\sigma_{0}^{2}+\sigma^{2}}\frac{x_{n}}{\sigma^{2}}\\
 & =\frac{\sigma_{n}^{2}}{\sigma_{n-1}^{2}}\mu_{n-1}+\frac{\sigma_{n}x_{n}}{\sigma^{2}}.
\end{align*}
On the other hand, using the sequential update formula we have that
\begin{align*}
f(\mu|x_{1},\ldots,x_{n}) & =f(\mu)f(x_{n}\vert\mu)=f(\mu\vert x_{1},\ldots,x_{n-1})f(x_{n}\vert\mu)=f_{\mathrm{N}(\mu_{n-1},\sigma_{n-1}^{2})}(\mu)f_{\mathrm{N}(\mu,\sigma^{2})}(x_{n})\tag{1}\\
 & \propto\exp\left(-\frac{1}{2\sigma_{n-1}^{2}}(\mu-\mu_{n-1})^{2}-\frac{1}{2\sigma^{2}}(x_{n}-\mu)^{2}\right)\\
 & =\exp\left(-\frac{1}{2}\left(\frac{\mu^{2}-2\mu_{n-1}\mu+\mu_{n-1}^{2}}{\sigma_{n-1}^{2}}+\frac{x_{n}^{2}-2\mu x_{n}+\mu^{2}}{\sigma^{2}}\right)\right)\\
 & =\exp\left(-\frac{1}{2}\left(\mu^{2}\left(\frac{1}{\sigma_{n-1}^{2}}+\frac{1}{\sigma^{2}}\right)-2\mu\left(\frac{\mu_{n-1}}{\sigma_{n-1}^{2}}+\frac{x_{n}}{\sigma^{2}}\right)\right)-\frac{\mu_{n-1}^{2}}{2\sigma_{n-1}^{2}}-\frac{x_{n}^{2}}{2\sigma^{2}}\right)\\
 & \propto\exp\left(-\frac{1}{2}\left(\mu^{2}\left(\frac{1}{\sigma_{n-1}^{2}}+\frac{1}{\sigma^{2}}\right)-2\mu\left(\frac{\mu_{n-1}}{\sigma_{n-1}^{2}}+\frac{x_{n}}{\sigma^{2}}\right)\right)\right).
\end{align*}
Therefore, by \ref{lem: gaussian completion of squares}, $\mu_{n}|x_{1},...,x_{n}\sim\mathrm{N}(\mu_{n}=(\frac{1}{\sigma_{n-1}^{2}}+\frac{1}{\sigma^{2}})^{-1}(\frac{\mu_{n-1}}{\sigma_{n-1}^{2}}+\frac{x_{n}}{\sigma^{2}}),\sigma_{n}=(\frac{1}{\sigma_{n-1}^{2}}+\frac{1}{\sigma^{2}})^{-1})$.
We can conclude by noting 
\begin{align*}
\mu_{n} & =\left(\frac{\sigma_{n-1}^{2}\sigma^{2}}{\sigma_{n-1}^{2}+\sigma^{2}}\right)\left(\frac{\sigma^{2}\mu_{n-1}+\sigma_{n-1}^{2}x_{n}}{\sigma_{n-1}^{2}\sigma^{2}}\right)=\frac{\sigma^{2}\mu_{n-1}+\sigma_{n-1}^{2}x_{n}}{\sigma_{n-1}^{2}+\sigma^{2}}=\frac{\mu_{n-1}}{\sigma_{n-1}^{2}}+\frac{\sigma_{n}x_{n}}{\sigma^{2}}.\\
\frac{1}{\sigma_{n}} & =\frac{1}{\sigma_{n-1}^{2}}+\frac{1}{\sigma^{2}}=\frac{\sigma_{n-1}^{2}+\sigma^{2}}{\sigma_{n-1}^{2}\sigma^{2}},
\end{align*}
which is the same as what we derived previously. \\

\begin{cBoxA}[Problem 2.40 - Bayesian update for multivariate gaussian ]{}
 Consider a $D$ dimensional Gaussian random variable $x$ with distribution
$\mathrm{N}(x\vert\mu,\Sigma)$ in which the covariance $\Sigma$
is known and for which we wish to infer the mean $\mu$ from a set
of observations $X=(x_{1},\ldots,x_{n}).$ Given a prior distribution
$f(\mu)=\mathrm{N}(\mu\vert\mu_{0},\Sigma_{0})$, find the corresponding
posterior distribution $f(\mu\vert X)$. 
\end{cBoxA}

Note that 
\begin{align*}
f(\mu\vert x_{1},...,x_{n}) & \propto f(\mu\vert\mu_{0},\Sigma_{0})\prod_{i=1}^{n}f(x_{i}\vert\mu,\Sigma)\\
 & \propto\exp\left(-\frac{1}{2}(\mu-\mu_{0})^{T}\Sigma_{0}^{-1}(\mu-\mu_{0})-\frac{1}{2}\sum_{i=1}^{n}(x_{i}-\mu)^{T}\Sigma^{-1}(x_{i}-\mu)\right)\\
 & =\exp\left(-\frac{1}{2}\left(\mu^{T}\Sigma_{0}^{-1}u-2\mu^{T}\Sigma_{0}^{-1}\mu_{0}+\mu_{0}^{T}\Sigma_{0}^{-1}\mu_{0}+\sum_{i=1}^{n}(x_{i}^{T}\Sigma^{-1}x_{i}-2\mu^{T}\Sigma^{-1}x_{i}+\mu^{T}\Sigma^{-1}\mu)\right)\right)\\
 & =\exp\left(-\frac{1}{2}\left(\mu^{T}(\Sigma_{0}^{-1}+n\Sigma^{-1})\mu-2\mu^{T}\bigg(\Sigma_{0}^{-1}\mu_{0}-\Sigma^{-1}\sum_{i=1}^{n}x_{i}\bigg)+\mu_{0}^{T}\Sigma_{0}^{-1}\mu_{0}+\sum_{i=1}^{n}(x_{i}^{T}\Sigma^{-1}x_{i})\right)\right).
\end{align*}
Therefore, it follows from \ref{lem: gaussian completion of squares}
that 
\begin{align*}
\mu\vert x_{1},\ldots,x_{n} & \sim\mathrm{MVN}\bigg((\Sigma_{0}^{-1}+n\Sigma^{-1})^{-1}\bigg(\Sigma_{0}^{-1}\mu_{0}-\Sigma^{-1}\sum_{i=1}^{n}x_{i}\bigg),(\Sigma_{0}^{-1}+n\Sigma^{-1})^{-1}\bigg)\\
 & \sim\mathrm{MVN}\left((\Sigma_{0}^{-1}+n\Sigma^{-1})^{-1}(\Sigma_{0}^{-1}\mu_{0}-n\Sigma^{-1}\mu_{ML}),(\Sigma_{0}^{-1}+n\Sigma^{-1})^{-1}\right),
\end{align*}
where $\mu_{ML}=\frac{1}{n}\sum_{i=1}^{n}x_{i}.$\\

\begin{cBoxA}[Problem 2.41 - Gamma density is normalized]{}
 Use the definition of the gamma function (1.141) to show that the
gamma distribution (2.146) is normalized. 
\end{cBoxA}

Recall that the Gamma density with parameter $a,b$ is given by 
\[
f(x)=\frac{1}{\Gamma(a)}b^{a}x^{a-1}e^{-bx}\mathbbm{1}_{\{x\geq0\}}.
\]
Therefore, we have by \ref{thm: transformation thm}
\begin{align*}
\int_{[0,\infty)}f(x)dx & =\int_{[0,\infty)}f(T(y))\left|\det J_{T}\right|dy\tag{where \ensuremath{T(y)=\frac{y}{b}}}\\
 & =\frac{1}{\Gamma(a)}\int_{0}^{\infty}b^{a}\frac{y^{a-1}}{b^{a-1}}e^{-b\frac{1}{b}y}\cdot\frac{1}{b}dy=\frac{1}{\Gamma(a)}\underbrace{\int_{0}^{\infty}y^{a-1}e^{-y}dy}_{=\Gamma(a)}=1.
\end{align*}
\\

\begin{cBoxA}[Problem 2.42 - Gamma distribution's mean, mode, variance]{}
 Evaluate the mean, variance, and mode of the gamma distribution
(2.146).
\end{cBoxA}
 

To find the mean, note that 
\begin{align*}
\E[X] & =\frac{1}{\Gamma(a)}\int_{0}^{\infty}xb^{a}x^{a-1}e^{-bx}dx=\frac{1}{\Gamma(a)}\int_{0}^{\infty}b^{a}x^{a}e^{-bx}dx=\frac{1}{\Gamma(a)}\int_{0}^{\infty}b^{a}\frac{y^{a}}{b^{a}}e^{-y}\frac{1}{b}dy\\
 & =\frac{1}{b\Gamma(a)}\int_{0}^{\infty}y^{a+1-1}e^{-y}dy=\frac{\Gamma(a+1)}{b\Gamma(a)}=\frac{a}{b}.
\end{align*}
To find the variance, we first note find 
\begin{align*}
\E[X^{2}] & =\frac{1}{\Gamma(a)}\int_{0}^{\infty}x^{2}b^{a}x^{a-1}e^{-bx}dx=\frac{1}{\Gamma(a)}\int_{0}^{\infty}x^{a+1}b^{a}e^{-bx}dx=\frac{1}{\Gamma(a)}\int_{0}^{\infty}\frac{y^{a+1}}{b^{a+1}}b^{a}e^{-y}\frac{1}{b}dy\\
 & =\frac{1}{b^{2}\Gamma(a)}\int_{0}^{\infty}y^{a+2-1}e^{-y}dy=\frac{\Gamma(a+2)}{b^{2}\Gamma(a)}=\frac{(a+1)a}{b^{2}}.
\end{align*}
Hence, we have 
\[
\mathrm{Var}[X]=\E[X^{2}]-(\E[X])^{2}=\frac{a^{2}+a}{b^{2}}-\frac{a^{2}}{b^{2}}=\frac{a}{b^{2}}.
\]
To find the mode, we differentiate the density function and set its
derivative to 0: 
\[
\frac{d}{dx}f(x)=\frac{d}{dx}\left(\frac{1}{\Gamma(a)}x^{a-1}b^{a}e^{-bx}\right)=(a-1)x^{a-2}e^{-bx}-bx^{a-1}e^{-bx}=0.
\]
Rearranging it a bit we get $x^{a-1}\left(\frac{a-1}{x}-b\right)e^{-bx}=0$.
Since $x>0$ by assumption, we can further reduce to $\frac{a-1}{x}-b=0,$
which implies that $x=\frac{a-1}{b}.$\\

\begin{cBoxA}[Problem 2.43 - Generalized univariate Gaussian distribution ]{}
The following distribution 
\[
p(x\vert\sigma^{2},q)=\frac{q}{2(2\sigma^{2})^{1/q}\Gamma(1/q)}\exp\left(-\frac{\left|x\right|^{q}}{2\sigma^{2}}\right)\tag{1}
\]
is a generalization of the univariate Gaussian distribution. Show
that this distribution is normalized so that 
\[
\int_{-\infty}^{\infty}p(x\vert\sigma^{2},q)dx=1
\]
and that it reduces to the Gaussian when $q=2.$ Consider a regression
model in which the target variable is given by $t=y(x,w)+\varepsilon$
and $\varepsilon$ is a random noise varaible drawn from distribution
Eq.(1). Show that the log-likelihood function over $w$ and $\sigma^{2}$,
for an observed date set of input vectors $X=(x_{1},\ldots,x_{n})$
and corresponding target variables $t=(t_{1},\dots,t_{n})$ is given
by 
\[
\ln p(t\vert X,w,\sigma^{2})=-\frac{1}{2\sigma^{2}}\sum_{i=1}^{n}\left|y(x_{n},w)-t_{n}\right|^{q}-\frac{n}{q}\ln(2\sigma^{2})+\mathrm{const.}
\]
\end{cBoxA}

Note that by change of variable of $u=\frac{x^{q}}{2\sigma^{2}},$
which implies that $x=(2\sigma^{2}u)^{1/q}$ and $du=\frac{qx^{q-1}}{2\sigma^{2}}dx$,
we have

\begin{align*}
\int f(x\vert\sigma^{2},q)dx & =\frac{q}{2(2\sigma^{2})^{1/q}\Gamma(1/q)}\int_{\mathbb{R}}\exp\left(-\frac{\left|x\right|^{q}}{2\sigma^{2}}\right)dx\\
 & =\frac{q}{(2\sigma^{2})^{1/q}\Gamma(1/q)}\int_{0}^{\infty}\exp\left(-\frac{x^{q}}{2\sigma^{2}}\right)dx\\
 & =\frac{q}{(2\sigma^{2})^{1/q}\Gamma(1/q)}\int_{0}^{\infty}\exp(-u)\frac{2\sigma^{2}}{qx^{q-1}}du\\
 & =\frac{q}{(2\sigma^{2})^{1/q}\Gamma(1/q)}\int_{0}^{\infty}\exp(-u)\frac{2\sigma^{2}}{q((2\sigma^{2}u)^{1/q})^{q-1}}du\\
 & =\frac{2\sigma^{2}}{(2\sigma^{2})^{1/q}\Gamma(1/q)}\cdot(2\sigma^{2})^{1/q-1}\cdot\int_{0}^{\infty}\exp(-u)u^{1/q-1}du\\
 & =(2\sigma^{2})^{1-1/q}(2\sigma^{2})^{1/q-1}\frac{1}{\Gamma(1/q)}\Gamma(1/q)\\
 & =1.
\end{align*}
Next, we show that this distribution recovers the normal gaussian
distribution when $q=2.$ But first, we need a lemma. 
\begin{lem}
For all $s\in\mathbb{C},$ $\Gamma(s)\Gamma(1-s)=\frac{\pi}{\sin\pi s}.$\label{lem: gamma function reflection property}
\end{lem}
\noindent Hence, we let $s=\frac{1}{2}$ in the previous lemma; and
we will get $\Gamma(1/2)^{2}=\pi,$ which implies that $\Gamma(1/2)=\sqrt{\pi}.$
Therefore, we plugin and get 
\[
f\left(x\vert\sigma^{2},\frac{1}{2}\right)=\frac{2}{2(2\sigma^{2})^{1/2}\sqrt{\pi}}\exp\left(-\frac{x^{2}}{2\sigma^{2}}\right)=\frac{1}{(2\pi\sigma^{2})^{1/2}}\exp\left(-\frac{x^{2}}{2\sigma^{2}}\right).
\]

For the regression model, $t=y(x,w)+\varepsilon$ where $\varepsilon$
is a random noise variable drawn from the generalized univariate gaussian
distribution, we note that $t-y(x,w)=\varepsilon$, which is generalized
univariate gaussian. Hence, the likelihood function 

\[
\P(t\vert x,w,\sigma^{2})=\prod_{i=1}^{n}\frac{q}{2(2\sigma^{2})^{1/q}\Gamma(1/q)}\exp\left(-\frac{\left|t_{i}-y(x_{i},w\right|^{q}}{2\sigma^{2}}\right).
\]
Then we take the logarithm to get the likelihood function: 
\begin{align*}
\ell(t\vert x,w,\sigma^{2}) & =\log\P(t\vert x,w,\sigma^{2})=\sum_{i=1}^{n}\log\left(\frac{q}{2(2\sigma^{2})^{1/q}\Gamma(1/q)}\exp\left(-\frac{\left|t_{i}-y(x_{i},w)\right|^{q}}{2\sigma^{2}}\right)\right)\\
 & =\sum_{i=1}^{n}\frac{\left|t_{i}-y(x_{i},w)\right|^{q}}{2\sigma^{2}}-\log(2\sigma^{2})+\mathrm{const}.
\end{align*}
\\

\begin{cBoxA}[Problem 2.44 - Posterior of Gaussian with Gauss-Gamma is Gauss Gamma]{}
Consider a univariate Gaussian distribution $N(x|\mu,\lambda^{-1})$
having conjugate Gaussian-gamma prior given by (2.154), and a data
set $x=\{x_{1},\cdots,x_{n}\}$ of i.i.d observations. Show that the
posterior distribution is also a Gaussian-gamma distribution of the
same functional form as the prior, and write down expressions for
the parameters of this posterior distribution
\end{cBoxA}

First, we note that by Eq.(2.152) in the book, 
\begin{align*}
\P(X\vert\mu,\lambda) & =\prod_{i=1}^{n}\left(\frac{\lambda}{2\pi}\right)^{1/2}\exp\left(-\frac{\lambda}{2}(x_{n}-\mu)^{2}\right)\propto\left[\lambda^{1/2}\exp\left(-\frac{\lambda\mu^{2}}{2}\right)\right]^{n}\exp\left(\lambda\mu\sum_{i=1}^{n}x_{i}-\frac{\lambda}{2}\sum_{i=1}^{n}x_{i}^{2}\right).
\end{align*}
And by assumption, 
\[
\P(\mu,\lambda)=\mathrm{N}(\mu\vert\mu_{0},(\beta\lambda)^{-1})\mathrm{Gamma}(\lambda\vert a,b)=(\beta\lambda)^{1/2}\exp\left(-\frac{\beta\lambda}{2}(\mu^{2}-\mu_{0})^{2}\right)\lambda^{a-1}\exp(-b\lambda).
\]
Then it follows that 
\begin{align*}
\P(\mu,\lambda\vert X) & \propto\P(X\vert\mu,\lambda)\P(\mu,\lambda)\\
 & \propto\left[\lambda^{\frac{1}{2}}\exp\left(-\frac{\lambda\mu^{2}}{2}\right)\right]^{n}\exp\left(\lambda\mu\sum_{i=1}^{n}x_{i}-\frac{\lambda}{2}\sum_{i=1}^{n}x_{i}^{2}\right)(\beta\lambda)^{\frac{1}{2}}\exp\left(-\frac{\beta\lambda}{2}(\mu^{2}-\mu_{0})^{2}\right)\lambda^{a-1}\exp(-b\lambda)\\
 & =\lambda^{\frac{n}{2}}(\beta\lambda)^{\frac{1}{2}}\lambda^{a-1}\exp\bigg(-\frac{n\lambda\mu^{2}}{2}+\lambda\mu\sum_{i=1}^{n}x_{i}-\frac{\lambda}{2}\sum_{i=1}^{n}x_{i}^{2}-\frac{\beta\lambda}{2}(\mu^{2}-2\mu\mu_{0}+\mu_{0})-b\lambda\bigg)\\
 & =\beta^{\frac{1}{2}}\lambda^{\frac{1}{2}+\frac{n}{2}+a-1}\exp\bigg(\mu^{2}\left(-\frac{n\lambda}{2}-\frac{\beta\lambda}{2}\right)+\mu\bigg(\beta\lambda\mu_{0}+\lambda\sum_{i=1}^{n}x_{i}\bigg)-b\lambda-\frac{\lambda}{2}\sum_{i=1}^{n}x_{i}^{2}-\frac{\beta\lambda}{2}\mu_{0}^{2}\bigg)\\
 & =\beta^{\frac{1}{2}}\lambda^{\frac{1}{2}+\frac{n}{2}+a-1}\exp\bigg(-\frac{\lambda(n+\beta)}{2}\mu^{2}+\mu\lambda\bigg(\beta\mu_{0}+\sum_{i=1}^{n}x_{i}\bigg)-\lambda\bigg(b+\frac{1}{2}\sum_{i=1}^{n}x_{i}^{2}+\frac{\beta}{2}\mu_{0}^{2}\bigg)\bigg)\\
 & =\beta^{\frac{1}{2}}\lambda^{\frac{1}{2}+\frac{n}{2}+a-1}\exp\bigg(-\frac{\lambda(n+\beta)}{2}\bigg(\mu^{2}-\frac{\beta\mu_{0}+\sum_{i=1}^{n}x_{i}}{n+\beta}\bigg)^{2}-\lambda\bigg(b+\frac{1}{2}\sum_{i=1}^{n}x_{i}^{2}+\frac{\beta}{2}\mu_{0}^{2}-\frac{(\beta\mu_{0}+\sum_{i=1}^{n}x_{i})^{2}}{2(n+\beta)}\bigg)\bigg)\\
 & =(\lambda(n+\beta))^{\frac{1}{2}}\exp\bigg(-\frac{\lambda(n+\beta)}{2}\bigg(\mu^{2}-\frac{\beta\mu_{0}+\sum_{i=1}^{n}x_{i}}{n+\beta}\bigg)^{2}\bigg)\\
 & \ \ \ \ \ \ \ \ \ \ \ \ \ \ \ \ \ \ \ \ \ \ \ \times\lambda^{a+\frac{n}{2}-1}\exp\bigg(-\bigg(b+\frac{1}{2}\sum_{i=1}^{n}x_{n}^{2}+\frac{\beta}{2}\mu_{0}^{2}-\frac{(\beta\mu_{0}+\sum_{i=1}^{n}x_{i})^{2}}{2(n+\beta)}\bigg)\lambda\bigg).
\end{align*}
By comparing coefficients, we get that 
\[
(\lambda(n+\beta))^{\frac{1}{2}}\exp\bigg(-\frac{\lambda(n+\beta)}{2}\bigg(\mu^{2}-\frac{\beta\mu_{0}+\sum_{i=1}^{n}x_{i}}{n+\beta}\bigg)^{2}\bigg)\sim\mathrm{N}(\mu\vert\mu_{n},\lambda_{n}^{-1}),
\]
where 
\[
\mu_{n}=\frac{\beta\mu_{0}+\sum_{i=1}^{n}x_{i}}{n+\beta},\ \ \lambda_{n}=\frac{1}{\lambda(n+\beta)}.
\]
And that 
\[
\lambda^{a+\frac{n}{2}-1}\exp\bigg(-\bigg(b+\frac{1}{2}\sum_{i=1}^{n}x_{n}^{2}+\frac{\beta}{2}\mu_{0}^{2}-\frac{(\beta\mu_{0}+\sum_{i=1}^{n}x_{i})^{2}}{2(n+\beta)}\bigg)\lambda\bigg)\sim\mathrm{Gamma}(\lambda\vert a_{n},b_{n}),
\]
where 
\[
a_{n}=a+\frac{n}{2},\ \ b_{b}=b+\frac{1}{2}\sum_{i=1}^{n}x_{i}^{2}+\frac{\beta}{2}\mu_{0}^{2}-\frac{(\beta\mu_{0}+\sum_{i=1}^{n}x_{n})^{2}}{2(n+\beta)}.
\]
\\

\begin{cBoxA}[Problem 2.45 - Wishart distribution is conjugate prior of Gaussian
precision]{}
 Verify that the Wishart distribution defined by (2.155) is indeed
a conjugate prior for the precision matrix of multivariate Gaussian. 
\end{cBoxA}

Note that by Bayes update formula, 
\begin{align*}
f(\Lambda\vert X_{1},\ldots,X_{n}) & \propto f(X_{1},\ldots,X_{n}\vert\Lambda)f(\Lambda)=\prod_{i=1}^{n}f_{\mathrm{N}(\mu_{0},\Lambda^{-1})}(x_{i})f_{\mathrm{Wishart}(W,\nu)}(\Lambda)\\
 & =\bigg(\prod_{i=1}^{n}\frac{\left|\det\Lambda\right|{}^{1/2}}{(2\pi)^{d/2}}\bigg)\exp\bigg(-\frac{1}{2}\sum_{i=1}^{n}(x_{i}-\mu)^{T}\Lambda(x_{i}-\mu)\bigg)\cdot B\left|\det\Lambda\right|^{(v-d-1)/2}\exp\left(-\frac{1}{2}\mathrm{tr}(W^{-1}\Lambda)\right)\\
 & =\bigg(\prod_{i=1}^{n}\frac{\left|\det\Lambda\right|{}^{1/2}}{(2\pi)^{d/2}}\bigg)\exp\bigg(-\frac{1}{2}\sum_{i=1}^{n}(x_{i}-\mu)^{T}\Lambda(x_{i}-\mu)\bigg)\cdot B\left|\det\Lambda\right|^{(v-d-1)/2}\exp\left(-\frac{1}{2}\mathrm{tr}(W^{-1}\Lambda)\right)\\
 & \propto\left|\det\Lambda\right|^{n/2+(v-d-1)/2}\exp\bigg(-\frac{1}{2}\sum_{i=1}^{n}\mathrm{tr}((x_{i}-\mu)^{T}\Lambda(x_{i}-\mu)-\frac{1}{2}\mathrm{tr}(\Lambda W^{-1})\bigg)\\
 & =\left|\det\Lambda\right|^{(n+v-d-1)/2}\exp\bigg(-\frac{1}{2}\mathrm{tr}\bigg(\Lambda\bigg(\sum_{i=1}^{n}(x_{i}-\mu)(x_{i}-\mu)^{T}+W\bigg)\bigg)\bigg)\\
 & \propto f_{\mathrm{Wishart}(\sum_{i=1}^{n}(x_{i}-\mu)(x_{i}-\mu)^{T}+W,n+v)}(\Lambda).
\end{align*}
Therefore, we have shown that Wishart distribution is a conjugate
prior for the precision matrix of multivariate Gaussian. 

\begin{cBoxA}[Problem 2.46 - ]{}
\end{cBoxA}

\begin{cBoxA}[Problem 2.47 - Student $t$-distribution converges to Gaussian]{}
 Show that in the limit $v\rightarrow\infty$, the $t$-distribution
(2.159) becomes a Gaussian. Hint: ignore the normalization coefficient,
and simply look at the dependence on $x$.
\end{cBoxA}

Before, we go into the solution, we need an asymptotic approximation
of ratio of Gamma function. We provide this approximation and give
a proof as follows.
\begin{lem}
For any $\alpha\in\mathbb{R},$ we have 
\[
\lim_{x\rightarrow\infty}\frac{\Gamma(x+\alpha)}{\Gamma(x)x^{\alpha}}=1.
\]
\label{lem: gamma stirling approximation}
\end{lem}
%
\begin{proof}
First, we note that by OEIS A046968 and A046969 we have that for $z\in\mathbb{R}$
\begin{align*}
\log\Gamma(z) & =z\log z-z+\frac{1}{2}(\log2\pi+\log\frac{1}{z})+\frac{1}{12z}-\frac{1}{360z^{3}}+\frac{1}{1260z^{5}}-\cdots\\
 & =z(\log z-1)+\frac{1}{2}(\log2\pi+\log\frac{1}{z})+\frac{1}{12z}+O\bigg(\frac{1}{z^{3}}\bigg).\tag{1 }
\end{align*}
Hence, it suffices to prove that $\log\frac{\Gamma(x+\alpha)}{\Gamma(x)x^{\alpha}}\rightarrow0$
as $x\rightarrow\infty.$ Note that 
\begin{align*}
\log\frac{\Gamma(x+\alpha)}{\Gamma(x)x^{\alpha}} & =\log\Gamma(x+\alpha)-\log\Gamma(x)-\alpha\log x\\
 & =(x+\alpha)(\log(x+\alpha)-1)+\frac{1}{2}\bigg(\log2\pi+\log\frac{1}{x+\alpha}\bigg)+\frac{1}{12(x+\alpha)}+O\bigg(\frac{1}{(x+\alpha)^{3}}\bigg)\\
 & \ \ -x\log x-x-\frac{1}{2}\bigg(\log2\pi+\log\frac{1}{\alpha}\bigg)-\frac{1}{12x}-O\bigg(\frac{1}{x^{3}}\bigg)-\alpha\log x\\
 & =x\log\bigg(1+\frac{\alpha}{x}\bigg)-\frac{1}{2}\log\bigg(1+\frac{\alpha}{x}\bigg)+\alpha\log\bigg(1+\frac{\alpha}{x}\bigg)+\frac{1}{12}\bigg(\frac{1}{x+\alpha}-\frac{1}{x}\bigg)+O\bigg(\frac{1}{x^{3}}\bigg).\tag{2}
\end{align*}
Note that $\log(1+\frac{\alpha}{x})=\frac{\alpha}{x}+O(\frac{1}{x^{2}})=\frac{\alpha}{x}-\frac{\alpha^{2}}{2x^{2}}+O(\frac{1}{x^{3}})$
via Taylor expansion. So substituting back we have 

\begin{align*}
\mathrm{Eq}.(2) & =x\bigg(\frac{\alpha}{x}-\frac{\alpha^{2}}{2x^{2}}+O\bigg(\frac{1}{x^{3}}\bigg)\bigg)-\frac{1}{2}\biggp{\frac{\alpha}{x}+O\biggp{\frac{1}{x^{2}}}}+O\biggp{\frac{1}{x^{2}}}-\alpha\biggp{\frac{\alpha}{x}+O\biggp{\frac{1}{x^{2}}}}-\alpha+O\biggp{\frac{1}{x^{3}}}\\
 & =-\frac{\alpha^{2}}{2x}+\frac{\alpha^{2}}{x}-\frac{\alpha}{2x}+O\biggp{\frac{1}{x^{3}}}+O\biggp{\frac{1}{x^{2}}}\\
 & =\frac{\alpha(\alpha-1)}{2x}+O\biggp{\frac{1}{x^{2}}},
\end{align*}
where the last equality follows from the assumption $x\geq1$ and
the fact that $O(f(x))+O(g(x))=O(\max(f(x)+g(x))$. Now we take the
limit of $x\rightarrow\infty,$ we get that 
\[
\lim_{x\rightarrow\infty}\log\frac{\Gamma(x+\alpha)}{\Gamma(x)x^{\alpha}}=\lim_{x\rightarrow\infty}\left[\frac{\alpha(\alpha-1)}{2x}+O\biggp{\frac{1}{x^{2}}}\right]=0\implies\lim_{x\rightarrow\infty}\frac{\Gamma(x+\alpha)}{\Gamma(x)x^{\alpha}}=1.
\]
\end{proof}
Now we observe that 
\begin{align*}
f_{\mathrm{St}(x\vert\mu,\lambda,v)}(x) & =\frac{\Gamma(\frac{v}{2}+\frac{1}{2})}{\Gamma(\frac{v}{2})}\biggp{\frac{\lambda}{\pi v}}^{1/2}\left[1+\frac{\lambda(x-\mu)^{2}}{v}\right]^{-v/2-1/2}\\
 & =\frac{\Gamma(\frac{v}{2}+\frac{1}{2})}{\Gamma(\frac{v}{2})(\frac{v}{2})^{1/2}}\underbrace{\bigg(\frac{v}{2}\bigg)^{1/2}\biggp{\frac{\lambda}{\pi v}}^{1/2}\exp\left[-\frac{v+1}{2}\log\biggp{1+\frac{\lambda(x-\mu)^{2}}{v}}\right]}_{:=\mathcal{H}_{1}(x)}.
\end{align*}
Now note that by \ref{lem: gamma stirling approximation}, $\frac{\Gamma(\frac{v}{2}+\frac{1}{2})}{\Gamma(\frac{v}{2})(\frac{v}{2})^{1/2}}\rightarrow1$
as $v\rightarrow\infty.$ And by Taylor expansion of logarithm, 
\begin{align*}
\mathcal{H}_{1}(x) & =\frac{1}{(2\pi\lambda^{-1})^{1/2}}\exp\left[-\frac{v+1}{2}\biggp{\frac{\lambda(x-\mu)^{2}}{v}+O\biggp{\frac{1}{v^{2}}}}\right]\\
 & =\frac{1}{(2\pi\lambda^{-1})^{1/2}}\exp\left[-\frac{v+1}{v}\frac{(x-\mu)^{2}}{2\lambda^{-1}}+O\biggp{\frac{1}{v}}\right]\\
 & \xrightarrow{v\rightarrow\infty}\frac{1}{(2\pi\lambda^{-1})^{1/2}}\exp\left(\frac{(x-\mu)^{2}}{2\lambda^{-1}}\right).
\end{align*}
Hence, combined together we have that $f_{\mathrm{St}(x\vert\mu,\lambda,v)}(x)\xrightarrow{v\rightarrow\infty}f_{\mathrm{N}(\mu,\lambda^{-1})}(x).$ 
\begin{rem}
The argument we used in solution is almost perfectly rigorous except
a subtle point that we need to show that convergence in density functions
implies convergence in distribution. This is actually a quite famous
result, called Scheffe's lemma. We state without a proof the theorem
statement below. A proof can be found in the reference provided. 
\end{rem}
\begin{lem}[{\cite[Lemma 8.2.1]{resnickProbabilityPath2014}}]
Let $\{X_{n}\}_{n\in\mathbb{N}}$ be absolutely continuous random
variables with densities $f_{X_{n}}$, such that $f_{X_{n}}(x)\rightarrow f(x)$
almost everywhere, where $f$ is the density of the absolutely continuous
random variable $X$. Then $X_{n}$ converges to $X$ in total variation,
and therefore, also in distribution. 
\end{lem}
%
\begin{cBoxA}[qua]{}
\end{cBoxA}

